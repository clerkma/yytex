% Plain TeX test file to show all characters in a font
% Customize by changing the fonts selected in the first few lines

\nopagenumbers

% Following defines the font that you want to test (by its TFM file name):

% \font\test=cmr10
% \font\test=sy
\font\test=tir
\font\test=lbi at 20pt

% Following defines the font used for printing the character number:

% \font\numfont=cmssbx10 at 5pt
\font\numfont=hvb at 5pt

% Following defines the font used for printing the \fontdimen information

% \font\dimenfont=cmss10
\font\dimenfont=hv

\newcount\k\newcount\n\newcount\m

% Print 16 x 16 table with character code numbers and characters

\numfont
\k=0\n=0\m=0
\noindent
{\loop\ifnum\n<16
{\loop\ifnum\m<16
\the\k\char58\space
{\test\char\k},\space
\global\advance\k by1
\advance\m by1\repeat}
\hfill\break
\m=0\vskip 1mm\noindent
\advance\n by1\repeat}

% Show fontdimensions of the selected font

% \test
% \n=1
% \noindent
% {\dimenfont
% {\loop\ifnum\n<8
% \the\n: \the\fontdimen\n\test,\hfill\break
% \advance\n by1\repeat}
% }

\dimenfont

\settabs 6 \columns
\+slant:& \the\fontdimen1\test,\cr
\+space:& \the\fontdimen2\test,\cr
\+stretch:& \the\fontdimen3\test,\cr
\+shrink:& \the\fontdimen4\test,\cr
\+xheight:& \the\fontdimen5\test,\cr
\+emquad:& \the\fontdimen6\test,\cr
\+extra:& \the\fontdimen7\test,\cr



\end