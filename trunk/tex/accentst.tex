% Copyright 2007 TeX Users Group.
% You may freely use, modify and/or distribute this file.

\typesize=12pt		% This is for YTeX

\font\xmr=xmr10	

% xmr10 is BSR CMR10 outline font from Y&Y reencoded using texansi.vec
% xmr10 is BSR CMR10 outline font from Y&Y reencoded using texannew.vec

% NOTE:  The following is crude, TeX macros could make this much cleaner.

% NOTE:  Many accented characters may be best dealt with using ligatures
%        To do this, insert ligatures in the AFM file and run AFMtoTFM.

% Some simple examples follow:

\def\aa{\char229} 	% aring

% was: \def\aa{\accent23a}

\def\AA{\char197} 	% Aring

% was: \def\AA{\leavevmode\setbox0\hbox{h}\dimen@\ht0\advance\dimen@-1ex%
%        \rlap{\raise.67\dimen@\hbox{\char'27}}A}

\def\cc{\char231}	% ccedilla

% \def\CC{\char199}	% Ccedilla
\def\CC{\char215}	% Ccedilla

% etc etc

\def\ql{\char145} % \quoteleft is actually 145 (since grave replaced it)

\noheaders

\xmr

\noindent This is a test of xmr10, a reencoded version of cmr10 that 
uses the upper character code range (160--255) for the 58 \ql standard'
composite/accented characters 

The accented characters are here accessed directly using {\TeX}'s 
\ql char'.
They could also be set up as macros (to replace existing macros for
accented characters, such as \ql aa' etc.) --- or using ligatures 
(simply add new ligatures to the AFM file before running AFMtoTFM).

\vskip .05in

\noindent
Agrave: \char192.
Aacute: \char193.
Acircumflex: \char194.
Atilde: \char195.
Adieresis: \char196.
Aring: \char197.
% Ccedilla: \char199.
Ccedilla: \char215.
Egrave: \char200.
Eacute: \char201.
Ecircumflex: \char202.
Edieresis: \char203.
Igrave: \char204.
% Iacute: \char205.
Iacute: \char247.
% Icircumflex: \char206.
Icircumflex: \char230.
% Idieresis: \char208.
Idieresis: \char207.
Ntilde: \char209.
Ograve: \char210.
Oacute: \char211.
Ocircumflex: \char212.
Otilde: \char213.
Odieresis: \char214.
% Scaron: \char215.
Scaron: \char138.
Ugrave: \char217.
Uacute: \char218.
Ucircumflex: \char219.
Udieresis: \char220.
Yacute: \char221.
% Ydieresis: \char222.
Ydieresis: \char159.
Zcaron: \char223.

\vskip .05in

\noindent
agrave: \char224.
aacute: \char225.
acircumflex: \char226.
atilde: \char227.
adieresis: \char228.
aring: \char229.
ccedilla: \char231.
egrave: \char232.
eacute: \char233.
ecircumflex: \char234.
edieresis: \char235.
igrave: \char236.
iacute: \char237.
icircumflex: \char238.
idieresis: \char240.
ntilde: \char241.
ograve: \char242.
oacute: \char243.
ocircumflex: \char244.
otilde: \char245.
odieresis: \char246.
% scaron: \char247.
scaron: \char154.
ugrave: \char249.
uacute: \char250.
ucircumflex: \char251.
udieresis: \char252.
yacute: \char253.
ydieresis: \char254.
zcaron: \char255.

\vskip .05in

% $$\Gamma, \Gamma, \Gamma, \Gamma$$

\noindent
Note that many non-clone PS interpreters have a bug that may interfere 
with rendering of composite characters:
there is an \ql invalidfont' error unless the
base characters and accent characters are present in the
encoding (This applies for example to the ALW II NT).
Some clone interpreters do the \ql right thing', that is,
look up the character name in StandardEncoding and then find a
character with that name in the fonts CharString dictionary
(this is what is done in the NewGen Turbo 480, for example).

This is one of two serious bugs in Adobe interpreters 
(now documented in the new white-red PostScript book).
To deal with this problem,
DVIPSONE not only copies across the required CharStrings, 
but also inserts the required base and accent characters in the encoding.

\end
