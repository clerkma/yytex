% Copyright 2007 TeX Users Group.
% You may freely use, modify and/or distribute this file.

% This is the test file `ansitest.tex' which shows use of text fonts with
% `Windows ANSI' encoding.

% The original PFM files for typical text fonts are set up for Windows ANSI. 
% So NO changes are required for the Windows/ATM setup.  
% Just install the fonts using ATM.  Nothing else is required for DVIWindo.

% For DVIPSONE, we need a font substitution file requestion reencoding to
% `Windows ANSI' (containing lines like `tir *remap* ansinew'). 
% This is the file `ansiacce.sub', shown in part at the end.

% For TeX we need to make suitable TFM files based on Windows ANSI, e.g.
% afmtotfm -vadjx -c=ansinew c:\afm\tir

% Also, since plain TeX and LaTeX are hard-wired for CM's `TeX text' encoding,
% we'll need to redefine the macros for special characters and accents.
% The file `ansiacce.tex' takes care of this.

% See end for discussion of advantages and disadvantages of `Windows ANSI'

\input ansiacce		% to adjust special characters & accents for ANSI

\nopagenumbers		% avoid calling CMR10 for page number ...

\font\tir=tir at 12.345pt	% for main text

\font\com=com at 12.345pt	% for `program fragments'

\chardef\bs=92		% backslash

\chardef\degree=176	% `ring' or `degree' in Windows ANSI

\tir 

This is the file `{\com ansitext.tex}'
using Times-Roman reencoded to `{\com ansinew}'

\bigskip

We used  `{\com afmtotfm -vadjx -c=ansinew c:\bs afm\bs tir}'
to make TFM file

\bigskip	

And ran DVIPSONE using `{\com dvipsone -v -s=ansiacce -d=lpt1 ansitest}'

\bigskip	

The PFB and PFM files used are the ones straight from Adobe

(in the case of the substitution file asking for downloading).

\bigskip	

{\TeX} pseudo ligatures: 
endash -- emdash --- exclamdown !` questiondown ?` 

\bigskip	

Lets check for extra pseudo ligatures:
``English''  ,,German'' <<French>>

\bigskip

Lets check special characters:
\ae\ \oe\ \o\ \ss\ \aa\ \cc\ \AE\ \OE\ \O\ \AA\ \CC

\bigskip	

Lets check on icelandic and other characters:
\th\ \TH\ \dh\ \DH\ \pounds\ \copyright

\bigskip	

Lets check for real ligatures (there shouldn't be any): 
fi fl ff ffi ffl

% Altough, kerning in a font can make them almost look like ligatures!

\bigskip	

Now enable use of extra `ligatures' in TFM file 

(to gain easy direct access to 58 pre-built accented characters in the font).

\bigskip	

% The following lines are commented out in `ansiacce.tex' - we can
% uncomment them here since the TFM file DOES have the extra `ligatures'

\def\`{\char96 }	% grave
\def\'{\char180 }	% acute
\def\v{\char141 }	% caron
\def\={\char175 }	% macron
\def\^{\char136 }	% circumflex
\def\~{\char152 }	% tilde
\def\"{\char168 }	% dieresis
\def\c{\char184 }	% cedilla

Test some of the accented characters:
\`a \'a \^a \~a \"a \c c
\`A \'A \^A \~A \"A \c C

\bigskip

The main advantage of Windows ANSI encoding use in TeX is that

it harmonizes with the way all Windows applications use text fonts.

No need to make up new PFM files, or to reencode the PFB files.

\bigskip

So what does one lose by using `Windows ANSI' encoding?

\bigskip

In text fonts one loses: Lslash, lslash, Zcaron, zcaron.

Some accents: breve, caron, dotaccent, hungarumlaut, ogonek, dotlessi.

Some math characters: fraction (and perhaps minus and degree).

And ligatures: fi, fl, (most text fonts don't have ff, ffi, and ffl).

\bigskip

Here is (part of) the font substitution file `{\com ansiacce.sub}'

\bigskip

It shows three ways of achieving the desired effect.

\bigskip

\com

(Note how thin Courier looks as a fixed width font!).

\bigskip

; The following calls for reencoding of a RESIDENT font:

tir Times-Roman *force* *reside* *remap* ansinew

\bigskip

; Following calls for reencoding of a DOWNLOADABLE font:

; tir *remap* ansinew

\bigskip

; An alternate way of reencoding a font to be downloaded:

; tir tir *remap* ansinew

\end
