% Copyright 2007 TeX Users Group.
% You may freely use, modify and/or distribute this file.
\typesize=12pt

\noheaders

\def\bbf{\bigsize\bf}

\noindent {\bbf Program:} \quad\quad PFMANAL

\vskip .1in

\noindent {\bbf Functions:}

\vskip .05in 

\bpar Analyze a MicroSoft Windows Font Metric (PFM) file.

\bpar Help in comparative anatomy of a set of binary files.

\vskip .1in

\goodbreak

\noindent {\bbf Inputs:}

\vskip .05in

\beginbullets

\bpar A MicroSoft Windows Font Metric (PFM) file.

\bpar A set of binary files to be compared.

\endbullets

\vskip .05in

\noindent {\bbf Outputs:}

\vskip .05in

\bpar Listing on standard output of what was found in the PFM file.

\bpar For multiple files, listing of what each file contains at
particular offset.

\vskip .1in

\goodbreak

\noindent {\bbf Controlling Behavior:}

\vskip .05in

\noindent The actions taken by PFMANAL are controlled by variables in
the source code and by whether there is a single argument or several.
There is also a command line flag (`-') that turns on detailed tracing.

Sample invocations of PFMANAL:

\vskip .05in

\verb@pfmanal c:\psfonts\pfm\tir_____.pfm@

\verb@pfmanal 23 c:\psfonts\pfm\*.pfm@

\vskip .05in

\noindent
If there is more than one arguments, the first must be a number, 
and the rest names of files.
The number specifies an offset in bytes from the beginning of each file.

\vskip .1in

\goodbreak

\noindent{\bbf Flags and Variables:}

\vskip .05in

\noindent Additional control is provided by settings of variables and flags in
the program itself, listed at the beginning of the source code of PFMANAL.
Typically these are variables that would only rarely be changed.
In some cases these are variables that have become `frozen', 
when settings other than the ones in the code were determined to be
less useful or unsatisfactory.
% (changing these may lead to unsatisfactory performance of FONTONE).

\vskip .1in

\goodbreak

\noindent{\bbf Notes:}

\vskip .05in

\bpar Part of the output produced by PFMANAL is in a format
reminiscent of that found in an AFM file.  It is not possible to
construct an AFM file, however, since so much information required in
an AFM file is not found in a PFM file.
For example, the character encoding is unknown, the character
bounding box is unknown, and ligature information is missing.

\bpar In the `comparative anatomy' mode (more than one argument), PFMANAL
shows what is at the specified offset in each file.
Shown are first the byte at that offset as a character, then as a
number (ASCII code), 
then the 16 bit number obtained by combining this byte with the following one,
then the 24 bit number obtained by combining three bytes,
finally the 32 bit number obtained by combining four bytes.

\vskip .1in

\goodbreak

\noindent{\bbf Bugs in ATM for Windows and PostScript Interpreters:}

\vskip .05in

\vskip .1in

\goodbreak

\noindent{\bbf Restrictions:}

\vskip .05in

\bpar Expected order:  FONTONE expects characters in the HEX file, the
AFM file and the HNT file to appear in order of numerical code.

\bpar Lack of copyright message generality:
Presently only three different copyright messages (and corresponding 
`time capsules') can be generated automatically.

\end

