% Torture test for math fonts (Mostly from TeX book, some via Karl Berry)
% Copyright 2007 TeX Users Group.
% You may freely use, modify and/or distribute this file.

% run in plain TeX

%\tracingoutput = 1
%\tracingboxes

\def\newpage{\vfill\eject}

% Following is for even bigger sizes than plain provides (for CMEX10 at 10pt)
\def\biggg#1{{\hbox{$\left#1\vbox to20.5pt{}\right.$}}}
\def\bigggl{\mathopen\biggg}
\def\bigggr{\mathclose\biggg}
\def\Biggg#1{{\hbox{$\left#1\vbox to23.5pt{}\right.$}}}
\def\Bigggl{\mathopen\Biggg}
\def\Bigggr{\mathclose\Biggg}

% NOTE: The above is subsumed by code in lcdplain.mac...

% \input lcdplain.mac	% for use around here
\input lcdplain		% for use around here
% \input accents.tex	% with texannew encoding
\input ansiacce		% with texannew encoding
\catcode96=12	% other - for \challenge below

% \input lcdplain.tex	% for sending out
% \input stanacce.tex	% with standardencoding

% Change general typesetting parameters.
% 
\rm
% \headline = {{\tenrm\folio} \hfil \timestamp}
\headline = {\hfil {\tenrm\folio} \hfil}
\footline = {}
\advance\vsize by .1in
\raggedbottom
\raggedright
\rightskip = 2em plus 5em minus 2em
\parindent = 0pt
\parskip = .5\baselineskip


% Test the basic character repertoire: we go through the math symbols
% listed in Appendix F of the TeXbook.  This code is based on the \math
% test in testfont.tex.  

% Apply \\ to roman and Greek upper- and lowercase, and script
% uppercase.  Use \ii and \jj for `i' and `j', since sometimes we want
% the dotless versions.
\def\lettertrial{Letters: $
  \\A \\B \\C \\D \\E \\F \\G \\H \\I \\J \\K \\L \\M \\N \\O \\P \\Q
  \\R \\S \\T \\U \\V \\W \\X \\Y \\Z
  $\par$
  \\a \\b \\c \\d \\e \\f \\g \\h \\\ii \\\jj \\k \\l \\m \\n \\o \\p
  \\q \\r \\s \\t \\u \\v \\w \\x \\y \\z
  $\par$
  \\\alpha \\\beta \\\gamma \\\delta \\\epsilon \\\zeta \\\eta \\\theta
  \\\vartheta \\\iota \\\kappa \\\lambda \\\mu \\\nu \\\xi \\\pi
  \\\varpi \\\rho \\\sigma \\\tau \\\upsilon \\\phi \\\varphi \\\chi
  \\\psi \\\omega
  $\par$
  \\\Gamma \\\Delta \\\Theta \\\Lambda \\\Xi \\\Pi \\\Sigma \\\Upsilon
  \\\Phi \\\Psi \\\Omega
  $\par$
  \\\scriptA \\\scriptB \\\scriptC \\\scriptD \\\scriptE \\\scriptF
  \\\scriptG \\\scriptH \\\scriptI \\\scriptJ \\\scriptK \\\scriptL
  \\\scriptM \\\scriptN \\\scriptO \\\scriptP \\\scriptQ \\\scriptR
  \\\scriptS \\\scriptT \\\scriptU \\\scriptV \\\scriptW \\\scriptX
  \\\scriptY \\\scriptZ
  $\par
}

% Test miscellanous Ord symbols, except for \imath and \jmath.
\def\miscordtrial{Miscellanous ordinary symbols: $
  \\\aleph \\\hbar \\\ell \\\wp \\\Re \\\Im \\\partial \\\infty
  \\\emptyset \\\nabla \\\surd \\\top \\\bot \\\| \\\angle \\\triangle
  \\\backslash \\\forall \\\exists \\\neg \\\flat \\\natural \\\sharp
  \\\clubsuit \\\diamondsuit \\\heartsuit \\\spadesuit \\! \\? \\. \\|
  \\/ \\` \\@ \\"
  $\par
}

% Test the various styles of digits.
\def\digittest{Digits: $0123456789$, $\oldstyle 0123456789$\par}

% Test large operators.
\def\largeoptrial{Large operators: $
  \\\sum \\\prod \\\coprod \\\int \\\oint \\\bigcap \\\bigcup
  \\\bigsqcup \\\bigvee \\\bigwedge \\\bigodot \\\bigotimes \\\bigoplus
  \\\biguplus
  $\par
}

% Binary operators.
\def\binoptrial{Binary operators: $
  \\\pm \\\cap \\\vee \\\mp \\\cup \\\wedge \\\setminus \\\uplus
  \\\oplus \\\cdot \\\sqcap \\\ominus \\\times \\\sqcup \\\otimes \\\ast
  \\\triangleleft \\\oslash \\\star \\\triangleright \\\odot \\\diamond
  \\\wr \\\dagger \\\circ \\\bigcirc \\\ddagger \\\bullet
  \\\bigtriangleup \\\amalg \\\div \\\bigtriangledown \\* \\+ \\-
  $\par
}

% \bmod and \pmod, from p.164 of the TeXbook.
\def\modtest{Mod: 
  $\gcd(m,n) = \gcd(n,m \bmod n)$, and
  $x\equiv y+1 \pmod{m^2}$.
  \par
}

% Relations.
\def\relationtrial{Relations: $
  \\\leq \\\geq \\\equiv \\\prec \\\succ \\\approx  \\\preceq \\\succeq
  \\\asymp \\\subset \\\supset \\\sim \\\subseteq
  \\\supseteq \\\simeq \\\sqsubseteq \\\sqsupseteq
  \\\cong \\\in \\\ni \\\vdash \\\models 
  \\\parallel \\= \\< \\> \\\mid \\\propto \\\ll \\\gg 
  \\\bowtie \\\dashv \\\smile \\\frown \\\doteq \\\perp \\: 
% \\\ne
  $\par
}

% Negated relations (for LucidaBright test to use actual composite chars)
\def\modrelationtrial{Relations: $
  \\\notleq \\\notgeq \\\notequiv \\\notprec \\\notsucc \\\notapprox
  \\\notpreceq \\\notsucceq 
  \\\notasymp \\\notsubset \\\notsupset \\\notsim \\\notsubseteq 
  \\\notsupseteq \\\notsimeq \\\notsqsubseteq \\\notsqsupseteq
  \\\notcong \\\notin \\\notni \\\notvdash \\\notmodels
  \\\notparallel \\\noteq \\\notless \\\notgreater \\{\notmid}
% no compound character available for the following
  \\{\not\propto} \\{\not\ll} \\{\not\gg} 
  \\{\not\bowtie} \\{\not\dashv} \\{\not\smile} \\{\not\frown}
  \\{\not\doteq} \\{\not\perp} 
% \\{\not:} \\{\not\ne}
  $\par
}

% Negated relations (version for CM)
\def\negrelationtrial{Negated \begingroup
  \let\temp = \\%
  \def\\##1{\temp{\not##1}}%
  \let\par = \relax
  \relationtrial
  $\temp\notin$
  \endgraf
\endgroup}

% Negated relations (modified for LucidaBright)
\def\negrelationtrial{Negated \begingroup
  \let\temp = \\%
  \def\\##1{\temp{##1}}%
  \let\par = \relax
  \modrelationtrial
  $\temp\notin$
  \endgraf
\endgroup}

% Arrows.
\def\arrowtrial{Arrows: $
 \\\leftarrow \\\longleftarrow \\\uparrow \\\Leftarrow \\\Longleftarrow
 \\\Uparrow \\\rightarrow \\\longrightarrow \\\downarrow \\\Rightarrow
 \\\Longrightarrow \\\Downarrow \\\leftrightarrow \\\longleftrightarrow
 \\\updownarrow \\\Leftrightarrow \\\Longleftrightarrow \\\Updownarrow
 \\\mapsto \\\longmapsto \\\nearrow \\\hookleftarrow \\\hookrightarrow
 \\\searrow \\\leftharpoonup \\\rightharpoonup \\\swarrow
 \\\leftharpoondown \\\rightharpoondown \\\nwarrow \\\rightleftharpoons
 $\par
}

% \buildrel.
\def\buildreltest{Buildrel:
  $\buildrel \alpha\beta \over \longrightarrow$, and
  $\buildrel \rm def \over =$. 
%   , and
%   $\defineequal$.	% only in lcdbright
  \par
}

% Delimiters are more complicated, since we want to see them grow.  This
% code is from the Metafontbook, the end of Appendix H.
\let\interitem = .
\newcount\delimpaircount

\def\testdelimpair#1#2{%
  \Bigggl{#1}\interitem
  \bigggl{#1}\interitem
  \Biggl{#1}\interitem
  \biggl{#1}\interitem
  \Bigl{#1}\interitem
  \bigl{#1}\interitem
  \left#1 \bullet
  \right#2\interitem
  \bigr{#2}\interitem
  \Bigr{#2}\interitem
  \biggr{#2}\interitem
  \Biggr{#2}\interitem
  \bigggr{#2}\interitem
  \Bigggr{#2}%
  \advance\delimpaircount by 1
  \ifodd\delimpaircount \hfil \else \expandafter \foo \fi
}

\def\foo{\cr}

\def\delimtest{Delimiters: $$\displaylines{\delimpaircount = 0
  \testdelimpair ()
  \testdelimpair \lbrack\rbrack
  \testdelimpair \lgroup\rgroup
  \testdelimpair \lmoustache\rmoustache
  \testdelimpair \vert\Vert
  \testdelimpair \arrowvert\Arrowvert
  \testdelimpair \uparrow\downarrow
  \testdelimpair \updownarrow\Updownarrow
  \testdelimpair \Uparrow\Downarrow
  \testdelimpair \bracevert{\delimiter"342}
  \testdelimpair \backslash/
  \testdelimpair \langle\rangle
  \testdelimpair \lbrace\rbrace
  \testdelimpair \lceil\rceil
  \testdelimpair \lfloor\rfloor
  \testdelimpair \ldbrack\rdbrack
  \crcr
  }$$
  %
  % Throw in square roots, too, as long as we're here.
  $$\sqrt{\sqrt{\sqrt{\sqrt{\sqrt{\sqrt{\sqrt{\sqrt{\sqrt{-1}}}}}}}}}$$
}

% Punctuation.
\def\puncttest{Punctuation: 
  $f\colon A\rightarrow B$,
  $L(a, b; c\colon x, y;z)$,
  $a.b : c, d$,
  \dots, and $x_1 + \cdots + x_n$,
  don't forget $f(x_1, \ldots, x_n)$.
  \par
}



% Test fractions.
\def\fractiontrial{Fractions: $
  \\\over
  \\\atop
  \\\choose
  \\{\above.5pt}
  \\{\overwithdelims<>}
  \\{\atopwithdelims[]}
  \\{\abovewithdelims\lbrace\rbrace1pt}
  $\par
}

% Test roots.
\def\roottest{Roots:
  $\sqrt{2i}$,
  $\root 3 \of 2$,
  $\root n \of {x^n + y^n}$,
  $\sqrt{\mathstrut a} + \sqrt{\mathstrut d} + \sqrt{\mathstrut y}$,
  and $\root n+1 \of Q$.
  \par
}

% Test various fills.
\def\filltrial{Filling: $
  \\\overline \\\underline \\\widehat
  \\\widetilde \\\overbrace \\\underbrace
  $\par
}

% Test accents.
\def\accenttrial{Accents: $
  \let\ii = \imath
  \let\jj = \jmath
  \\\hat \\\check \\\tilde \\\acute \\\grave \\\dot \\\ddot \\\breve
  \\\bar \\\vec
  $\par
}


% Challenges, from pp.180-182 of the TeXbook.
\def\challenge{
$n^{\rm th}$ root

% \bf omitted here, since we're not trying to test bold.
% ${S^{\rm-1}TS=dg}(\omega_1,\ldots,\omega_n) =\Lambda$
% no bold face \Lambda ...
${\bf S^{\rm-1}TS=dg}(\omega_1,\ldots,\omega_n) =\Lambda$

$\Pr(\,m=n\mid m+n=3\,)$

$\sin18^\circ={1\over4}(\sqrt5-1)$

$k=1.38\times10^{-16}\rm\,erg/^\circ K$

$\bar\Phi\subset NL_1^*/N=\bar L_1^*
          \subseteq\cdots\subseteq NL_n^*/N=\bar L_n^*$

$I(\lambda)=\int\!\!\int_Dg(x,y)e^{i\lambda h(x,y)}\,dx\,dy$

$\int_0^1\!\cdots\int_0^1f(x_1,\ldots,x_n)\,dx_1\ldots\,dx_n$

$$x_{2m}\equiv\cases{Q(X_m^2-P_2W_m^2)-2S^2&($m$ odd)\cr
               \noalign{\vskip2pt} % spread the lines apart a little
               P_2^2(X_m^2-P_2W_m^2)-2S^2&($m$ even)\cr}\pmod N.$$

$$(1+x_1z+x_1^2z^2+\cdots\,)\ldots(1+x_nz+x_n^2z^2+\cdots\,)
          ={1\over(1-x_1z)\ldots(1-x_nz)}.$$
        
$$\prod_{j\ge0}\biggl(\sum_{k\ge0}a_{jk}z^k\biggr)
    =\sum_{n\ge0}z^n\,\Biggl(\sum_
       {\scriptstyle k_0,k_1,\ldots\ge0\atop
        \scriptstyle k_0+k_1+\cdots=n}
     a_{0k_0}a_{1k_1}\ldots\,\Biggr).$$

$${(n_1+n_2+\cdots+n_m)!\over n_1!\,n_2!\ldots n_m!}
    ={n_1+n_2\choose n_2}{n_1+n_2+n_3\choose n_3}
      \ldots{n_1+n_2+\cdots+n_m\choose n_m}.$$

$$\def\\##1##2{(1-q^{##1_##2+n})} % to save typing
  \Pi_R{a_1,a_2,\ldots,a_M\atopwithdelims[]b_1,b_2,\ldots,b_N}
    =\prod_{n=0}^R{\\a1\\a2\ldots\\aM\over\\b1\\b2\ldots\\bN}.$$

$$\sum_{p\rm\;prime}f(p)=\int_{t>1}f(t)\,d\pi(t).$$

$$\{\underbrace{\overbrace{\mathstrut a,\ldots,a}
        ^{k\;a\mathchar`'\rm s},
      \overbrace{\mathstrut b,\ldots,b}
        ^{l\;b\mathchar`'\rm s}}_{k+l\rm\;elements}\}.$$
        
$$\pmatrix{\pmatrix{a&b\cr c&d\cr}&
               \pmatrix{e&f\cr g&h\cr}\cr
             \noalign{\smallskip}
             0&\pmatrix{i&j\cr k&l\cr}\cr}.$$

$$\det\left|\,\matrix{
    c_0&c_1\hfill&c_2\hfill&\ldots&c_n\hfill\cr
    c_1&c_2\hfill&c_3\hfill&\ldots&c_{n+1}\hfill\cr
    c_2&c_3\hfill&c_4\hfill&\ldots&c_{n+2}\hfill\cr
    \,\vdots\hfill&\,\vdots\hfill&
         \,\vdots\hfill&&\,\vdots\hfill\cr
    c_n&c_{n+1}\hfill&c_{n+2}\hfill&\ldots&c_{2n}\hfill\cr
    }\right|>0.$$

$$\mathop{{\sum}'}_{x\in A}f(x)\mathrel{\mathop=^{\rm def}}
    \sum_{\scriptstyle x\in A\atop\scriptstyle x\ne0}f(x).$$

$$2\uparrow\uparrow k\mathrel{\mathop=^{\rm def}}
    2^{2^{2^{\cdot^{\cdot^{\cdot^2}}}}}
      \vbox{\hbox{$\Big\}\scriptstyle k$}\kern0pt}.$$

$$\def\normalbaselines{\baselineskip20pt
    \lineskip3pt \lineskiplimit3pt }
  \def\mapright##1{\smash{
      \mathop{\longrightarrow}\limits^{##1}}}
  \def\mapdown##1{\Big\downarrow
    \rlap{$\vcenter{\hbox{$\scriptstyle##1$}}$}}
  \matrix{&&&&&&0\cr
    &&&&&&\mapdown{}\cr
    0&\mapright{}&{\cal O}_C&\mapright\iota&
      \cal E&\mapright\rho&\cal L&\mapright{}&0\cr
    &&\Big\Vert&&\mapdown\phi&&\mapdown\psi\cr
    0&\mapright{}&{\cal O}_C&\mapright{}&
      \pi_*{\cal O}_D&\mapright\delta&
      R^1f_*{\cal O}_V(-D)&\mapright{}&0\cr
    &&&&&&\mapdown{\theta_i\otimes\gamma^{-1}}\cr
    &&&&&&\hidewidth R^1f_*\bigl({\cal O}
      _V(-iM)\bigr)\otimes\gamma^{-1}\hidewidth\cr
    &&&&&&\mapdown{}\cr
    &&&&&&0\cr}$$
}


% We need the calligraphic capitals available via a control sequence.
\def\scriptA{{\cal A}} \def\scriptN{{\cal N}}
\def\scriptB{{\cal B}} \def\scriptO{{\cal O}}
\def\scriptC{{\cal C}} \def\scriptP{{\cal P}}
\def\scriptD{{\cal D}} \def\scriptQ{{\cal Q}}
\def\scriptE{{\cal E}} \def\scriptR{{\cal R}}
\def\scriptF{{\cal F}} \def\scriptS{{\cal S}}
\def\scriptG{{\cal G}} \def\scriptT{{\cal T}}
\def\scriptH{{\cal H}} \def\scriptU{{\cal U}}
\def\scriptI{{\cal I}} \def\scriptV{{\cal V}}
\def\scriptJ{{\cal J}} \def\scriptW{{\cal W}}
\def\scriptK{{\cal K}} \def\scriptX{{\cal X}}
\def\scriptL{{\cal L}} \def\scriptY{{\cal Y}}
\def\scriptM{{\cal M}} \def\scriptZ{{\cal Z}}


% Ordinarily, use the normal letters.
\let\ii = i
\let\jj = j

% Run #1 in each of the four main styles.
\def\styletest#1{\begingroup
  (Display style) \everymath = {\displaystyle}#1%
  (Text style) \everymath = {\textstyle}#1%
  (Script style) \everymath = {\scriptstyle}#1%
  (Scriptscript style) \everymath = {\scriptscriptstyle}#1%
\endgroup}


\def\printparam#1{%
  $\hbox{\tt #1} = \hbox{\rm \expandafter\the\csname #1\endcsname}$%
}

\def\interspace{\allowbreak\hskip 1em plus.5em minus.5em}



% Run the tests.
\centerline{\bf This is a torture test for math fonts.}

\vskip .1in

\noindent
Testing with \printparam{thinmuskip}, \printparam{medmuskip},
\printparam{thickmuskip}, \printparam{scriptspace},
\printparam{delimitershortfall}, \printparam{delimiterfactor},
\printparam{mathsurround}, \printparam{nulldelimiterspace}.


\def\\#1{-#1'}\lettertrial
\def\\#1{\allowbreak\quad#1x}\miscordtrial
\styletest\digittest

% Tests the large operators first in text size, then display size.
\def\\#1{#1^a_b B\interspace}\largeoptrial
\def\\#1{\displaystyle #1^b_a B \interspace}\largeoptrial

\def\\#1{A#1x\allowbreak\quad}\binoptrial
\styletest\modtest

\def\\#1{x#1A\allowbreak\quad}\relationtrial
\def\\#1{y#1B\allowbreak\quad}\negrelationtrial
\def\\#1{B#1y\allowbreak\quad}\arrowtrial

\newpage % \vskip .25in % try and force page break ...

\styletest\buildreltest
\delimtest
\styletest\puncttest

\def\\#1{{Z #1 W}\interspace}\fractiontrial
\def\\#1{{g #1 d}\interspace}\fractiontrial
\def\\#1{{b #1 a}\interspace}\fractiontrial

\newpage % \vskip .25in % try and force page break ...

\styletest\roottest

\def\\#1{#1{o}\interspace}\filltrial
\def\\#1{#1{A + \cdots - Q - \cdots + g}\interspace}\filltrial

\def\\#1{- #1A + #1q}\accenttrial

% %%% %%% %%% %%% %%% %%% %%% %%% %%% %%% %%% %%% %%% %%% %%% %%% %%% %%%

%% recent additions:

% extra stuff from mathtest

$
\hat{A}\hat{B}\hat{C}\hat{D}\hat{E}\hat{F}\hat{G}\hat{H}
\hat{I}\hat{J}\hat{K}\hat{L}\hat{M}\hat{N}\hat{O}\hat{P}
\hat{Q}\hat{R}\hat{S}\hat{T}\hat{U}\hat{V}\hat{W}\hat{X}
\hat{Y}\hat{Z}
%
\quad
%
\hat{a}\hat{b}\hat{c}\hat{d}\hat{e}\hat{f}\hat{g}\hat{h}
\hat{i}\hat{j}\hat{k}\hat{l}\hat{m}\hat{n}\hat{o}\hat{p}
\hat{q}\hat{r}\hat{s}\hat{t}\hat{u}\hat{v}\hat{w}\hat{x}
\hat{y}\hat{z}
$

% Define uppercase Greek italic to use the math italic font.
\mathchardef\varGamma="100 \mathchardef\varDelta="101
\mathchardef\varTheta="102 \mathchardef\varLambda="103 \mathchardef\varXi="104
\mathchardef\varPi="105 \mathchardef\varSigma="106 \mathchardef\varUpsilon="107
\mathchardef\varPhi="108 \mathchardef\varPsi="109 \mathchardef\varOmega="10A

% Upright uppercase Greek is actually in LBR

$
\hat{\Gamma}\hat{\Delta}\hat{\Theta}\hat{\Lambda}\hat{\Xi}\hat{\Pi}
\hat{\Sigma}\hat{\Upsilon}\hat{\Phi}\hat{\Psi}\hat{\Omega}
%
\quad
%
\hat{\varGamma}\hat{\varDelta}\hat{\varTheta}\hat{\varLambda}\hat{\varXi}\hat{\varPi}
\hat{\varSigma}\hat{\varUpsilon}\hat{\varPhi}\hat{\varPsi}\hat{\varOmega}
$

$
\hat{\alpha}\hat{\beta}\hat{\gamma}\hat{\delta}\hat{\epsilon}\hat{\zeta}
\hat{\eta}\hat{\theta}\hat{\iota}\hat{\kappa}\hat{\lambda}\hat{\mu}\hat{\nu}
\hat{\xi}\hat{\pi}\hat{\rho}\hat{\sigma}\hat{\tau}\hat{\upsilon}\hat{\phi}
\hat{\chi}\hat{\psi}\hat{\omega}\hat{\varepsilon}\hat{\vartheta}\hat{\varpi}
\hat{\varrho}\hat{\varsigma}\hat{\varphi}
%
\quad
%
\hat{\partial}\hat{\ell}\hat{\imath}\hat{\jmath}\hat{\wp}
$

$
\hat{{\cal A}}\hat{{\cal B}}\hat{{\cal C}}\hat{{\cal D}}\hat{{\cal E}}
\hat{{\cal F}}\hat{{\cal G}}\hat{{\cal H}}\hat{{\cal I}}\hat{{\cal J}}
\hat{{\cal K}}\hat{{\cal L}}\hat{{\cal M}}\hat{{\cal N}}\hat{{\cal O}}
\hat{{\cal P}}\hat{{\cal Q}}\hat{{\cal R}}\hat{{\cal S}}\hat{{\cal T}}
\hat{{\cal U}}\hat{{\cal V}}\hat{{\cal W}}\hat{{\cal X}}\hat{{\cal Y}}
\hat{{\cal Z}}
$

% %%% %%% %%% %%% %%% %%% %%% %%% %%% %%% %%% %%% %%% %%% %%% %%% %%% %%%

%% extra random stuff added from TeX book

% page 133

$x + y - z$, \quad $x + y * z$, \quad $z * y / z$, \quad 
$(x+y)(x-y) = x^2 - y^2$, 

$x \times y \cdot z = [x\, y\, z]$, \quad $x\circ y \bullet z$, \quad
$x\cup y \cap z$, \quad $x\sqcup y \sqcap z$, \quad

$x \vee y \wedge z$, \quad $x\pm y\mp z$, \quad
$x=y/z$, \quad $x:=y$, \quad $x\le y \ne z$, \quad $x \sim y \simeq z$
$x \equiv y \notequiv z$, \quad $x\subset y \subseteq z$

$\sin2\theta=2\sin\theta\cos\theta$, \quad
$\hbox{O}(n\log n\log n)$, \quad
$\Pr(X>x)=\exp(-x/\mu)$,

$\bigl(x\in A(n)\bigm|x\in B(n)\bigr)$, \quad
$\bigcup_n X_n\bigm\|\bigcap_n Y_n$

% page 178

In text matrices $1\,1\choose0\,1$ and $\bigl({a\atop 1}{b\atop m}{c\atop n}\bigr)$

% page 142

$$a_0+{1\over\displaystyle a_1 +
{\strut 1\over\displaystyle a_2 +
{\strut 1\over\displaystyle a_3 +
{\strut 1\over\displaystyle a_4}}}}$$

% page 143

$${p \choose 2}x^2y^{p-2} - {1\over 1 - x}{1 \over 1 - x^2}
=
{a+1\over b}\bigg/{c+1\over d}.$$

%% page 145

$$\sqrt{1+\sqrt{1+\sqrt{1+\sqrt{1+\sqrt{1+x}}}}}$$

%% page 147

$$\left({\partial^2\over\partial x^2} + {\partial^2\over\partial y^2}\right)
\bigl|\varphi(x+iy)\bigr|^2=0$$

%% page 149

% $$\pi(n)=\sum_{m=2}^n\left\lfloor\biggl(\sum_{k=1}^{m-1}\bigl
% \lfloor(m/k)\big/\lceil m/k\rceil\bigr\rfloor\biggr)^{-1}\right\rfloor.$$

$$\pi(n)=\sum_{m=2}^n\left\lfloor\Biggl(\sum_{k=1}^{m-1}\bigl
\lfloor(m/k)\big/\lceil m/k\rceil\bigr\rfloor\Biggr)^{-1}\right\rfloor.$$

% page 168

$$\int_0^\infty {t - i b\over t^2 + b^2}e^{iat}\,dt=e^{ab}E_1(ab), \quad
a,b > 0.$$

% page 176

$$A = \left(\matrix{x-\lambda&1&0\cr
0&x-\lambda&1\cr
0&0&x-\lambda\cr}\right).$$

$$\left\lgroup\matrix{a&b&c\cr d&e&f\cr}\right\rgroup
\left\lgroup\matrix{u&x\cr v&y\cr w&z\cr}\right\rgroup$$

% page 177

$$A = \pmatrix{a_{11}&a_{12}&\ldots&a_{1n}\cr
a_{21}&a_{22}&\ldots&a_{2n}\cr
\vdots&\vdots&\ddots&\vdots\cr
a_{m1}&a_{m2}&\ldots&a_{mn}\cr}$$

$$M=\bordermatrix{&C&I&C'\cr
C&1&0&0\cr I&b&1-b&0\cr C'&0&a&1-a\cr}$$

%% page 186

$$\sum_{n=0}^\infty a_nz^n\qquad\hbox{converges if}\qquad
|z|<\Bigl(\limsup_{n\to\infty}\root n\!\of{|a_n|}\,\Bigr)^{-1}.$$

$${f(x+\Delta x)-f(x)\over\Delta x}\to f'(x)
\qquad \hbox{as $\Delta x\to0$.}$$

$$\|u_i\|=1,\qquad u_i\cdot u_j=0\quad\hbox{if $i\ne j$.}$$

%% page 191

$$\it\hbox{The confluent image of}\quad\left\{
\matrix{\hbox{an arc}\hfill\cr\hbox{a circle}\hfill\cr
\hbox{a fan}\hfill\cr}
\right\}\quad\hbox{is}\quad\left\{
\matrix{\hbox{an arc}\hfill\cr
\hbox{an arc or a circle}\hfill\cr
\hbox{a fan or an arc}\hfill\cr}\right\}.$$

%% page 191

$$\eqalign{T(n)\le T(2^{\lceil\lg n\rceil})
&\le c(3^{\lceil\lg n\rceil}-2^{\lceil\lg n\rceil})\cr
&<3c\cdot3^{\lg n}\cr
&=3c\,n^{\lg3}.\cr}$$

$$\left\{
\eqalign{\alpha&=f(z)\cr \beta&=f(z^2)\cr \gamma&=f(z^3)\cr}
\right\}
\qquad
\left\{
\eqalign{x&=\alpha^2-\beta\cr y&=2\gamma\cr}\right\}.$$

%% page 192

$$\eqalignno{
(x+y)(x-y)&=x^2-xy+yx-y^2;&(3)\cr
&=x^2-y^2;&(4)\cr
(x+y)^2&=x^2+2xy+y^2.&(5)\cr}$$

%% page 192

$$\eqalignno{
\biggl(\int_{-\infty}^\infty e^{-x^2}\,dx\biggr)^2
&=\int_{-\infty}^\infty\int_{-\infty}^\infty e^{-(x^2+y^2)}\,dx\,dy\cr
&=\int_0^{2\pi}\int_0^\infty e^{-r^2}\,dr\,d\theta\cr
&=\int_0^{2\pi}\biggl(-{e^{-r^2}\over2}
\biggl|_{r=0}^{r=\infty}\,\biggr)\,d\theta\cr
&=\pi.&(11)\cr}$$

%% page 197

$$\prod_{k\ge0}{1\over(1-q^kz)}=
\sum_{n\ge0}z^n\bigg/\!\!\prod_{1\le k\le n}(1-q^k).$$

$$\sum_{\scriptstyle0< i\le m\atop\scriptstyle0<j\le n}p(i,j) \,\ne
%
% $$\sum_{i=1}^p \sum_{j=1}^q \sum_{k=1}^r a_{ij} b_{jk} c_{ki}$$
%
\sum_{i=1}^p \sum_{j=1}^q \sum_{k=1}^r a_{ij} b_{jk} c_{ki} \,\ne
%
\sum_{{\scriptstyle 1\le i\le p \atop \scriptstyle 1\le j\le q}
\atop \scriptstyle 1\le k\le r} a_{ij} b_{jk} c_{ki}$$

$$\max_{1\le n\le m}\log_2P_n \quad \hbox{and} \quad
\lim_{x\to0}{\sin x\over x}=1$$

$$p_1(n)=\lim_{m\to\infty}\sum_{\nu=0}^\infty\bigl(1-\cos^{2m}(\nu!^n\pi/n)\bigr)$$

%% new extra stuff imported from math test

$$\widehat{s}, \widehat{ss}, \widehat{sss}, \widehat{ssss}, 
\widehat{sssss},
%
\widehat{f}, \widehat{ff}, \widehat{fff}, \widehat{ffff}, 
\widehat{fffff}$$

$$\widetilde{s}, \widetilde{ss}, \widetilde{sss}, \widetilde{ssss}, 
\widetilde{sssss},
%
\widetilde{f}, \widetilde{ff}, \widetilde{fff}, \widetilde{ffff}, 
\widetilde{fffff}$$

% %%% %%% %%% %%% %%% %%% %%% %%% %%% %%% %%% %%% %%% %%% %%% %%%

$$
\midint f(x) \, dx \quad
\bigg\largeint f(x) \, dx \quad
\Bigg\largeint f(x) \, dx \quad
\biggg\largeint f(x) \, dx \quad
\Biggg\largeint f(x) \, dx $$

% %%% %%% %%% %%% %%% %%% %%% %%% %%% %%% %%% %%% %%% %%% %%% %%%

$$\int_0^1f(x)\,dx={\sqrt{3}\over2} \neq {\sqrt{2\pi}\over\sqrt{3}}$$

% %%% %%% %%% %%% %%% %%% %%% %%% %%% %%% %%% %%% %%% %%% %%% %%%

$$\sqrt{x}, \sqrt{\pi}, \sqrt{x^2}, \sqrt{{a \over b}}, 
\sqrt{{a^2+b^2\over a^2 -b^2}}, 
\sqrt{\sum_{i \neq j}^{n < m}{a^2+b^2\over a^2 -b^2}}
$$

$${(x+10y)(x-10y)\over x^2-100y^2} = 1 + {a+{x\over y}+c \over 2 + {5^2\over\epsilon_2}-9}$$

$$30^{\circ}, 60^{\circ}, 90^{\circ}, 120^{\circ}$$

$$\sum_{i=1}^{n} \int_0^x f(x)\,dx = \root n+1\of{a^n+b^n}
= {\pi\over 2}$$

$$\overbrace{x+y} = \underbrace{x+y} = x_{n-2}^{i}$$

$$\overline{x+y} = \underline{x+y} = x_n^{-2}$$

$$\left( \left( \left( \left( \left( \left( \left( x 
\right)^2 \right)^2 \right)^2 \right)^2 \right)^2 \right)^2 \right)^2 
\quad \quad 
\left[ \left[ \left[ \left[ \left[ \left[ \left[ x 
\right]^2 \right]^2 \right]^2 \right]^2 \right]^2 \right]^2 \right]^2$$

$$\pmatrix{
A&B&C&D&E&F\cr
G&H&H&I&J&K\cr
L&M&N&O&P&Q\cr
R&S&T&U&V&W\cr} =
\left[\matrix{
A&B&C&D&E&F\cr
G&H&H&I&J&K\cr
L&M&N&O&P&Q\cr
R&S&T&U&V&W\cr}\right].
$$

$$\pmatrix{\pmatrix{\pmatrix{a&b\cr c&d\cr}&
               \pmatrix{e&f\cr g&h\cr}\cr
             \noalign{\smallskip}
             0&\pmatrix{i&j\cr k&l\cr}\cr}&
\pmatrix{\pmatrix{a&b\cr c&d\cr}&
               \pmatrix{e&f\cr g&h\cr}\cr
             \noalign{\smallskip}
             0&\pmatrix{i&j\cr k&l\cr}\cr}\cr
\pmatrix{\pmatrix{a&b\cr c&d\cr}&
               \pmatrix{e&f\cr g&h\cr}\cr
             \noalign{\smallskip}
             0&\pmatrix{i&j\cr k&l\cr}\cr}&
\pmatrix{\pmatrix{a&b\cr c&d\cr}&
               \pmatrix{e&f\cr g&h\cr}\cr
             \noalign{\smallskip}
             0&\pmatrix{i&j\cr k&l\cr}\cr}\cr}.$$

% %%% %%% %%% %%% %%% %%% %%% %%% %%% %%% %%% %%% %%% %%% %%% %%%


\challenge

\vskip .25in % \vskip 0.5in

%% Texture

$$\hbox to \hsize{\cleaders\vbox to .5in{\cleaders\hbox{\TeX}\vfil}\hfil}$$

\bye
