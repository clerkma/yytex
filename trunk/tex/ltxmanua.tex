% Copyright 2007 TeX Users Group.
% You may freely use, modify and/or distribute this file.

% LaTeX + SliTeX Font manual % This is in YTeX

\typesize=10pt 	% for final version
% \typesize=12pt		% for proofing

% \hsize=4.66in % for letter paper version (default)
% \hsize=5.5in % for legal paper version

\font\manual=logo10
\font\sc=cmcsc10
% \font\vt=cmvtt10

\input memo.mac

\newdimen\dwidth
\newdimen\dheight

\def\showimage#1#2#3{
\dwidth=#2 \dheight=#3 
\edef\width{\number\dwidth} \edef\height{\number\dheight}
\special{insertimage: #1 \width \space \height}
}

% \def\subsubsection#1{{\bf #1}}

\def\LaTeX{{\rm L\kern-.36em\raise.3ex\hbox{\sc a}\kern-.15em
    T\kern-.1667em\lower.7ex\hbox{E}\kern-.125emX}}

\def\SliTeX{S{\sc LI}{\TeX}}

% \def\Y&Y{Y\kern-.25em\hbox{{\smllsize \&}}\kern -.25em Y}
\def\Y&Y{Y\kern-.21em\hbox{{\smllsize \&}}\kern -.23em Y}

\def\decreasepageno{\global\advance\pageno-1}
\def\newpage{\vfill\eject}

\def\METAFONT{{\manual META}\-{\manual FONT}}

\def\DVIWindo{{\sc dvi\-w}indo}
\def\DVIPSONE{{\sc dvi\-ps\-one}}
\def\REENCODE{{\sc re\-encode}}
% \def\CHARSET{{\sc char\-set}}
\def\AFMTOPFM{{\sc afm}\-to\-{\sc pfm}}
\def\AFMTOTFM{{\sc afm}\-to\-{\sc tfm}}
\def\TIFFTAGS{{\sc tifft}ags}
\def\CLEANUP{{\sc clean\-up}}

\def\TWOUP{{\sc two\-up}}

% \def\PKTOPS{{\sc pk\-to\-ps}}
% \def\DOWNLOAD{{\sc down\-load}}
% \def\MODEX{{\sc mod\-ex}}
% \def\SERIAL{{\sc ser\-ial}}

\def\ANSI{{\sc ansi}}
\def\ASCII{{\sc ascii}}

\def\UART{{\sc uart}}
\def\FIFO{{\sc fifo}}

\def\DOS{{\sc dos}}

\def\PSPATH{{\sc pspath}}
\def\VECPATH{{\sc vecpath}}

% \def\BIOS{{\sc bios}}
% \def\PATH{{\sc path}}

\def\AFM{{\sc afm}}
\def\INF{{\sc inf}}
\def\TFM{{\sc tfm}}
\def\PFM{{\sc pfm}}
\def\DVI{{\sc dvi}}

\def\VEC{{\sc vec}}

\def\PS{{\sc ps}}
\def\PK{{\sc pk}}

\def\PFA{{\sc pfa}}
\def\PFB{{\sc pfb}}

\def\EPS{{\sc eps}}

\def\EPSF{{\sc epsf}}
\def\EPSI{{\sc epsi}}

\def\PSFIG{{\sc psfig}}

\def\PIF{{\sc pif}}

\def\DSC{{\sc dsc}}

% \def\VM{{\sc vm}}
% \def\MS{{\sc ms}}

\def\ATM{{\sc atm}}

\def\ATMINI{{\tt atm.ini}}
\def\WININI{{\tt win.ini}}

\def\ATMQLC{{\tt atmfonts.qlc}}

\def\PSCRIPT{{\sc pscript.drv}}

\def\PFM{{\sc pfm}}
\def\PFB{{\sc pfb}}

\def\CM{{\sc cm}}

% \def\COPY{{\sc copy}}
% \def\MODE{{\sc mode}}
% \def\PRINT{{\sc print}}

% \def\XONXOFF{{\sc xon/xoff}}
% \def\DTRDSR{{\sc dtr/dsr}}
% \def\ETXACK{{\sc etx/ack}}

\def\IBM{{\sc ibm}}
\def\PC{{\sc pc}}

% \def\COMPAQ{{\sc compaq}}

\def\TUG{{\sc tug}}

% \def\SMARTDRV{{\sc smart\-drv.sys}}
\def\SMARTDRV{{\sc smart\-drv.exe}}
\def\RAMDRV{{\sc ram\-drv.sys}}
\def\RAM{{\sc ram}}

\def\AUTOEXEC{{\tt auto\-exec.bat}}

\def\UNIX{{\sc unix}}
\def\DVIPS{{\sc dvips}}
\def\DVITWOPS{{\sc dvi2ps}}
\def\DVIALW{{\sc dvialw}}
\def\DVITOPS{{\sc dvitops}}
\def\DVILASER{{\sc dvilaser/ps}}

% \def\TIFF{{\sc tiff}}

\newbox\boxa

\setbox\boxa=\vbox{
\hbox{{\bf WARNING:} Keep a copy of the original font metric files ({\PFM})}
\hbox{and the font outline files ({\PFB}) in a safe place before making}
\hbox{modifications using the utilities supplied with {DVI\-Windo}.}
}

\def\warning{\centerline{\boxit{\copy\boxa}}}

% \def\registered{{\ooalign
% 	{\hfil\raise.02ex\hbox{{\sc r}}\hfil\crcr\mathhexbox20D}}}

\font\sy=sy

\def\registered{{\sy \char210}}

\def\TM{$^{\smlsize TM}$}

\chardef\bs=92

\def\coverpage{
\noheaders
% \topglue 2in
\topglue 1.5in
% \centerline{\bigggsize\bf BSR CM Fonts}
\centerline{\bigggsize\bf Extra {\LaTeX} + {\SliTeX} Fonts}
\vskip .1in
% \centerline{\bigggsize\bf Computer Modern Outline Fonts}
\vskip .2in
% \centerline{\bigsize Copyright {\copyright} 1991, 1992 {\Y&Y}. All rights reserved.}
\centerline{\bigsize Copyright {\copyright} 1991 -- 1993 {\Y&Y}, Inc. All rights reserved.}

\vskip 3in

% This is the version for printing with the logo

\centerline{\hbox to 1.33in{\special{illustration c:/dvitest/y&ylogo.eps}\hfill}}

% This is the version for showing the logo on screen

% \centerline{\hbox to 1.33in{\showimage{y&ylogo.tif}{1.33in}{2in}\hfill}}

\vskip 1in

\centerline{\bigsize {\Y&Y}, Inc. 45 Walden Street, Concord MA 01742, USA}

\vskip .02in

\centerline{\bigsize 
% (800) 742--4059 --- 
(508) 371--3286 (voice) ---
(508) 371--2004 (fax)}

\newpage
\decreasepageno

\noheaders
\topglue 2in
\hbox{ }	% to get blank page
\newpage
\decreasepageno
}

\coverpage  % comment out for draft version

\noheaders

% \versorightheader={COPYRIGHT {\copyright} 1991, 1992 {\Y&Y}}
\versorightheader={COPYRIGHT {\copyright} 1991 -- 1993 {\Y&Y}, Inc.}

\nsection{Extra {\LaTeX} + {\SliTeX} Outline Fonts }

The extra {\LaTeX} + {\SliTeX} Adobe Type~1
outline fonts from {\Y&Y} come on one diskette.
Included are:

\vskip .05in

\bpar The actual outline font files themselves
---- `Printer Font Binary' (PFB) format for IBM PC, 
`Printer Font ASCII' (PFA) format for Unix/NeXT, 
or Mac outline font files for Macintosh);

\vskip .05in

\bpar Platform specific metric files 
(`Printer Font Metric' (PFM) Windows metrics for IBM PC, 
or `screen fonts' for Macintosh);

\vskip .05in

\bpar Human readable `Adobe Font Metric' (AFM) files.

\vskip .05in

\noindent
The AFM files are not needed by Adobe Type Manager (ATM),
but are required by Display PostScript Systems. 
Note that
the AFM files are the ultimate repository of font metric information,
since all other formats, including PFM and TFM are incomplete.

\nsubsection{Purpose of the Fonts}

{\LaTeX} uses the basic Computer Modern fonts for most
purposes, but also rely on a few extra fonts, 
such as the `line' and `circle' fonts used in the 
`picture environment' for drawing simple diagrams.
{\SliTeX} also needs sans serif fonts in the `{\tt lcmss}' family.
In addition, for `color' work, {\SliTeX} uses some `invisible' fonts,
which have the same metrics as the corresponding normal fonts, 
but have no visible glyphs.

Altogether, the extra {\LaTeX} + {\SliTeX} font set includes:

\vskip .05in

\bpar {\LaTeX} `line' fonts `{\tt line10}' (regular) 
and `{\tt linew10}' (bold);

\vskip .05in

\bpar {\LaTeX} `circle' fonts `{\tt lcircle1}' (regular) 
and `{\tt lcirclew}' (bold);

\vskip .05in

\bpar {\LaTeX} `symbol' fonts in six design sizes:
`{\tt lasy5}', `{\tt lasy6}', `{\tt lasy7}', `{\tt lasy8}', `{\tt lasy9}',
`{\tt lasy10}' (regular), as well as `{\tt lasyb10}' (bold);

\vskip .05in

\bpar {\SliTeX} sans serif fonts:
`{\tt lcmss8}' (regular), `{\tt lcmssb8}' (bold), and `{\tt lcmssi8}' (italic);

\vskip .05in

\bpar `Invisible' versions of the {\SliTeX} sans serif fonts:
`{\tt ilcmss8}' (regular), `{\tt ilcmssb8}' (bold), and `{\tt ilcmssi8}' (italic);

\vskip .05in

\bpar `Invisible' versions of Computer Modern math fonts:
`{\tt icmmi8}' (math italic),
`{\tt icmsy8}' (math symbol),
`{\tt icmex10}' (math extension), 
as well as `{\tt icmtt8}' (fixed width text), and
`{\tt ilasy8}' ({\LaTeX} symbol).

\vskip .05in

\bpar Three design sizes of the `logo' fonts used for typesetting
the word \METAFONT: % {\manual METAFONT}:
`{\tt logo8}', `{\tt logo9}', `{\tt logo10}', as well as
`{\tt logobf10}' (bold), and `{\tt logosl10}' (slanted).

\vskip .05in

\noindent
The `logo' fonts are not strictly part of the extra {\LaTeX}
+ {\SliTeX} font set, but are included here for convenience.
To conserve space and limit the length of font menus you may
wish to not load these unless they are really needed.
The same goes for the `invisible' fonts.

The fonts are all fully hinted (both character-level and font-level), and so
will render well even on low-resolution (on screen) and medium-resolution
(laser printer) devices.  The fonts are completely compliant with the Type~1 
specification and, in addition, are ATM compatible (a tighter constraint).  
They also include `work arounds' for numerous bugs in `clone' interpreters.

It may interest you to know that the {\LaTeX} fonts and the `logo' fonts
were converted directly from the \METAFONT\  % {\manual METAFONT} 
source by {Y\&Y} to ensure
perfect accuracy.  This was possible because the {\LaTeX} fonts and `logo'
fonts use \METAFONT\  % {\manual METAFONT} 
in a rather restricted way.
Outlines for the three sans serif {\SliTeX} fonts were created by
Blue Sky Research.

\nsection{Installation --- IBM PC compatibles}

For use with Adobe Type Manager (ATM) in IBM PC Windows, simply install using
the ATM control panel.  When not using ATM, install instead using the Windows
PostScript driver.  For use with DVIWindo in Windows, install using ATM.

For use with DVIPSONE, install using ATM as above, or, when not using ATM,
simply copy the PFB files to a directory where DVIPSONE expects outline fonts
to reside (typically `{\tt c:\bs psfonts}').  Metric files are not needed in this case.

With some other applications, you will need to determine where they keep
outline font files (PFB or PFA) and font metric files (PFM or AFM).

\nsection{Installation --- Macintosh}

Install for use with Adobe Type Manager (ATM).
Details depends on whether you are using 
System 6, System 7.0, or System 7.1.
For System 7.1, drag the outline fonts themselves to the
{\it closed\/} system folder, then drag the `screen fonts'
to the closed system folder.
Refer to the documentation that comes with ATM for details.

Note that a font utility like `SuiteCase II' makes font installation
and font removal much more convenient.

\nsection{Installation --- UniX/NeXT}

You can use Adobe's `font installer' application if you have it.
Alternatively, use the shell script `{\tt nextinst.bat}' supplied with the fonts.
Read the printed instructions (file `{\tt nextfont.txt}')
supplied separately for installation
on NeXT with Display PostScript (DPS).

For Display PostScript for X-Windows in Solaris 2.3 please
refer to the instructions supplied with that DPS system.

DOS diskettes can be read on Sun workstations using MTOOLS.
Use  `{\tt man mtools}'
to see what commands are available for manipulating DOS diskettes.
You can for example do `{\tt mdir a:}' to see what files are on a diskette
and `{\tt mcopy a:*.*}' to copy all the files in the top-level directory
on the diskette to your current directory.

If you can't find `mtools' on your system, try the InterNet.
For example, look in `{\tt /src/unixdos/mtools}' on `{\tt world.std.com}'.

Note that MTOOLS interprets DOS file names as all uppercase.  This is OK
for the font files themselves, since the fonts have all upper case names
anyway.  But you may find it convenient to rename some other files.
For example, having extensions `TFM' (instead of `tfm') on `TeX Metric Files'
files may not work with your {\TeX} implementation.

For use with DVIPS, fonts in Type~1 format have to be listed in the
file `{\tt psfonts.map}' (typically found in `{\tt /usr/lib/tex/ps}'). If the
fonts have PostScript FontNames different from the TFM file name,
then both TFM file name and PostScript FontName must be mentioned on
the line listing a particular font (see `{\tt psfonts.lt}').

DVIPS can be asked to `automatically download' the outline font for
the duration of the print job.  In this case each line added to
`{\tt psfonts.map}' should end with `<' followed by the file name of the
corresponding PFA (or PFB) file.  DVIPS will typically look for the
outline font files in PFA (or PFB) format in the directory
`{\tt /usr/lib/tex/ps}'  (see `{\tt psfonts.ltx}' or `{\tt psfonts.ltz}').

% Older versions of DVIPS had some problems with fonts that have
% repeat encodings (a character appearing in more than one place
% in the encoding vector).  Please use the latest version of DVIPS.

\nsection{Remapping of character codes}

Since many applications cannot effectively use character codes 0--31 and 127,
% cannot be generated from the keyboard on the Macintosh, 
these characters have been remapped to the 161--196 range.
Character codes in the range 0--9 can be accessed via codes 161--170, while
10--32 can be accessed via 173--195, and 127 can be accessed via 196.

All of these codes can be generated by suitable key chords using the shift
and option keys on the Macintosh.  On IBM PC compatibles these characters can
be generated by holding down the `{\tt Alt}' key, typing `{\tt 0}' and the numeric code
on the numeric key pad (with `{\tt Num Lock}' on).  
Of course, you don't need to use these methods when
the fonts are used with TeX.

\nsection{IBM PC utility programs supplied with the PC fonts}

PFBtoPFA converts from the compact binary form to the printable ASCII format.

PFAtoPFB converts from printable ASCII format to the compact binary format.

PFBtoPFA and PFAtoPFB come in handy when a change is to be made to an outline
font file.  A PFB file cannot normally be edited directly, since it contains
binary length fields.

SPACIFY can be used to put a `{\tt space}' character in character code
position 32.  This is useful for use of these fonts with applications
other than TeX.  Please read `spacify.txt' for additional details.

REENCODE permanently changes the encoding vector in a PFB or PFA file.
The encoding vector is stored in a file that can be specified on the
command line.  Each line of the file simply contains a number followed by a
space or tab and the PostScript name of the corresponding character.
For example:	`{\tt 65	A}'.  This provides a more efficient way of reencoding
a font than directly editing the encoding vector in the PFA file.

Some applications (such as DVIPSONE) can reencode a font on the fly, and so
don't require permanent reencoding of a font.  However, the Windows
PostScript driver (and Adobe Type Manager) do not provide for reencoding,
so REENCODE can come in handy when a font is used with a Windows application.

DOWNLOAD can be used to permanently download a font to the printer, to
reset the printers virtual memory, or to print out a list of printer resident
fonts.  Note that some of the PostScript commands used by DOWNLOAD are not
standardized -- they work on the Apple Laser Writer and its relatives, but
may not work on some `clone' printers.  It is therefore preferably to use
specialized utility programs supplied with the printer instead of DOWNLOAD.

SERIAL can be used to send PostScript files to a printer connected over
a serial link that does not support hardware hand-shaking  (The DOS
COPY command can be used for printers connected over serial links that
support hardward hand-shaking).  Serial is also handy when the information 
sent back over the serial line from the printer is of interest.

All utility programs from {\Y&Y} give a brief usage summary when invoked
without arguments.  To see more detail, use the command line flag `{\tt -?}'.
So to see what arguments DOWNLOAD takes, invoke it as follows:

\verb|	download -?|

% leaving out the bullshit about automatic versus manual downloading...

\nsection{PostScript `Clone' Interpreters Bugs and Test Files} 

Most clone PostScript interpreters have some `misfeature' or other,
particularly when it comes to Type~1 font interpretation.  Such interpreters
may work prefectly well on plain vanilla Adobe text fonts, yet fail on more
complex fonts that fall well within in the Type~1 specification --- and
are even ATM compatible --- which is a tighter constraint.

% PostScript Interpreter Test Programs:

In the PS subdirectory you will find nine PostScript test programs, and the
error handler `{\tt ehandler.ps}'.  These test files may come in useful should
there be a problem printing PostScript files using these fonts.  The test
programs check for various known bugs in `clone' interpreters, particularly 
as regards Type~1 font interpretation.  There is a short `{\tt readme.txt}' file
in the same directory that describes the tests in more detail.

\nsection{Problems with ATM for Windows}

Older versions of ATM for Windows (ATM 2.5 and before) have some bugs
such as being unable to handle repeated encodings.  We recommend:

\vskip .05in

{\bf Get the latest version of ATM}

\vskip .05in

\noindent
A character that appears more than once in the
encoding vector can only be accessed through the first mentioned encoding.
So if, for example, the character `{\tt suppress}' appears in character code
positions 32, 128, and 195, and the `{\tt dup 195 /space def}' appears first in the
encoding vector, then that character can only be accessed using 
`{\tt Alt 0 1 9 5}', not by typing the `{\tt space}' key, 
or using `{\tt Alt 0 1 2 8}'.  
What is worse, old versions of ATM go into `invisible ink mode' if you try to
access the character using one of these alternatives --- characters in the
same string after this point will not be shown on screen.

Similarly characters in code positions 0 - 31 should be accessed through
their alternate codes (161 - 170 or 172 - 194) in non-TeX applications.

We also recommend that you use the utility `{\tt spacify}' to put a `{\tt
space}'  character in code position 32.  This is not neccessary in the case
of the Macintosh version of the fonts since the `coordinated' fonts already 
have a space character in character code position 32.

\nsection{Problems with Adobe Font Downloaders and Adobe Font Foundry}

On IBM PC compatibles, 
we cannot recommend Adobe Font Downloaders, since they are slow and not very
user friendly --- use the DOWNLOAD utility instead if possible.  If you are
forced to use a downloader from Adobe, be aware that some of them will
*only* recognize fonts that have file names that have been padded out to a
full eight characters using the underscore character.  These downloaders also
will not recognize fonts that have more than six real characters in the file
name (i.e. names with fewer than two trailing underscore characters).
The file names of both PFB and PFM files must have this form.

We do not recommend making bitmapped versions of these fonts, but if you
do use Adobe Font Foundry (supplied with any Adobe font set), then please
note that the foundry requires not only PFB and PFM files, but also AFM
and INF files.  The default directories for these files are
`{\tt c:\bs psfonts}',
`{\tt c:\bs psfonts\bs pfm}', `{\tt c:\bs psfonts\bs afm}', 
`{\tt c:\bs psfonts\bs fontinfo}'.  
ATM does  {\it not\/} copy the AFM and INF files from the font distribution
diskettes (since it doesn't need them).  You'll have to copy them to the appropriate
directories. 
Also note that the font foundry program also does recognizes only fonts
with names that have been padded out with underscore characters...

\nsection{Problems with old versions of DVIPS}

If you have an older version of DVIPS, the best advice is:  

\vskip .05in

{\bf Get the latest version of DVIPS!}

\vskip .05in

\noindent
Seriously, it will be well worth your while.  DVIPS is constantly evolving
and improving, as well as adapting new features found in other DVI drivers. 
Here is some advice should you be FORCED to use an older version:

(1)	Older versions of DVIPS can only handle Type~1 fonts in PFA format.
	They cannot unpack the more compact PFB format.  To deal with this,
	you will have to run the PFB files through the utility PFBtoPFA.

(2)	Older versions of DVIPS have a problem with fonts that use repeated
	encodings, because DVIPS modifies the metrics of `PostScript' fonts
	based on information in the TFM files.  Many fonts for use with TeX
	use repeated encodings because (a) TeX expects certain characters to
	be in the control character range (0 - 31) and at the same time (b)
	many applications cannot access characters in that range.

DVIPS redefines the metrics of a font by referring to the TFM file in a way
that causes the second appearance of a character in the encoding to end up
with zero width.  This means that ligatures and Greek characters will
overprint, since the higher numbered encodings are listed first 
(in order to deal with the repeat encoding `misfeature' of ATM for Windows).  

% *	So please get the latest version of DVIPS!

\nsection{Using the extra {\LaTeX} + {\SliTeX} fonts with GhostScript}


Please add the contents of the file fontmap.add to the end of the 
GhostScript FontMap file in order to use the extra LaTeX + SliTEX fonts with
GhostScript. 


\end

\endinput












The fonts are ready for installation using Adobe Type Manager (ATM) --- 
as described in the next section. 

You may be using these fonts with an application that has its own
installation procedure, in which case % and so 
the following may not apply.  
For example, if you will be using the fonts only with
% for use with 
{\Y&Y}'s DVI\-PS\-ONE, then just copy all of the PFB files to some
convenient directory (\verb@c:\psfonts@ say) on your hard disk
and set the environment variable PSFONTS to that directory.

\vskip .05in

{\narrower \sl

\noindent
You can pretty much ignore most of this document
if you will be using the % Blue Sky Research Computer Modern outline fonts 
extra {\LaTeX} + {\SliTeX} outline fonts from {\Y&Y} only with DVI\-PS\-ONE.

}

% explain what to do with DVIPS ???

\vskip .05in

\noindent
But, if you plan to use these fonts with Windows applications
(such {\Y&Y}'s DVI\-Windo, MicroSoft Word for Windows, 
Aldus PageMaker, Adobe Illustrator etc.) 
then please read at least sections 2 and 3.  % want to 

\nsection{Font installation using Adobe Type Manager (ATM)}

To install the outline fonts for use with MicroSoft Windows, % ~3.0 or 3.1
insert the diskettes containing PFB
and PFM files into a diskette drive.  
Launch Windows and double-click on the ATM control panel icon
(in the `Main' program group).
Select `{\tt Add}' and choose the fonts to be installed from the diskette.
% Install the desired fonts from the ATM control menu.  
You can select several fonts at once by holding down the control key
while clicking on font names ---
or you can hold the mouse button down and drag the cursor to select
a range of fonts.  After selecting fonts, click `Add' again.
Repeat the process using the second diskette.  
% Note that with older versions of ATM, 
% the newly installed fonts will not be accessible from Windows
% applications until you exit and relaunch Windows.

During the installation process, the PFB font files will be copied to the
directory \verb@c:\psfonts@, while the PFM files will be copied to 
\verb@c:\psfonts\pfm@ (assuming ATM uses the default directories).  
The installation procedure also adds `softfont' entries to each section
of `{\tt win.ini}' file designated for a PostScript printer 
% in the {\tt win.ini} file in your Windows directory 
(make sure to select `Install as Auto-Download Fonts' when adding fonts
using ATM).  If you later add another PostScript printer,
% (or add another port for an existing PostScript printer), 
you will need to reinstall the fonts.
You can do this from the original diskettes or from the PFM file
directory using ATM as above.
% manually copy the softfont entries into the corresponding section of {\tt
% win.ini}. 
% Alternatively, reinstall the fonts after adding the new printer driver.

By the way, older versions of ATM do not do a good job maintaining the
{\tt win.ini} file. % very well
Make sure to use the latest version of ATM (at least 2.5 or 2.6).
%
% They % Older version of ATM 
% do {\it not} remove softfont entries from
% {\tt win.ini} when a font is `{\tt Removed}'.
% You can use a text editor to manually delete all softfont entries in
% {\tt win.ini} relating to the deleted font.
% Also, older versions of ATM create duplicate entries in {\tt win.ini}.
% when a font is reinstalled.
For additional information, please refer to the ATM manual.

It is best to always use ATM to `Add' or `Remove' fonts.
%
In any case, {\it never} replace existing PFB or PFM files while Windows is
running --- doing this can cause ATM to overwrite random parts of memory.
%
% Also, to install a new version of a font, it may appear to be easier to
% simply replace the PFB file in \verb@c:\psfonts@  and the PFM file in  
% \verb@c:\psfonts\pfm@ with the new versions.
% If you use this short-cut technique, however, make sure to delete ATM's
% `quick load cache' file ({\tt atmfonts.qlc}) in the  \verb@c:\psfonts@
% directory ---  ATM will rebuilt it the next time you launch Windows.
% If you cannot find {\tt atmfonts.qlc}
% then you most likely have a very old version of ATM.

If you would like to check on what fonts are installed, double
click the ATM icon in the `Main' program group.
Alternatively, look in the file {\tt atm.ini} in your Windows directory.
Each line of this file gives the Windows name of a
font and both the PFB and PFM file names (as well as the style of the font
% whether the font is considered 
--- BOLD or ITALIC---or both).  
This is not a very reliable list, by the way, since fonts are listed even if
the corresponding PFB and PFM files are missing or corrupted.

If you have DVI\-Windo, it is better to select
`{\tt Show Font}' from the `{\tt Font}' menu.  
All installed scalable outline fonts 
(Type~1, TrueType, FaceLift and so on) are listed, 
and their character sets can be inspected. 
Problem fonts (missing or % with obviously 
corrupted PFB or PFM files) will not be listed. 
Alternatively use the Windows applet `Character Map' in the `Accessories'
program group.

Note that with ATM installed, Type~1 fonts can be used {\it not\/} just with
PostScript printers, but any of hundreds of different printers with Windows
drivers, including such `printers' as fax boards and % Adobe 
Acrobat PDF Writer.

% {\bf Note:} Some applications, such as older versions of MicroSoft Word for
% Windows, need to be prodded to update their cached font list when new fonts
% are installed by ATM. 
% Start Word for Windows % after installing the fonts 
% and choose `{\tt Printer Setup}'
% from the `{\tt File}' menu; 
% then click on the name of your printer and select `{\tt OK}'.

\nsection{`Automatic' versus `manual' downloading of fonts}
	
When you use Type~1 fonts normally with Windows applications, 
then the fonts needed by a particular print job
will be sent to the printer as part of the PostScript file.
These fonts will be  % automatically removed
disappear from the printer at the end of the print job.  
This is referred to as `automatic' or `temporary' downloading.

If you use the same fonts all the time, and if your printer has enough
memory, and if you are not sharing the printer with other users,
% who might object, 
then it may make sense to download the fonts `manually' or
`permanently' ahead of time.  
Fonts that are downloaded `manually' remain resident in the printer until
the printer is rebooted or power cycled.  
This saves the time for downloading the fonts with each print job
%  While this may save some time, 
--- although it is somewhat less convenient. It is also not
really necessary if your printer driver uses techniques such as 
`partial font downloading' --- as {\Y&Y}'s DVI\-PS\-ONE does. 

% Fonts can be either downloaded `manually' (i.e. permanently) or
% `automatically' (i.e. only for the duration of a particular print job).
% Fonts that are downloaded manually remain resident in the printer until
% the printer is rebooted or power cycled.  

Usually applications using outline fonts come with utilities for downloading.
You can, for example, use one of the Adobe downloader 
(PCSEND for serial ports, PSDOWN for parallel ports, or WINDOWN for Windows)
to preload the CM fonts you plan to use.  
Alternatively, use the utility DOWNLOAD supplied with the CM fonts.  
To see what command line arguments DOWNLOAD takes, invoke it using 
`{\tt download -?}'.

% Show serverdict begin 0 exitserver line ?

Fonts downloaded `permanently' are treated exactly like printer
resident fonts. 
Note, however, that the virtual memory (VM) on some older printer
% 's virtual memory (VM)  
may be filled by just a handful of fonts.
%
% There is thus an advantage to `automatic downloading' --- particularly if the
% application uses {\it partial font downloading} 
% (presently only {\Y&Y}'s DVI\-PS\-ONE uses partial font downloading).  

If you are using ATM~2.0 or later, then the installation procedure will set
things up for `automatic downloading'
if you select `Install as Auto-Download Fonts' when adding fonts.
If you do {\it not\/} want fonts to be automatically downloaded,
then make sure to uncheck this box when installing the fonts.
Alternatively, you can alter the behavior of the
% to be downloaded automatically by the 
Windows PostScript printer driver by
removing the reference to a font's PFB file in all `softfont' entries
referring to that font in your `{\tt win.ini}' file.

% With earlier versions of ATM, to provide for automatic downloading of a
% particular font when you are using Windows, you need to make a small change
% in the file {\tt win.ini} in your Windows directory.  
% Simply extend the softfont
% entry in {\tt win.ini} with the path to the PFB file.  
% For example, change:
% (recent versions of ATM may already do this for you):

% \vskip .03in

% 	\verb@softfont29=c:\psfonts\pfm\cmr10.pfm@

% \noindent to

%	\verb@softfont29=c:\psfonts\pfm\cmr10.pfm,c:\psfonts\cmr10.pfb@

% \vskip .03in

% \noindent
% where it is assumed that the PFB font file resides in 
% \verb@c:\psfonts@.

% Note that there should be no space after the comma above, and the order of
% entries should be as shown.  Also remember that there is a softfont entry
% for every printer/port combination you have `installed' in Windows, 
% so this alteration may need to be made in more than one place in {\tt win.ini}.

\nsection{Accents and hyphenation in non-English languages}

As you may note by looking at the AFM files, 
the BSR CM text fonts from {\Y&Y} come with
the set of 58 `standard' accented characters found in 
ISO Latin~1, Windows ANSI, and the Macintosh standard roman encoding.  
These are not characters normally found in Computer Modern fonts, 
and hence they do not appear in the encoding.  
Refer to the AFM files to find out exactly which accented characters
are available.  

To gain access to these pre-built accented characters you may wish to create
new fonts  % (with different names) 
using the utility REENCODE, or use some other method for reencoding the font
--- such as on-the-fly reencoding provided for by the font substitution
mechanism in {\Y&Y}'s DVI\-PS\-ONE.
To see what command line arguments REENCODE takes, invoke
it using `{\tt reencode -?}'.  
Reencode requires an encoding vector file ---
this is % simply 
a file containing lines like 

\vskip .03in

65 \quad A

\vskip .03in

\noindent
giving the mapping from
a character number to a character name.  
Several sample encoding vector files (extension `{\tt .vec}')
may be found on diskette B. % (with extension `{\tt .vec}').

The rationale for including the accented characters ---
despite the fact that they can be constructed using {\TeX}'s
\verb@\accent@ macro from the base and accent characters --- is that a word
with an accented character that was constructed using \verb@\accent@
contains explicit kerning, and so is {\it not\/} 
hyphenated properly by {\TeX}.  
This means that providing for hyphenation of words in non-English
languages is awkward. % , if not impossible.  
The problem evaporates if the font contains the accented characters directly.

It does, however, take a bit of work to fully exploit this feature, 
since the plain {\TeX} and {\LaTeX} macros % set up 
for these accented characters  (such as \verb@\aa@)
have to be replaced with new ones using the appropriate character code. 
%
Here is an example:

\vskip .03in

% \noindent
Presently: \quad \verb@\def\aa{\accent'27a}@ $\,$ ({\TeX} book, page 356)

% \noindent
Change to: $\,$ \verb@\def\aa{\char225}@ $\,$	(assuming `{\tt aacute}' is decimal 225)

\vskip .03in

\noindent
% We recommend that the remapped font maintain the original encoding below 128,
% and use some `standard' encoding above 128.  
What encoding to use is a matter of taste.
% When working with Computer Modern fonts in {\TeX} it may make sense to
% retain the encoding below 128.
When working in Windows, Windows ANSI may be a reasonable choice.
Another possibility is the encoding agreed upon at the Cork, Ireland TUG
meeting in 1990.  There are some constraints on reencoding fonts
containing composite characters (see below).
%  --- for additional details look in the file {\tt composite.txt}.

A good way of using pre-built % even better way of dealing with 
accented characters is to build pseudo-ligatures into the TFM files so that
an accent character followed by a letter is automatically replaced with
the corresponding accented letter (provided it exists in pre-built form).
So, for example, a `ligature' may be included that maps 
`{\tt dieresis}' followed by `{\tt a}' 
to `{\tt adieresis}'
(NOTE: we are referring here to the actual `{\tt dieresis}' accent, 
{\it not\/} `{\tt quotedbl}').
The utility AFMtoTFM  does this when given suitable
command line flags.  For example, 

{\tt afmtotfm -vadjx -c=tex256 cmr10}

\noindent
creates a TFM file set up for the 256 character {\TeX} `Cork' encoding with
suitable pseudo-ligatures.   To make use of the new pseudo-ligatures
transparently with existing {\TeX} source files, one can simple modify the
definitions of the accent macros.  For example, in `{\tt plain.tex}'
we find:

\verb|\def\"#1{{\accent"7F #1}}|

\noindent
which invokes {\TeX} \verb|\accent| macro with argument 127,
(the character code of `dieresis' in Computer Modern's `TeX text'
encoding).  Replace this with %  We above

\verb|\chardef\"=4|

\noindent
where 4 is the character code of `{\tt dieresis}' in the `Cork' encoding.

\nsection{Constraints on reencoding fonts}

The composite characters in the BSR CM fonts from {\Y&Y} 
are constructed using a standard Type~1 Char\-String operator called
`StandardEncoding Accented Character' % ({\tt seac}), 
as are % all 
composite characters in standard Adobe text fonts.
This works as it should in most PostScript interpreters that support Type~1
fonts. %  when the font is used with Adobe's StandardEncoding vector.  

Some PostScript interpreters, however, 
may not work properly when the font is reencoded.  
The symptom of this problem is an `{\tt invalidfont}' error in `{\tt show}' 
when a string is rendered that contains a composite character.
% that is affected by this problem. 
%
Many non-clone PostScript interpreters require that base characters and accents
must be in the encoding vector (i.e. it is not sufficient that there are 
{\tt Char\-Strings} for these glyphs).  So when reencoding a font, make
sure that all base characters and accents actually show up somewhere
in the encoding.
%  (that is, it is {\it not\/} enough that the font
% contain the character programs for the base and accent characters).

ATM for Windows has a different problem.  It has hard-wired positions
for base characters and accents.  On screen rendering will be affected
when accents appear in other positions.  
% (For a real blast try an encoding
% where `{\tt ecircumflex}' is in position 136, as it is in IBM OEM encoding).
What is more, the place where ATM expects to find accents in the encoding
depends on which version of ATM is being used! % you are using!
(If you have DVI\-Window, you can get around this problem by using the
utility SAFESEAC on the PFB files).
%
% Other PS interpreters `do the right thing' --- that is map the accent
% character number into a character name using StandardEncoding, and then look
% up that name in the reencoded font.  Somewhat surprisingly, many of the
% `clone' interpreters work correctly, while the non-clone interpreters do not
% (see discussion of problems with clone interpreters below).  
%
% The moral of the story is:  check first that your printer can handle
% remapped fonts with composite (accented) characters before investing a lot
% of time redoing the {\TeX} macros for accented characters.  This can be done
% even with one of Adobe's own fonts:  Use the REENCODE utility to move the
% accents from their standard position and print some text containing
% accented characters using the modified font (see next section for method 
% for creating such a test file).
%
For additional details see the file {\tt composite.txt}.

\nsection{Printer Font ASCII from Printer Font Binary format}

For the PC, the fonts are supplied in the compact binary (PFB) format which
conserves disk space.  These font files cannot be sent directly to the
printer in this format, since the PostScript interpreter is expecting 
`ASCII' % \hbox{7-bit}
characters only.  The fonts have to be unpacked into the more verbose
hexadecimal (PFA) format on the way to the printer.  This is usually taken
care of by the application (for example, {\Y&Y}'s DVI\-PS\-ONE), or the Windows
PostScript driver ({\tt pscript.drv}).  
The downloader supplied with the fonts,
DOWNLOAD, also does the conversion, as do Adobe's downloaders.

Some applications are not able to do the unpacking, and so require fonts
in the verbose PFA format.  PFA format is normally also required when the
fonts are to be used with Display PostScript on Unix systems. % /NeXT

%% flush some of the following if space is needed.

A utility (PFB\-to\-PFA) is supplied with the fonts that unpacks 
the PFB files into PFA format.  
A second utility (PFA\-to\-PFB) can be used to convert back
to the PFB format.  
% This is useful if you want to manually change the encoding vector, or
This may be useful if you want to manually change the
font name --- something that cannot be done directly
in the PFB file, since the lengths of the ASCII section is encoded in the
binary file format, and the font will not work if the encoded length does not
match the actual length of the ASCII section of the font file. 
% (Actually, it
% is usually more convenient to just use the REENCODE utility to do this). 

The PFB\-to\-PFA utility also makes it easy to manually download a font,
and is useful for constructing simple test files.  For example, suppose
PFB\-to\-PFA has been used to create {\tt cmr10.pfa} from {\tt cmr10.pfb}. 
Then read the PFA file into an editor (in text only mode) and add the lines:

\vskip .03in

\def\nl{\hfill\linebreak}

{\narrower

\tt
\noindent
/CMR10 findfont 72 scalefont setfont\nl
100 100 moveto (This is CMR10) show showpage

}

\vskip .03in

\noindent
to the end of the file. Then use the DOS COPY command:
%  to send it to the printer:

\vskip .03in

{\tt	COPY cmrtest.ps lpt1:/b}

\vskip .03in

\noindent
(assuming that the PostScript printer is on LPT1).
%  and that the new file is called {\tt cmrtest.ps}).

\nsection{Changing the case of the PostScript FontName}

The PostScript `FontName's of these CM fonts are all upper case, in order
to be consistent with the Macintosh versions (where font name contraction
would cause lower case names to become non-unique).  Many applications,
such as DVI\-PS\-ONE, actually don't care what the FontName is  (since they
read it from the PFB file itself).  If you download the font to the
printer `permanently' ahead of time, however, then there has to be a match
between the PostScript FontName and the name used by the application.  

A utility called NAMECASE is provided that will change the case of `FontName
in a PFB or PFM file so that fonts can be used with applications that
expect lower case FontNames.  To see what command line arguments NAMECASE
takes, invoke it using `{\tt namecase -?}'.  
The FontName case should only be changed if really necessary, 
since this will reduce portability.

\nsection{Dealing with large numbers of fonts in MS Windows}

Adobe Type Manager 2.0 and later has a limit of about 900 installed fonts.
% (about 460 in earlier versions of ATM). 
In practice, however, you will encounter another limit long before this: 
the Windows PostScript printer ({\tt pscript.drv})
version 3.5 or later (supplied with Windows 3.1) has a limit of about 330
installed fonts. % The limit for {\tt pscript.drv} 
% (version 3.4 --- supplied with Windows 3.0 --- had a limit of about 165).

You may find that certain operations 
(such as launching Windows, or launching an application)
% that establishes a `Printer Device Context' immediately when loaded) 
slow down % a lot 
when a large number of fonts are installed.  In this
case it may make sense to install only those fonts that are really needed.
% When PFB files are requested, 
In the case of the CM fonts, note that diskette A contains the 40 most commonly
used fonts.  The 35 fonts on diskette B are used only rarely.

% There are some other things that can be done if this slow-down is a problem.
% One is to copy all of the PFM files to a RAM disk.  This can be done
% conveniently using a line in the {\tt autoexec.bat} file.  
% If this is done, both {\tt win.ini} and {\tt atm.ini} should be updated
% to reflect the new location of the metric files 
% (and {\tt atmfonts.qlc} should be deleted after making the change).

Also, try and get the latest version of ATM (2.5 or later)
and the latest version of the Windows PostScript driver 
(3.55 or later).  
These versions of ATM and the PostScript printer driver
cache metric information in a file, greatly speeding up Windows launch
and opening of printer device contexts. % in applications.

% Another `trick' is to start some application like MS Windows Write, and use
% `Printer Setup' from the file menu to select the printer that will be used
% in other applications later on.  Then minimize the application (turning it
% into an icon).  
% This forces the PostScript printer driver ({\tt pscript.drv}) to
% remain resident and thus speeds up creating of printer device contexts.

\nsection{PostScript `clone' interpreters}

PostScript `clone' interpreters (i.e. non-Adobe interpreters)
%  developed independently)
% that can deal with Adobe Type~1 fonts 
usually work just fine on Type~1 fonts from
Adobe, as well as fonts created using tools licensed from Adobe.  
Adobe font creation software % (i.e. {\tt BuildFont}) 
uses the Type~1 format in a very stylized way.  
Consequently, being able to properly interpret fonts in the `standard'
Adobe form does not assure that an interpreter can deal with arbitrary Type~1
fonts at all.  In fact, a font may be a perfectly valid Type~1 font, and even
be `ATM compatible' to boot (a much tighter constraint) and yet not run on
some `clone' interpreter.
% (We have yet to find a single `clone' interpreter that
% does not have at least one bug in Type~1 font interpretation).

The Blue Sky Research Computer Modern outline fonts contain numerous
work-arounds for known bugs in non-Adobe PostScript font rendering software
(In fact several highly desirable optimization had to be omitted for
this reason). 
There is one problem % are some problems that are 
that is purposefully not dealt with, however.  
This is the inability of some interpreters to deal with characters that have
widths that are not integer multiples of 1/1000 of an `em'.  All fonts from
Adobe use integer widths, but the CM fonts cannot --- since that would
violate the basic metrics of the Computer Modern fonts.

If there are problems printing with the CM fonts, use the PFBtoFPA utility
to construct a simple test file as described in the previous section and
send that to the printer, after first sending down the error handler
({\tt ehandler.ps}).  
% If there are errors reported (such as `{\tt invalidfont}' or
% `incorrect number of parameters for \verb@start_char@'), 
% or if the printer `hangs' and has to be rebooted, then you may have a
% PostScript interpreter that cannot handle non-integer character widths. 
% In this case, {\Y&Y} can supply a `simplified' font set that has rounded-off
% character widths.
% The errors introduced by this are small, since usually tiny errors
% in one direction are more or less balanced by tiny errors in the other
% direction (unless the line consists of a long string made up by
% repeating the same character). 
%
Note that there are some PostScript test files in directory `{\tt ps}' on
diskette C % (diskette D for Unix/NeXT version)
that may be helpful in pinning down clone interpreter problems.

% or if the font is mono spaced !

% When we note on the order form that you are using a printer that is known to
% have this problem, we automatically call to check whether you would prefer to
% receive the `simplified' version of the fonts.  The modified fonts have the
% letters NG after the version number in the first line of the PFB file.

\nsection{Using the CM outline fonts with DVIPS}

For use with DVIPS, append the contents of one of the files `{\tt psfonts.cm},'
`{\tt psfonts.cmx},' or `{\tt psfonts.cmz}' to DVIPS's `{\tt psfonts.map}' file
(typically found in \verb|/usr/lib/tex/ps|).  
Use `{\tt psfonts.cm}'
if fonts are `permanently' downloaded to the printer ahead of time;
use `{\tt psfonts.cmx}' if fonts are kept in PFA form;
use `{\tt psfonts.cmz}' if fonts are kept in PFB form.
If you have a recent version of DVIPS, then instead just add the line

\verb|p+ psfonts.cm|

\noindent
to the end of the `{\tt config.ps}' file used by DVIPS.   This will in effect 
append the entries in `{\tt psfonts.cm}' to the list of fonts in Adobe Type~1
format that are available to DVIPS.  Please consult your DVIPS manual.

Very old versions of DVIPS have a problem with outline fonts that
contain repeated encodings. DVIPS redefines the metrics of a font, and
characters that appear twice in the encoding can end up with zero width.
This means that ligatures and Greek characters will overprint,
as will some characters on the 0 -- 31 range, such as 
`$-$', `$+$', `$=$' etc. 
(Printer drivers --- such as DVIPSONE --- that do not refer to the TFM files
for metric information do not have this problem).

The solution is to get the latest version of DVIPS
and check the information that comes with DVIPS.
% regarding work-arounds for this problem.

% There are two work-arounds for this problem.  One is to prevent DVIPS
% from redefining the font's metrics.  
% In the file `{\tt texps.pro}' change

% \vskip .03in
% 
% {\tt /Metrics currentdict put}
% \noindent to
% {\tt /Bogus currentdict put}
% \vskip .03in
%% Mention other methods ?

% \noindent
% Another possibility --- if you are desperate and cannot get the latest
% version --- is to remove the repeated
% encodings from the fonts themselves. Simply delete the lines starting with 
% `{\tt dup 161 ... put}'
% through 
% `{\tt dup 196 ... put}' 
% after the line 
% `{\tt /Encoding 256 array}'.  

% This change should be made on a PFA file, {\it not\/} a PFB file, since
% the latter contains binary segment length codes which will be invalidated by
% any editing. If you are using PFB files, convert them first to PFA format
% using PFB\-to\-PFA, edit the PFA file, and then convert back to PFB form
% using PFA\-to\-PFB.  Of course, this is assuming that you will never want
% to access characters in the `control character' range (0--31 and 127) using
% the alternate character codes. 

DVIPS also uses a % particular 
coordinate system, and produces code that has a
tendency to trigger bugs in HP PostScript printer floating point arithmetic.
If you get occasional odd jumps in horizontal position in output from DVIPS
when printing to one of the 600 dpi HP PS printers, then make sure you have
the latest ROM revision for that printer, since early versions had an
arithmetic error in the Intel floating point library linked into the
interpreter.  While this bug has apparently been fixed in the latest HP PS
printers, a similar problem just appeared recently in new HP printers 
(These problems do not occur with PS output from DVIPSONE). 

DVIPS PS output also interacts with a problem in some NewGen printers.
As mentioned above, DVIPS redefines the metrics of fonts (without changing
the UniqueID of a font).  And DVIPS uses relative positioning most of 
the time.  nder certain circumstances, such as when it comes to a superscript
or a subscript, however, DVIPS may switch to absolute positioning.  Sudden
jumps in position can occur if errors in position have accumulated at that
point.  Some NewGen printers do not allow redefinition of font metrics
(unless the font's UniqueID is also changed). 
The result are jarring discontinuities, mostly at superscripts in subscripts
(These problems do not occur with PS output from DVIPSONE).

% explain about psfonts.map ???

\nsection{Accessing special characters and font encoding}

Alphanumeric characters can be accessed in Windows applications just by
typing the appropriate key, of course, but Computer Modern fonts also contain
many special characters that are located in `control character' positions.
Unfortunately, you can't effectively use the codes between 0 and 31 in most
Windows applications.  
So these have been remapped to positions 161--195 (see the table at the end).  
Also, note that {\TeX} uses character code 32 for Polish
`{\tt suppress}' % or `{\tt cancel}'
--- to get a real space, use character code 160
% (Ironically, the `{\tt suppress}' in CM fonts is too short and too shallow
% to be useful for constructing a real `lslash' or `Lslash').

Many Windows applications (like PageMaker) do use the kerning
information in the PFM files (although only above some user-selectable
point size).  
Unfortunately, however, Windows applications cannot deal with ligatures
(for a start, the PFM file format does not provide for ligature information).
To get the ligatures commonly used in {\TeX}, you will need to
explicitly use numeric codes derived from the table below 
(e.g. ff is 174, fi is 175, and fl is 176). 

\nsection{Accessing the `space' character}

The CM fonts can be used with any application that understands 
Adobe Type~1 fonts. 
There is a serious inconvenience though in that % most 
fonts designed to work with {\TeX} 
% the original CM fonts 
do not contain a `{\tt space}'.  
For this reason, we have added a `{\tt space}' in position 160 to help,
but it is awkward to have to type ALT `0' `1' `6' `0' just to get a space.  
One way around this problem is to interchange `{\tt suppress}' and `space'.
A program called `{\tt spacify}' can do this for you.  
Note that {\it both} the PFB and the PFM file must be modified. 
See the file `{\tt spacify.txt}' for details.
Note that you will lose the ability to construct the
% Note that this only makes sense if you will never use the 
% Polish characters `{\tt lsuppress}' or `{\tt Lsuppress}'.
Polish `{\tt lslash}' and `{\tt Lslash}'.
(Ironically, the `{\tt suppress}' in CM fonts is too short and too shallow
to be useful for constructing a real `lslash' or `Lslash' anyway).

\nsection{Accessing characters with codes greater than 127}

To gain access to the codes above 127 in Windows applications 
(when using the US English version of Windows), hold
down the `Alt' key, type 0 (important) and then the (decimal) code
number on the numeric key pad  % See the table below:
(with `Num Lock' {\it on}).	%		/* 1993/March/28 */

\vskip .1in

% \settabs 6 \columns
\settabs 9 \columns

\+TeX&Window&charname&&&TeX&Window&charname\cr

\vskip .1in

\+0&	161&	Gamma	&&&16&	179&	dotlessi\cr
\+1&	162&	Delta	&&&17&	180&	dotlessj\cr
\+2&	163&	Theta	&&&18&	181&	grave	\cr
\+3&	164&	Lambda	&&&19&	182&	acute	\cr
\+4&	165&	Xi	&&&20&	183&	caron	\cr
\+5&	166&	Pi	&&&21&	184&	breve	\cr
\+6&	167&	Sigma	&&&22&	185&	macron	\cr
\+7&	168&	Upsilon	&&&23&	186&	ring	\cr
\+8&	169&	Phi	&&&24&	187&	cedilla	\cr
\+9&	170&	Psi	&&&25&	188&	germandbls\cr
\+10&	173&	Omega	&&&26&	189&	ae	\cr
\+11&	174&	ff	&&&27&	190&	oe	\cr
\+12&	175&	fi	&&&28&	191&	oslash	\cr
\+13&	176&	fl	&&&29&	192&	AE	\cr
\+14&	177&	ffi	&&&30&	193&	OE	\cr
\+15&	178&	ffl	&&&31&	194&	Oslash	\cr
\+-&	160&	space	&&&32&	195& 	suppress \cr
\+&&&&&127&	196&	dieresis\cr

% \+16&	179&	dotlessi\cr
% \+17&	180&	dotlessj\cr
% \+18&	181&	grave	\cr
% \+19&	182&	acute	\cr
% \+20&	183&	caron	\cr
% \+21&	184&	breve	\cr
% \+22&	185&	macron	\cr
% \+23&	186&	ring	\cr
% \+24&	187&	cedilla	\cr
% \+25&	188&	germandbls\cr
% \+26&	189&	ae	\cr
% \+27&	190&	oe	\cr
% \+28&	191&	oslash	\cr
% \+29&	192&	AE	\cr
% \+30&	193&	OE	\cr
% \+31&	194&	Oslash	\cr

% \section{Colophon}

% This manual was prepared using Lugaru's Epsilon$^{\smlsize TM}$ 
% (an implementation of Richard Stallman's Emacs), 
% run through Daniel Brotsky's {\YTeX} and printed on  
% a QMS PS 815 MR printer
% from output produced by {DVI\-PS\-ONE}, reprocessed by {TWOUP}. 
% Outline fonts used were from Blue Sky Research and \Y&Y. 

\newpage

% \hbox{ } \newpage

\def\leaderfill{\leaders\hbox to 1em{\hss.\hss}\hfill}

\line{{\bf 1.\enspace Blue Sky Research Computer Modern Outline Fonts\leaderfill 1}}

\vskip .05in

\line{{\bf 2.\enspace Font installation using Adobe Type Manager (ATM)\leaderfill 1}}

\vskip .05in

\line{{\bf 3.\enspace `Automatic' versus `manual' downloading of fonts\leaderfill 2}}

\vskip .05in

\line{{\bf 4.\enspace Accents and hyphenation in non-English languages\leaderfill 3}}

\vskip .05in % \vskip .1in % \vskip .05in

\line{{\bf 5.\enspace Constraints on reencoding fonts\leaderfill 4}}

\vskip .05in % \vskip .1in % \vskip .05in

\line{{\bf 6.\enspace Printer Font ASCII from Printer Font Binary format\leaderfill 5}}


\vskip .05in % \vskip .1in % \vskip .05in

\line{{\bf 7.\enspace Changing the case of the PostScript FontName\leaderfill 6}}

\vskip .05in % \vskip .1in % \vskip .05in

\line{{\bf 8.\enspace Dealing with large numbers of fonts in MS Windows\leaderfill 6}}

\vskip .05in % \vskip .1in % \vskip .05in

\line{{\bf 9.\enspace PostScript `clone' interpreters\leaderfill 7}}

\vskip .05in % \vskip .1in % \vskip .05in

\line{{\bf 10.\enspace Using the CM outline fonts with DVIPS\leaderfill 7}}

\vskip .05in % \vskip .1in % \vskip .05in

\line{{\bf 11.\enspace Accessing special characters and font encoding\leaderfill 8}}

\vskip .05in % \vskip .1in % \vskip .05in

\line{{\bf 12.\enspace Accessing the `space' character\leaderfill 9}}

\vskip .05in % \vskip .1in % \vskip .05in

\line{{\bf 13.\enspace Accessing characters with codes greater than 127\leaderfill 9}}

\vskip .05in % \vskip .1in % \vskip .05in

\vskip .5in

$$\def\normalbaselines{\baselineskip20pt
    \lineskip3pt \lineskiplimit3pt }
  \def\mapright#1{\smash{
      \mathop{\longrightarrow}\limits^{#1}}}
  \def\mapdown#1{\Big\downarrow
    \rlap{$\vcenter{\hbox{$\scriptstyle#1$}}$}}
  \matrix{&&&&&&0\cr
    &&&&&&\mapdown{}\cr
    0&\mapright{}&{\cal O}_C&\mapright\iota&
      \cal E&\mapright\rho&\cal L&\mapright{}&0\cr
    &&\Big\Vert&&\mapdown\phi&&\mapdown\psi\cr
    0&\mapright{}&{\cal O}_C&\mapright{}&
      \pi_*{\cal O}_D&\mapright\delta&
      R^1f_*{\cal O}_V(-D)&\mapright{}&0\cr
    &&&&&&\mapdown{\theta_i\otimes\gamma^{-1}}\cr
    &&&&&&\hidewidth R^1f_*\bigl({\cal O}
      _V(-iM)\bigr)\otimes\gamma^{-1}\hidewidth\cr
    &&&&&&\mapdown{}\cr
    &&&&&&0\cr}$$
% }

\end

% difference between {\ATM} & {\PSCRIPT} ?

% change with time in ATM ?

% {\ATM} problems   {\tt atm.ini} ATMFONTS.INI

% modifications to ATM.INI

% error messages

% trouble shooting

% do a `you' clean sweep

% magnification steps

% check on references to *.txt files

% (TWOUP -vzs -m=1.25 -x=-30 -d=PRN bsrmanua)

% TWOUP -vzg -m=1.4 -c=5 -d=LPT1 bsrmanua
% TWOUP -vzh -m=1.4 -c=5 -d=LPT1 bsrmanua

% TWOUP -vzg -m=1.45 -c=5 -d=COM1 bsrmanua
% TWOUP -vzh -m=1.45 -c=5 -d=COM1 bsrmanua


%% non PS printers.

% DVIPS and HP

% DVIPS and Newgen
