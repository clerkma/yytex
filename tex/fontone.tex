% Copyright 2007 TeX Users Group.
% You may freely use, modify and/or distribute this file.
\typesize=12pt

\noheaders

\def\bbf{\bigsize\bf}

\noindent {\bbf Program:} \quad\quad FONTONE

\vskip .1in

\noindent {\bbf Function:}

\vskip .05in 

\noindent Generate Adobe Type~1 fonts from PostScript Type~3 outline
fonts in HEX format.

\vskip .1in

\goodbreak

\noindent {\bbf Inputs:}

\vskip .05in

\beginbullets

\ftpar{(a)} PostScript Type~3 outline fonts in HEX format.

\ftpar{(b)} Font metric information files in Adobe Font Metric (AFM)
format or in simplified code-width (WID) format.

\ftpar{(c)} Font- and character-level hint information files (HNT).

\endbullets

\vskip .05in

HEX format is the Type~3 font format produced by CONVERT.

If no font metric file can be found, then the metric information in
the HEX input file is used. 
It is usually preferable to use information from the AFM file since
that ensures that application programs have precisely the same font
metric information as appears in the printer font.
(Also, the HEX format does not permit specification of non-integer
character escapements in standard Adobe units).

If no hint file can be found, then only the minimally required hinting
information (BlueValues) are inserted. 

\vskip .1in

\goodbreak

\noindent {\bbf Outputs:}

\vskip .05in

\noindent The output of FONTONE is an Adobe Type~1 font file in printer font
ASCII ({\tt .pfa}) format
(The output file is ATM compatible if the command line flag `g' is used).
This output file can be processed by the program BINCOMP to produce a
font file in printer font binary ({\tt .pfb}) format.
The output of FONTONE is normally written in the current directory 
(but see command line argument `p'), and has the same file
name as the input, normally with extension {\tt .pfa}
(but see command line argument `e').

More than one font file can be specified on the command line.
Wild card arguments can be used to specify a collection of input files.
All files will be processed with the same options selected by command
line flags and command line arguments in effect.

\vskip .1in

\goodbreak

\noindent {\bbf Width File Format:}

\vskip .05in

\noindent
Font metric information can be provided using either an ordinary AFM
file, or a simplified code-width (WID) format file.
A WID format file, contains one line per
character giving the character code followed by the width 
(in standard Adobe units)---in the form 

C 32 ; WX 625 ;

\noindent
Note that this simplified font metric information format does not provide for
automatic construction of composite characters 
(see command line flag `n' below).

\vskip .1in

\goodbreak

\noindent {\bbf Controlling Behavior:}

\vskip .05in

\noindent The actions taken by FONTONE are controlled by command line
flags and command line arguments, and by environment variables.  
A command line flag is a character preceded by `-'.
Several command line flags may be concatenated following a single `-'.

A command line argument is a single character preceded by `-',
followed by a space and an argument.
The argument may be a number, word, file name or directory, as appropriate
(The actual argument may not contain embedded space).


Sample invocation of FONTONE:

\vskip .05in

\verb@fontone -vgjnmqq -w c:\at&t -h c:\at&t c:\at&t\book*.hex@

\vskip .05in

\noindent
An argument on the command line that is not preceded by an appropriate command
line argument is treated as the name of the first input file to be converted.
If {\it no} input files are specified, FONTONE gives a brief usage list
of command line flags and command line arguments.

FONTONE also looks at some environment variables that can be used to
supply information that would normally be supplied by command line arguments.
Additional control---of features that are normally fixed---is provided
by flags and variables in the program itself. 

\vskip .1in

\goodbreak

\noindent {\bbf Command Line Flags:}

\vskip .05in

\noindent Command line flags are used for modifications of some action
of FONTONE that can be specified without requiring additional arguments.

\beginbullets

\ftpar{(v)} Verbose mode. Produces more detailed information (on
{\tt stdout}) about actions taken by FONTONE, as well as some
statistical data.  A dot is output for every character outline
encountered in each of the two passes.

\ftpar{(g)} Make output ATM compatible.  Disables various time- and
space-saving `tricks' that are not guaranteed to be compatible with
ATM.  (It probably no longer makes sense {\it not} to use this flag).
Also enables automatic generation of composite characters based on
information in Composites section (CC) of AFM file.

\ftpar{(k)} Treat input as {\TeX} font.  This means, amongst other
things, that the font name will be converted to upper case, that a
CharString will be added for the {\tt space} character,
and that `control character' positions 0--31 (as well as 128--159) 
may be used for encoded, printing characters.

\ftpar{(u)} Convert font name to upper case.  This is to deal with a
problem resulting from the way font names are abbreviated on the MacIntosh.
This is mostly useful for {\TeX} Computer Modern fonts, some of which
are not unique based on just the first 5 characters of the font name.

\ftpar{(b)} Insert character definition for {\tt space}.  
Since the {\tt space} character has no corresponding outline, it does
not occur in the input HEX file. 
The width for this character is taken from the AFM file.  
This is mostly useful for {\TeX}, where the space
character is not encoded in the typical Computer Modern (CM) font.

\ftpar{(j)} Treat characters with codes in the `control character'
ranges 0--31 and 128--159 as encoded printable characters.  
These positions are not used in the Adobe StandardEncoding vector.
It is therefore normally convenient to have special characters
in these positions in the HEX file that are {\it not} encoded.
{\TeX} however needs these character positions for encoded printable
characters, so the encoding of these character positions should {\it not} 
be suppressed.

\ftpar{(n)} Construct composite characters based on the 
Composites (CC) section of the AFM file.  Note that the base
characters and the accents must be in their usual position in the
Adobe StandardEncoding vector as dictated by the Type~1 specification.
% Hence it is not possible to construct arbitrary composites this way.

\ftpar{(l)} Remap encoding in the `control character' range 0--31 to
above 160
(using the scheme employed in Lucida Math fonts, which maps
0--9 to 161--171, 10--32 to 173--195, 127 to 196, and 32 to 128).
This is useful for fonts (such as {\TeX} fonts) that have encoded,
printable characters in these `control character' positions which some
applications (such as Windows applications on the PC) cannot access.

\ftpar{(r)} Reverse character outline paths. All paths must be
traversed with `ink on the left'---use this flag if they all have ink
on the right instead---hopefully this is now no longer needed!

\ftpar{(q)} Choose copyright message.  If absent, include BSR copyright.
If present, include Y\&Y copyright.
If repeated, include AT\&T copyright.

\ftpar{(m)} Neuter font so it will work on NewGen PostScript `clone' printer.
The NewGen PostScript interpreter has numerous bugs, some of which
the code produced by FONTONE already works around. This printer is
however also unable to handle non-integer character escapements and
insists on a Subrs array.
With this command line flag, character escapements are rounded off and
a dummy Subrs array is provided.  Normally these modifications are
not a problem, since character widths are given in standard
Adobe coordinate system as integers---but for {\TeX} fonts this
introduces a small inaccuracy, since character widths in Computer
Modern are mostly rational multiples of 1/36th of a point for a 10
point font (and 10*1000/36, for example, is not an integer).

\endbullets

There are also a number of command line flags that now seem of less
importance (and that do not appear in the brief usage list that FONTONE
prints when invoked without any files specified for conversion).

\beginbullets

\ftpar{(s)} Correct for scaling problem in BSR release 0.8 fonts
(that is, multiply by 72.27 over 72). Hopefully now obsolete!

\ftpar{(f)} Suppress spike artifact work-around.  Normally FONTONE modifies
outlines to avoid an `overshoot' artifact, described below, present in
most PostScript font rendering systems 
(It is not clear why one would want to suppress this).

\ftpar{(z)} Suppress the extra four characters usually inserted at
beginning of each CharString for backward compatibility with
PostScript interpreter release 21.0 and earlier 
(It is probably not worth the savings in file and virtual memory (VM)
space to lose this compatibility, 
even though there are only a few of these beasties left). 
Also, this is a good place to hide copyright messages, date and time
(see below).

\ftpar{(t)} Operate in one pass mode.  As explained below, this means
that bounding boxes may not be computed correctly and that the side-bearing 
encoded into the CharString will be set to zero rather than the correct
value (although the character will still render correctly). 
It is not clear why one would want to do this, since the extra pass
doesn't take  {\it that} long. 

\ftpar{(d)} Do not include the downloading code at the beginning of the
font file (which is normally commented out using `\%').

\endbullets

(There are also a number of other, obsolete command line flags,
such as `a', `i', `c', `y', `x', which make it is possible to produce
output that does not contain various layers of encoding and
encrypting---while this was invaluable in early debugging, 
it is now better dealt with by taking the normal {\tt .pfa} file
produced by FONTONE and passing it once or twice through the program DECRYPT).

\vskip .1in

\goodbreak

\noindent {\bbf Command Line Arguments:}

\vskip .05in

\noindent  Command line argument provide additional information to
FONTONE, such as drive and directory names.

\beginbullets

\ftpar{(w)} Next argument specifies drive and directory where font
metric information files (AFM or WID) can be found. FONTONE looks for
a file with the same name as the input HEX format file, but with
extension {\tt .afm}.  If it cannot find such a file it tries to find
a file with the extension {\tt .wid}. 
If this fails, it used the character width information in the input
HEX format file. 

\ftpar{(h)} Next argument specifies drive and directory where hinting
information files can be found. FONTONE looks for a file with the
same name as the input HEX format file, but with extension {\tt .hnt}.
If such a file cannot be found, only the minimally required
entries (BlueValues) is included.

\ftpar{(p)} Next argument specifies drive and directory where output
is to be sent.  If omitted, output goes into current directory.

\ftpar{(e)} Next argument is extension to use for output file name.
If omitted, the extension {\tt .pfa} is used.

\endbullets

% (There are a number of other, now obsolete command line arguments.)

\vskip .1in

\goodbreak

\noindent{\bbf Environment Variables:}

\vskip .05in

\noindent Sometimes it is convenient to set up environment variables to
avoid having to type the same command line arguments each time FONTONE
is invoked:

\beginbullets

{{\tt afmpath}} - specifies drive and directory for {\tt .afm files}.

{{\tt widpath}} - specifies drive and directory for {\tt .wid files}.

{{\tt hntpath}} - specifies drive and directory for {\tt .hnt files}.

\endbullets

\vskip .1in

\goodbreak

\noindent{\bbf Flags and Variables:}

\vskip .05in

\noindent Additional control is provided by settings of variables and flags in
the program itself, listed at the beginning of the source code of FONTONE.
Typically these are variables that would only rarely be changed.
In some cases these are variables that have become `frozen', 
when settings other than the ones in the code were determined to be
less useful or unsatisfactory.
% (changing these may lead to unsatisfactory performance of FONTONE).

Two examples of preprocessor variables are 
MINCURVETO (normally 3) and MINLINETO (normally 1).  
In a CharString, {\tt curveto}'s shorter than MINCURVETO are turned into
{\tt lineto}'s, and {\tt lineto}'s shorter than MINLINETO are removed
altogether.
An example of an ordinary variable is {\tt columns} (normally 128) which
determines how long the hexadecimal lines in the output will be.

\vskip .1in

\goodbreak

\noindent{\bbf Notes:}

\vskip .05in

\bpar If possible, FONTONE rearranges outlines so that they end with a
{\tt lineto}, which can then be eliminated since {\tt closepath}
generates an implicit {\tt lineto} if the path is not actually closed.
FONTONE actually finds the {\it optimum} reordering based on the
number of encoded bytes saved by omitting the {\tt lineto}, minus
any extra cost incurred because the initial {\tt moveto} has changed.
This may result in significant savings in font file size and virtual
memory (VM) usage.

\bpar For backward compatibility with older PostScript Interpreters,
each encrypted CharString starts with four bytes that are discarded
when the CharString is decrypted.  This is a convenient place to bury
copyright strings, date and time and any other message desired 
(This may possibly be useful later in pursuing a copyright infringement
or in determining exactly which version of a font is being used).  
Presently what gets inserted in this `time capsule' is controlled by
use of the command line `q' flag.  
The program DECRYPT has provisions for listing the contents
of this `time capsule'.

\bpar For character escapements that are not integers in the standard
Adobe coordinate system, FONTONE determines a good rational
approximation for {\tt hsbw}. 
It is possible to get very high accuracy this way using only a small
number of total encoded bytes (much better than say arbitrarily always
using 1000 as the divisor).
The algorithm is based on Euclid's method for finding the greatest
common divisor.
(This features is be disabled for the NewGen printer---command line
flag `m'---since it cannot handle {\tt div} before {\tt hsbw} in CharStrings).

\bpar For {\TeX} fonts, when an AFM file is not present, FONTONE makes
use of the fact that Computer Modern font metrics obey certain regularities.
This allows it to compute {\it exact} character escapements from the rounded
quantities appearing in the HEX file (or even in {\tt .tfm} files).

\bpar FONTONE normally passes over the HEX format file twice, in part
so it can get accurate bounding box information (but see command line
flag `t').  This is needed because
the left side bearing must be known before constructing a CharString.
The first pass is also used to obtain an accurate count of characters,
needed when constructing the CharString dictionary.

\bpar While the Adobe Type~1 manual claims that coordinates in
CharStrings are restricted to plus or minus 2000, the actual
restriction on most PostScript interpreters appears to much weaker,
possible a restriction on {\it increments} in coordinates being less
than 2048. 
In most cases this is not an issue, but some {\TeX} fonts, such as 
{\tt cmex10} use coordinates that lie outside the claimed maximum range.

\bpar The PostScript interpreter only requires newlines (ASCII 10), and
treats returns (ASCII 13) merely as white space.
Unfortunately, the MacIntosh uses returns as line separators.
Consequently, if fonts are to be used on the Mac, then the outer layer
at least (the unencrypted part) should have both newlines and returns.

\bpar The PostScript interpreter on most printers translates returns
into newlines when they come in over a serial communications channel,
so lines can be terminated by either newlines or returns.  
But, this transformation is {\it not} applied to data coming over an
AppleTalk connection.
This means that the PostScript operator {\tt readline}, for example,
can easily yield a {\tt rangecheck} error when the input overflows the
allocated string before a newline is found.  
Thus it is usually {\it not} satisfactory on a MacIntosh to terminate lines
{\it only} with returns.
Fortunately, inside the encrypted section it is safe to use only
newlines, since the MacIntosh never `sees' this data, and newlines is
what the PostScript interpreter wants to see.

% combining parallel segments cancelling anti-parallel segments

\bpar FONTONE looks for repeated knots and counts places where three
knots in a row have the same coordinates.  This is usually an
indication of some error in the outline (or exessive down scaling).
The outline is not altered, but the count shows up in the statistical
information at the end and provides a warning.

\bpar FONTONE has a character string that records the current task, 
so that, should a fatal error occur, then the error message will
include a notation indicating the task FONTONE thought it was working on.

\bpar Several possible error conditions (such as inability to find
character metric information or a hint file) are considered non-fatal
and FONTONE continues after giving an error message.  It is therefore
important to watch for these error messages, since generation of an
output file and normal termination is not always an indication that
everything was done in exactly the way the user expected.

\vskip .1in

\goodbreak

\noindent{\bbf Bugs in ATM for Windows and PostScript Interpreters:}

\vskip .05in

\bpar Some PostScript interpreters that use `eroding' techniques in
character rendering are subject to an `overshoot' artifact 
(sometimes called `epaulets' since they appear similar to military
uniform shoulder decorations on the left and right upright of the upper
case letter `M').
This artifact occurs where an acute angle in an outline comes close to
the opposite side of the outline.
FONTONE detects these conditions and inserts (very) short {\tt lineto}'s
that split the angle of turning in half and thus remove the potential
artifact. 

\bpar Some PostScript interpreters are fussy about the sizes of
dictionaries and arrays.  While it seems reasonable to complain when a
data structure is too {\it small} to contain the items to be inserted,
in some cases these interpreters actually crash when the allocation is
too {\it large}.  
FONTONE goes to some trouble to make sure that the allocations
for {\tt Subrs} and {\tt CharStrings}, for example, are correct.

\bpar There are number of bugs in ATM for Windows.  FONTONE contains 
work-arounds for the ones that are known to the author at this stage.
Other work-arounds are embedded in other utility programs.
BINCOMP, for example, writes the binary part of a font as one long
binary section, since it has been determined that the CharString
lying under a splice between two binary sections is damaged and will
not work---possible even crashing the machine.  
This despite the fact that the downloader documentation from Adobe
explicitly states that multiple binary sections may follow one
another---and in fact it is customary on the Mac to use sections
limited to lengths of 2046 bytes.

\bpar Due to a bug in ATM for Windows, only the {\it first}
mentioned code number of a multiply-encoded character works. 
So it is important to have the code number that will actually be used
appear first in the listing of the encoding vector. 
FONTONE assumes that the higher numbered reference should appear first
when the command line flag `l' is used for remapping.
The other entry in the encoding vector is retained, but is not functional 
when the font is used by ATM for Windows.

\bpar Some PostScript `clone' printers, such as the NewGen, do not
completely implement the definition of Adobe Type~1 fonts and so may
not render some characters, or, more commonly, will crash.  
This despite the fact that they work just fine on fonts in the Adobe
font library. 
The reason is that the fonts in the Adobe font library follow a very simple
stylized layout, and a program tested {\it only} with such fonts is not
necessarily able to interpret arbitrary Type 1 fonts.  
FONTONE includes work-arounds for some short comings of such printers.
It turns out, for example, that some printers insist that the trailing
512 zeros in a font file be segregated into 8 lines of 64 zero.
This despite the fact that they can handle lines of 128 character
length in the hexadecimal section of the font code.

\bpar Contrary to the definition of ATM compatibility in chapter~10 of
the Adobe Type~1 booklet, some PostScript interpreters make assumptions
about how code is divided into {\it lines} (instead of just parsing
{\it tokens}). So, for example, the `ND' closing off a series of Subrs
definitions must be on a separate line---it can not just be tagged on
to the last Subr (or, in the case of an empty, dummy Subrs array,
tagged onto the line containing the call to `array').
There may be other such bugs.  
It seems safest to slavishly follow the highly stylized format that Adobe
uses in its font library---even when this implies inefficiency in font
file size of font virtual memory (VM) usage---simply because the Adobe
fonts are the fonts that new PostScript interpreters typically are tested on.

\vskip .1in

\goodbreak

\noindent{\bbf Restrictions:}

\vskip .05in

\bpar Expected order:  FONTONE expects characters in the HEX file, the
AFM file and the HNT file to appear in order of numerical code.

\bpar Limit on number of characters in a font:
Unencoded characters (other than automatically generated composites)
must appear in the input HEX format file and therefore must be assigned
a numeric code.
This presently limits the total number of characters 
(encoded plus unencoded, but not counting composites) to 256.
One way to work around this limitation, should a font ever contain more
than 256 characters (other than composites),
would be to use character codes larger than 255 for unencoded
characters in the HEX format input file.

\bpar Limit on number of unencoded characters in a font:
It has proven convenient to insert the `unencoded'
characters in the `control character' ranges 0--31 and 128--159
(which are not occupied in Adobe StandardEncoding).
This limits the number of unencoded characters to a total of 64 
(not counting automatically generated composite characters).
Normally FONTONE does not insert the corresponding names into the
encoding vector (but see command line flag `j').
One way to work around this limitation, should a font ever contain more
than 64 unencoded characters (other than composites),
would be to use character codes larger than 255 for unencoded
characters in the HEX format input file.
(Note also, that presently the `unencoded' characters must appear in the
same place in the AFM file used to provide character metric information
and in the HNT file used to provide character hinting information).

\bpar Computation of bounding boxes:
While bounding box information is
generally not used by PostScript interpreters, it should be noted
that the font bounding box computed by FONTONE assumes that the
outlines follow the suggestions in the Adobe Type~1 booklet.
That is, there are knots at all extrema in the horizontal and vertical
position, no  {\tt curveto} turns through more than 90 degrees, and
there are no knots outside the bounding box of an outline.
% That is, there are knots at all extrema and no knots lie outside the
% bounding box.  
The bounding box can then be obtained simply by taking the max and
min of knot coordinates, {\it without} actually constructing the Bezier
curves. 

% (This is probably not a serious issue since the font bounding boxes are 
% probably never used by anyone)

\bpar Lack of Subrs:
FONTONE assumes that each outline is self-contained.
There is no provision for sharing of portions of outlines by means of Subrs.
(Hooks are there for including Subrs and OtherSubrs should these prove
of importance later on). 

\bpar Flex procedure:
FONTONE detects and counts opportunities for invoking the Flex
procedure (FlxProc), but does not insert the code for FlxProc.  
To include this capability, a bit of code would have to be written and
attached to the hook provided where presently FlxProc's are merely
{\it counted}, not actually implemented.   
Perhaps more seriously, the OtherSubrs and Subrs for FlxProc would need
to be injected into the output file---and these are rather big 
(pages 78 and 91--95 of the Adobe Type~1 booklet).

\bpar Hint substitution is not provided for:
The reason for this is that avoiding hint substitutions makes possible a
clean separation of character outline and character hinting information. 
This in turns helps greatly in automating the generation of
character-level hints.
There is little to indicate that the increased complexity of hint
substitution yields significant improvement in character rendering.  
Note also that SHOWCHAR has an optimization routine that searches for
the best possible {\it fixed} hints in a situation where hint
substitution might make sense because of overlapping stems.

\bpar Lack of copyright message generality:
Presently only three different copyright messages (and corresponding 
`time capsules') can be generated automatically.

\end

