% *** *** *** *** *** *** *** *** *** *** *** *** *** *** *** *** *** *** 
% Copyright (C) 1992 Y&Y.
% Copyright 2007 TeX Users Group.
% You may freely use, modify and/or distribute this file.
% *** *** *** *** *** *** *** *** *** *** *** *** *** *** *** *** *** *** 

% NOTE: this has changed with various versions of DVIPS, can't keep up ...

% ============================================================================
% Accented and composite characters in fonts that do not use TeX text encoding
%		DVIPS `neon' version   VERSION 0.9
% ============================================================================

% Plain TeX - and hence lplain TeX - have accents hard-wired to certain codes.

% A non-CM font can be reencoded to TeX text encoding ---
% in this case accents and composite character will be where TeX expects them.
% But, quite often it is advantageous to encode a font another way.
% This can provide access to many characters not found in TeX text encoding.

% When a font is reencoded, compound characters and accents may be moved.
% This will prevent TeX's normal handling of compound characters and accents.
% This file indicates how to deal with this ---
%	--- and concludes with sample code specific for `StandardEncoding'.

% Changes required for math constructions that use roman font are at the end

% You may wish to just \input this file in your TeX source.

% --- --- --- --- --- --- --- --- --- --- --- --- --- --- --- --- --- --- --- 

% Protect against style files that make quotedbl `active'

\chardef\dblcode=\catcode`\"	% save catcode of quotedbl
\catcode`\"=12			% make quotedbl what it should be

% --- --- --- --- --- --- --- --- --- --- --- --- --- --- --- --- --- --- --- 

% NOTE: plain TeX (and LaTeX) has the accent character positions hardwired to:

% 16 for `dotlessi', 17 for `dotlessj', 
% 18 for `grave', 19 for `acute', 20 for `caron', 
% 21 for `breve', 22 for `macron', 
% 23 for `ring', 24 for `cedilla',
% 25 for `germandbls', 26 for `ae', 27 for `oe', 
% 28 for `oslash', 29 for `AE', 30 for 'OE', 31 for `Oslash',
% 94 for `circumflex', 95 for `dotaccent', 125 for `hungarumlaut',
% 126 for `tilde', 127 for `dieresis',
% (see page 356 of the TeX book for additional information)

% --- --- --- --- --- --- --- --- --- --- --- --- --- --- --- --- --- --- --- 

% The following sample redefinitions are for DVIPS `neon' encoding:

% With DVIPS `neon' encoding it is not appropriate to use `stanacce.tex',
% since most of the accents in `neon' are in the same place as they are in
% `TeX type'.  Actually, the only changes really required for `neon' 
% encoding are the following: 

\chardef\pounds=163	% sterling		(36 in TeX italic)
\chardef\L=232		% Lslash
\chardef\l=248		% lslash

\def\.#1{{\accent199 #1}}	% dotaccent	(95  in TeX text)
\def\H#1{{\accent205 #1}}	% hungarumlaut	(125 in TeX text)

\def\dot{\mathaccent"70C7 }	% dotaccent	(95  in TeX text)

% *** *** *** *** *** *** *** *** *** *** *** *** *** *** *** *** *** *** *** 

% The rest is there to make it easy to get at composite characters:

\chardef\aa=151 \chardef\AA=207 % aring, Aring
\chardef\cc=149 \chardef\CC=203 % ccedilla, Ccedilla

%  copyright		not in DVIPS `neon'

% --- --- --- --- --- --- --- --- --- --- --- --- --- --- --- --- --- --- --- 

% The following provides access to the 58 accented/composite characters.
% Some convenient abbreviations conflict with macros in plain or lplain
% (for example, \aa, \Aa, \ae, \Ae, \oe, \Oe, \sc, \Sc in plain TeX)
% So these have had to be be named something slightly less menmonic.

% --- --- --- --- --- --- --- --- --- --- --- --- --- --- --- --- --- --- --- 

% 58 `standard' accented chars exist in many fonts - including BSR CM from Y&Y
% One can define control sequences to access these directly as follows.
% One may want to define other ways for accessing these, for example,
% using ligatures (in that case, need to remake TFM file using AFMtoTFM).

\chardef\ay=153 \chardef\ee=146 \chardef\ie=142 \chardef\oy=137
\chardef\ue=131 \chardef\ye=129 % a, e, i, o, u, y - dieresis

\chardef\Ay=210 \chardef\Ee=200 \chardef\Ie=195 \chardef\Oy=190
\chardef\Ue=159 \chardef\Ye=157 % A, E, I, O, U, Y - dieresis

\chardef\ax=155 \chardef\ea=148\chardef\ia=144 \chardef\oa=139
\chardef\ua=250 \chardef\ya=253 % a, e, i, o, u, y - acute

\chardef\Ax=212 \chardef\Ea=202 \chardef\Ia=197 \chardef\Oa=192
\chardef\Ua=218 \chardef\Ya=221 % A, E, I, O, U, Y - acute

\chardef\ag=152 \chardef\eg=145 \chardef\ig=141 \chardef\og=136
\chardef\ug=130 % a, e, i, o, u - grave

\chardef\Ag=209 \chardef\Eg=198 \chardef\Ig=194 \chardef\Og=181
\chardef\Ug=158 % A, E, I, O, U - grave

\chardef\ac=154 \chardef\ee=147 \chardef\ic=143 \chardef\oc=138
\chardef\uc=132 % a, e, i, o, u - circumflex

\chardef\Ac=211 \chardef\Ee=201 \chardef\Ic=196 \chardef\Oc=191
\chardef\Uc=160 % A, E, I, O, U - circumflex

\chardef\At=204 \chardef\Nt=193 \chardef\Ot=135 % A, N, O - tilde

\chardef\at=150 \chardef\nt=140 \chardef\ot=176 % a, n, o - tilde

\chardef\sr=134 \chardef\zr=128 % scaron, zcaron

\chardef\Sr=169 \chardef\Zr=156 % Scaron, Zcaron

% Remaining four `composites': aring, Aring, ccedilla, Ccedilla defined above.

% **************************************************************************

% If you want to use < for `guillesinglleft' and > for `guillesingright'
% then uncomment the following lines:

% \catcode`\<=\active \chardef<=172
% \catcode`\>=\active \chardef>=173

% If you use < for `exclamdown', > for `questiondown', and | for `emdash'
% then uncomment the following lines:

% \catcode`\<=\active \chardef<=14
% \catcode`\>=\active \chardef>=15
% \catcode`\|=\active \chardef|=208

\chardef\lq=96 \chardef\rq=39

% Note that \lq and \rq also provide access to ` and '

\catcode`\"=\dblcode		% restore original catcode of quotedbl

% **************************************************************************

% NOTE: definitions have embedded numbers that depend on the chosen encoding
% These will need to be changed if you use an encoding other than `neon'.

% **************************************************************************

% *********************************************************************
%	Y&Y, Inc. 45 Walden Street, Concord, MA 01742 USA  (978) 371-3286
%   http://www.YandY.com  mailto:support@YandY.com
% *********************************************************************
