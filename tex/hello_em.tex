% Copyright 2007 TeX Users Group.
% You may freely use, modify and/or distribute this file.

% hello_EM.tex --- LaTeX 2e sample for European Modern

\documentclass{article}

% \usepackage[LY1]{fontenc}	% select text font encoding
% \usepackage{em}		% switch text and math fonts

\usepackage[LY1]{em}		% equivalent to the above two

\begin{document}

\hrule \vskip 2mm

``Hello World!'' --- $\alpha = \Gamma -\sqrt{x^2+y^2}$.

\vskip 2mm

{\bf Bold}. {\it Italic}. {\tt Fixed width}. {\sf Sans Serif}.

\vskip 2mm

,,German`` ``English'' << French >>  >>Swedish<<

\vskip 2mm

The difficult first flight afflicts the efflux differently.

\vskip 2mm

na\"\i ve; belov\`ed; prot\'eg\'e; r\^ole co\"ordinator; souffl\'es, 

\vskip 2mm

cr\^epes, and p\^at\'es; \AE sop's \OE uvres, \AA ngstrom, Stra\ss e.

\vskip 2mm \hrule \vskip 2mm

\noindent
{\bf Note:} make sure you have `Font $>$ Encoding' in DVIWindo set to 
TeX 'n ANSI (LY1).

\vskip 2mm

EM fonts are based on CM, but include ready-made accented characters.
This makes it possible to set up \TeX\ to hyphenate words that contain
accented characters.

Unlike CM, EM is not restricted to fixed text font encoding layout.
You can use EM with any encoding you like, although most people
use it with TeX 'n ANSI (LY1) encoding, Textures (LM1) encoding,
or Cork (T1) encoding.

{\bf Note:} if you want to use Cork/T1 encoding use 
\verb+\usepackage[T1]{fontenc}+
%
However, you may find LY1 more useful than T1 since it provides access 
to 38 more useful glyphs, obviating the need for `text companion' fonts.

You can use standard hyphenation tables set up for T1 also with LY1
since the alphabetic characters, accented characters and special
characters are in the same place.

\vskip 2mm \hrule

\end{document}

