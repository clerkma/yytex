% Sample TeX source file to access Symbol font (and Times-Roman and Courier)
% Copyright 2007 TeX Users Group.
% You may freely use, modify and/or distribute this file.

\magnification=1200	% enlarge to 120% actual size

\nopagenumbers		% don't want page numbers here

\def\\{\char92 }	% to get backslash typeset easily

% Tell TeX what TFM file contains info on font `sy' (Symbol):

\font\sy=sy

% TeX's default is to use 10 pt font size.
% To get a different font size just add `at 12pt' or whatever:

\font\bigsy=sy at 16.78pt

% Tell TeX what TFM file contains info on font `com' (Courier):

\font\com=com

% Tell TeX what TFM files contain info on fonts `tir', `tii', `tib', `tibi'
% (Use the following if you have Adobe Times-Roman fonts ---
% that is, if you bought ATM separately)

% \font\tir=tir
% \font\tib=tib
% \font\tii=tii
% \font\tibi=tibi

% Tell TeX what TFM files contain info on fonts `mtr', `mti', `mtb', `mtbi'
% (Use the following if you have Adobe TimesNewRomanPS fonts ---
% that is, if ATM came bundled with some application)

\font\tir=mtr
\font\tib=mtb
\font\tii=mti
\font\tibi=mtbi

% *** *** *** *** *** *** *** *** *** *** *** *** *** *** *** *** *** *** % 

\tir	% switch to Times-Roman

\noindent {\tib THIS IS THE FILE `{\com pool1.tex}'
(SEE ALSO {\com pool1.dvi} AND {\com pool1.ps}).}

\vskip .25in

\noindent This is a test of the capability to 
{\tii insert\/}	% this shows how to get Times-Italic
characters from the Adobe Type~1 Symbol 
{\tib font}.	% this shows how to get Times-Bold

\vskip .1in

\bigsy	% switch to big Symbol font

\noindent This text should come out all in greek.

\vskip .1in

\tir	% switch to Times-Roman again.

% *** *** *** *** *** *** *** *** *** *** *** *** *** *** *** *** *** *** %

\noindent {\tib NOTE 1:} \enspace
For this to work, the corresponding TFM files must be where
{\TeX} expects to find TFM files.  That is, copy {\com sy.tfm}, 
{\com mtr.tfm}, {\com com.tfm}, etc.
from the DVIWindo diskette TFM directory to the directory on the
hard disk that has all the other TFM files.  

PC{\TeX}, for example, would have them in {\com c:\\pctex\\textfms}.
Often the environment variable TEXTFMS will point to the directory
containing the TFM files.  
This environment variable is normally set up in the
{\com autoexec.bat} file.

\vskip .1in

\noindent {\tib NOTE 2:} \enspace
Before doing any of this, install ATM.  
This will install the outline font files for the `standard' fonts that come
with ATM, including Symbol, Courier, Helvetica (or GillSans, or ArialMT), 
and Times-Roman (or TimesNewRomanPS).
It will also add some `{\com softfonts}' entries to the WIN.INI file in your 
Windows directory.
(Actually, before installing ATM, make sure that Windows knows about the 
printers you are planning to use --- if not already done, use Windows Setup 
to `install' printer drivers if needed --- see Windows documentation).

\vskip .1in

\noindent {\tib NOTE 3:} \enspace
When printing to a PostScript printer, the default in Windows is 
{\tii not\/} to download the fonts normally resident in the printer
(such as Times-Roman, Symbol, Helvetica, Courier), 
but to simply refer to the fonts present in the printers ROM.
If you wish to {\tii force\/} downloading of an outline font file,
you can make a small change in the file WIN.INI in your
Windows directory.

Search for a line referring to `{\com sy.pfm}'.  
If this only contains a referrence to the PFM file and not the PFB file ---
as in the following: 

\vskip .03in

{\com softfont116=c:\\psfonts\\pfm\\sy.pfm}

\vskip .03in

\noindent
then Windows will use the printer resident font.  
(The same applies if there is {\tii no\/} {\com softfont} entry referring to
this font at all). 
If you want to force the Windows printer driver to download the
outline font (thus overriding what is in the printer) then change the line 
to include a referrence to the PFB file --- as follows:

\vskip .03in

{\com softfont116=c:\\psfonts\\pfm\\sy.pfm,c:\\psfonts\\sy.pfb}

\vskip .1in

\noindent {\tib NOTE 4:} \enspace
In Windows 3.1 there are TrueType fonts more or less matching
the commonly used Type~1 printer fonts
(such as TimesNewRoman, Symbol, Arial, Courier).
Most of these have names that differ from the corresponding
Adobe Type~1 fonts, hence present no problems.
Unfortunately the `Symbol' font has the same name!
To avoid confusion and possible problems when printing,
comment out a line in the file WIN.INI in your Windows directory.
In the section headed `{\com [fonts]}', add a semicolon at the beginning of
the line `{\com Symbol (TrueType)=SYMBOL.FOT}'.

\end
