% ENCOTEST.TEX
% Copyright 2007 TeX Users Group.
% You may freely use, modify and/or distribute this file.

% Test file to show off DVIWindo's on-the-fly font reencoding capabilities

% It assumes that the env var ENCODING has been set to `texnansi'
% and that the TFM files are based on that encoding also.
% The TFM files were generated using AFMtoTFM with -vadj -c=texnansi

\input texnansi

% Need some ANSI encoded font that has a full glyph complement.
% We assume here that LucidaBright fonts have been installed.

\font\lbd=lbd at 14pt
\font\lbr=lbr at 12pt
\font\lbrsmall=lbr at 10pt
\font\lbi=lbi at 10pt

\nopagenumbers

\lbr

\noindent {\lbd This is a test of on-the-fly reencoding in DVIWindo.}

\vskip .2in

\noindent {\lbd Reencoding makes many new glyphs accessible:}

\vskip .1in

% Zcaron, zcaron, Lslash, lslash
% dotlessi, caron, ring, dotaccent, hungarumlaut, breve, ogonek
% fi, fl, fraction, minus

\noindent 
\char128\ \char144\			% Lslash, lslash
\char141\ \char157\			% Zcaron, zcaron
\char131\				% florin,
\char162\ \char163\ \char164\ \char165\ % cent, sterling, currency, yen 
\char133\ \char137\ \char166\		% ellipsis, perthousand, brokenbar
\char134\ \char135\			% dagger, daggerdbl
\char139\ \char155\			% guilsinglleft, guilsinglright
\char149\	  			% bullet
\char169\ \char174\ \char153\ 		% copyright, registered, trademark
\char182\ \char167\			% paragraph, section
\char12\ \char13\			% fi fl
\char170\ \char186\			% ordfeminine, ordmasculine
\char181\ \char172\ \char177\ 		% mu, logicalnot, plusminus
\char215\ \char247\			% multiply, divide
\char145\ \char143\			% fraction, minus
\char188\ \char189\ \char190\		% onequarter, onehalf, threequarters
\char208\ \char240\ \char222\ \char254\ % Eth, eth, Thorn, thorn
etc.

\vskip .1in

\noindent {\lbrsmall Several of the glyphs shown above are 
	{\lbi not\/} in Windows ANSI encoding {\lbi or\/} in Macintosh
	standard roman encoding, hence {\lbi inaccessible\/} using
	{\lbi any\/} other Windows or Macintosh {\TeX} implementation!}

\vskip .2in

\noindent {\lbd Reencoding can change character layout to
	  what plain {\TeX} assumes.}

\vskip .1in

\noindent Special characters (26--31): 
\quad \ae\quad \oe\quad \o\quad \AE\quad \OE\quad \O\quad  

\vskip .05in

\noindent {\lbrsmall The above do occur in Windows ANSI, 
           but in code positions 230, 156, 248, 198, 140, 216}

\vskip .1in

\noindent Accents (18--24): 
\quad \`{}\quad \'{}\quad \v{}\quad \u{}\quad \={}\quad 
\r{}\quad \c{}\quad \k{}\quad

% grave, acute, caron, breve, macron, ring, cedilla

\vskip .05in

\noindent {\lbrsmall Some of these are in Windows ANSI, 
	but in code positions 96, 180, ... 175, 184.}

\noindent {\lbrsmall Note that caron, breve, and ring 
	are {\lbi not\/} in Windows ANSI, yet accessible in DVIWindo!}

\vskip .1in

\noindent Accents (94--95, 125--127): 
\quad \^{}\quad \.{}\quad \H{}\quad \~{}\quad \"{}\quad 

% circumflex, dotaccent, hungarumlaut, tilde, dieresis

\vskip .05in

\noindent {\lbrsmall Some of these are in Windows ANSI, 
	but at code positions 136, ..., 152, 168.}

\noindent {\lbrsmall dotaccent, hungarumlaut, and ogonek are 
	{\lbi not\/} in Windows ANSI, yet accessible in DVIWindo!}

\vskip .1in

\noindent Special characters (16, 17, 25) \quad \i\quad \j\quad \ss\quad 

\vskip .05in

\noindent {\lbrsmall germandbls may be found in Windows ANSI encoding, but 
	{\lbi not\/} dotlessi and dotlessj.}

\vskip .2in

\noindent {\lbd Reencoding can make unencoded characters accessible:}

\vskip .1in

\noindent {\lbrsmall Many of the glyphs shown above are not in Windows ANSI.}

\vskip .1in

\noindent f-ligatures: 
\quad fi, \quad fl, \quad ff, \quad ffi, \quad ffl.

\vskip .05in

\noindent {\lbrsmall Note that Windows ANSI includes {\lbi no\/} ligatures at
	all, not even fi and fl!  
	And Macintosh standard roman encoding {\lbi only\/} has
	the fi and fl ligatures...} 

\vskip .2in

\noindent {\lbd Reencoding provides access to ready-made accented characters:}

\vskip .1in

\noindent
\`a\quad \'a\quad \"a\quad \^a\quad \~a\quad \char23a\quad
\c c\quad\`e\quad \'e\quad \"e\quad \^e\quad 
\`i\quad \'i\quad \"i\quad \^i\quad 
% \`\i\quad \'\i\quad \"\i\quad \^\i\quad 
\~n\quad

\vskip .1in

\noindent
\`o\quad \'o\quad \"o\quad \^o\quad \~o\quad
\`u\quad \'u\quad \"u\quad \^u\quad 
\v s\quad\'y\quad\"y\quad\v z

\vskip .1in

\noindent
\`A\quad \'A\quad \"A\quad \^A\quad \~A\quad \char23A\quad
\c C\quad\`E\quad \'E\quad \"E\quad \^E\quad 
\`I\quad \'I\quad \"I\quad \^I\quad 
\~N\quad

\vskip .1in

\noindent
\`O\quad \'O\quad \"O\quad \^O\quad \~O\quad
\`U\quad \'U\quad \"U\quad \^U\quad 
\v S\quad\'Y\quad\"Y\quad\v Z

\vskip .1in

\noindent {\lbrsmall Note that the above can be conveniently accessed using 
	`pseudo ligatures'.}

\vskip .2in

\noindent {\lbd {\TeX}'s pseudo ligatures still work correctly:}

\vskip .1in

\noindent Standard pseudo ligs: 
endash --\quad emdash ---\quad
exclamdown !`\quad questiondown ?`

\vskip .1in

\noindent Standard pseudo ligs: 
quotedblleft ``\quad quotedblright '' 

\vskip .1in

\noindent Extra pseudo ligs: 
quotedblbase ,,\quad guillemotleft << \quad guillemotright >>

\vskip .1in

\noindent {\lbrsmall There is a lot of flexibility in where these are placed in
	the encoding, since they are {\lbi only\/} accessed via pseudo
	ligatures.} 

\end
