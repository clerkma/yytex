% Copyright 2007 TeX Users Group.
% You may freely use, modify and/or distribute this file.
\typesize=12pt

\noheaders	% \nopagenumbers

\vsize=5in	% Better format for on screen

\input lcdplain	% Switch to Lucida Bright + Lucida New Math

\input texnansi	% text fonts are encoded to TeX 'n ANSI

\font\logo=logo10	% for typesetting the word METAFONT

\font\smallrm=lbr at 7pt	% for & in Y&Y

\font\sc=lbrsc at 10.5pt	% smallcaps for Acronyms

\font\calsy=cmsy10		% calligraphic for AMS logo

\font\lbma=lbma at 9.5pt

\font\cmr=cmr10			% to show how `thin' CM is

\font\bfhead=lbd at 16pt	% main heading font

\font\tir=tir at 11.5pt
\font\tii=tii at 11.5pt

\font\lstnr=lstnr

\def\MathTime{{\tii MathT\kern-1.5pt\i me}}

\let\menu=\it			% how to designate menu items

% following from Donald Story's dvi_help.sty

\def\green{0 0.5 0}
\def\blue{0 0 1}
\def\colorpush#1{\special{color push rgb #1}}
\def\colorpop{\special{color pop}}
\def\hb#1{\colorpush\blue{#1}\colorpop}
\def\hg#1{\colorpush\green{#1}\colorpop}

%
% Macros for hypertext - buttons and marks
%
\newcount\height
\newcount\width
%
\def\ref#1#2{\leavevmode
   \setbox0=\hbox{\hg{#1}}\height=\ht0\width=\wd0%
%   \special{button: \the\width\space \the\height\space "<#2>"}%
   \special{button: \the\width\space \the\height\space <#2>}%
   \unhbox0}
% \def\xref#1{\leavevmode\special{mark: \space "<#1>"}}
\def\xref#1{\leavevmode\special{mark: \space <#1>}}
\def\launch#1#2{\leavevmode
   \setbox0=\hbox{\hg{#1}}\height=\ht0\width=\wd0%
   \special{button: \the\width\space \the\height\space launch: #2}%
   \unhbox0}
\def\http#1{\leavevmode
   \setbox0=\hbox{\hg{#1}}\height=\ht0\width=\wd0%
   \special{button: \the\width\space \the\height\space #1}%
   \unhbox0}
\def\httpx#1#2{\leavevmode
   \setbox0=\hbox{\hg{#1}}\height=\ht0\width=\wd0%
   \special{button: \the\width\space \the\height\space #2}%
   \unhbox0}

\def\bfsection#1{\noindent\xref{#1}\noindent{\bf\hb{#1}}}
\def\bfsectionx#1#2{\noindent\xref{#1}\noindent{\bf\hb{#2}}}

\def\bfindex#1{\bpar\ref{#1}{#1}}
\def\bfindexx#1#2{\bpar\ref{#2}{#1}}

\def\Y&Y{Y\kern-1.5pt{\smallrm\&}\kern-1.5ptY}

\def\LaTeX{{\rm L\kern-.36em\raise.3ex\hbox{\sc a}\kern-.15em%
    T\kern-.1667em\lower.7ex\hbox{E}\kern-.125emX}}
 
\def\SLiTeX{{\rm S\kern-.06em{\sc l\kern-.035emi}\kern-.06emT\kern
   -.1667em\lower.7ex\hbox{E}\kern-.125emX}}

\def\LaTeXe{{\LaTeX}~$2_{\textstyle\varepsilon}$}

\def\METAFONT{{\logo META}\-{\logo FONT}}
\def\AMS{{\calsy AMS}}
\def\ATM{{\sc atm}}
\def\PDF{{\sc pdf}}
\def\URL{{\sc url}}
\def\PK{{\sc pk}}
\def\DVI{{\sc dvi}}
\def\TFM{{\sc tfm}}
\def\TIFF{{\sc tiff}}
\def\PS{{\sc ps}}
\def\CTAN{{\sc ctan}}
\def\DVIWindo{{\sc dviw}indo}
\def\DVIPSONE{{\sc dvipsone}}
\def\EMERGE{{\sc emerge}}

% add Keywords to PDF DocInfo
\special{PDF: Keywords TeX, PDF, Acrobat, Mathematics, Computer Modern}%

% set PDF Title
\special{PDF: Title Acrobat PDF from TeX}%

% set PDF Subject
\special{PDF: Subject How to make quality Acrobat PDF files}%

% set PDF Author
\special{PDF: Author Louis Vosloo  SN\# 6753  1996 Mar 04 12:16:19}

% set PDF CropBox
\special{PDF: BBox 56 140 555 650}%

\centerline{{\bfhead 
\special{color push rgb 0.6 0.1 0}%
Acrobat PDF from {\TeX}
\special{color pop}%
}}

\vskip 0.7in

\chardef\bs=92
\chardef\lb=123
\chardef\rb=125

\def\menudown{{\lbma\char44}}

\xref{logo}%

\hskip1.75in\special{insertimage: y&ylogo.tif 2500000}%
\hskip .5in \registered

\vskip .1in

\centerline{\httpx{{\Y&Y}, Inc.}{}

\vskip .10in

\bfsectionx{From TeX to Acrobat}{From {\TeX} to Acrobat}

\vskip .05in

\noindent 
{\TeX} --- the typesetting language invented by Donald E.~Knuth at
Stanford --- is widely used in the academic and research community for
typesetting technical articles and books. {\TeX} is unsurpassed in its
capabilities for typesetting complex math 
(\ref{see sample at end of this article}{sample})%
\footnote{${}^\dagger$}{Click on text in green to make hypertext jump.}
and is well known for its
superb dynamic programming hyphenation/line-breaking search algorithm.
%
Adobe Acrobat {\PDF} is a powerful platform-independent technology 
for document interchange.  It is natural to want to 
convert material typeset using {\TeX} to Acrobat {\PDF} format.

It is easy to do this using a {\TeX} system designed from the ground
up for scalable outline fonts such as {\ATM} compatible fonts in 
Adobe Type~1 format (a.k.a. `Post\-Script fonts' or `{\ATM} fonts').  
With the {\Y&Y} {\TeX} system,
for example, you simply typeset the material using
{\TeX}, check the result using the {\DVIWindo} previewer, then print 
to file using the special {\PS} printer driver {\DVIPSONE}
(which can be called % like any other printer driver 
directly from the previewer).
Finally, Acrobat Distiller is used to process the resulting PS
% Post\-Script 
file.

\vskip .1in

% \centerline{And that is pretty much all there is to it!}
%  --- aside from font licensing issues, that is.
\bfsection{And that is pretty much all there is to it!}

\vskip .10in

% \noindent {\bf Bitmapped Font Woes}
\bfsection{Bitmapped Font Woes}

\vskip .05in

\noindent
So what {\it is\/} the problem with Acrobat {\PDF} files made from {\TeX} output?

{\TeX} has traditionally been used with % the so-called 
`Computer Modern' fonts --- also designed by Donald E.~Knuth --- 
in bitmapped form ({\PK} font file format).
Converting Post\-Script files containing bitmapped fonts to {\PDF}
format leads to unattractive (and slow) rendering in the Acrobat Reader,
mostly because it is impossible to scale bitmapped fonts well.

The solution to this problem is to use scalable outline fonts in 
{\ATM} compatible Adobe Type~1 format.  
The Computer Moderns fonts {\it are\/} available in Type~1 format --- 
in both non-commercial form 
(on {\CTAN} --- the `Comprehensive {\TeX} Archive Network')
and commercial form (from {\Y&Y}, Inc. or Blue Sky Research).  
For details 
\ref{see end of this note}{contacts}
\ (note that both non-commercial and commercial forms 
have some restrictions on use).

\vskip .10in

% \noindent {\bf Font Choices}
\bfsection{Font Choices}

\vskip .05in

\noindent Of course, once one uses a {\TeX} system that properly
supports scalable outline fonts, there is no longer any neccessity to
stick with Computer Modern.  Literally {\it thousands\/} of text
fonts in Adobe Type~1 format can be used.  
When it comes to math fonts, however, the choices are much more limited,
because of % very 
special demands made by {\TeX} on math fonts.  
There are presently the following possibilities:

\vskip .05in

(1) Computer Modern (and extra {\LaTeX}, {\SLiTeX} and {\AMS} fonts);

\vskip .05in

(2) Lucida{\registered} Bright + Lucida New Math (and Lucida Bright Expert);

\vskip .05in

(3) {\MathTime} 1.1 (and {\MathTime} Plus) + Times, Helvetica, \&\ Courier;
% (plus perhaps Adobe Math Pi);

\vskip .05in

(4) Adobe Lucida, Lucida Sans, and Lucida Math.

\vskip .05in

\noindent
The first three font sets are available in Type~1 format from 
\httpx{{\Y&Y}, Inc.}% {http://www.YandY.com/yyfonts.htm},
{http://www.YandY.com/products.htm}, % #fonts
the last one from 
\httpx{Adobe Systems Inc.}{http://www.Adobe.com/type/main.html}
% depends on fonts from the Adobe Type library.
\ {\TeX} macro packages are available to make it easy to switch to these
fonts in  (i) plain {\TeX}, (ii) {\LaTeX} 2.09, and 
(iii) {\LaTeXe} dialects of {\TeX}.

\vskip .1in

\ipar
{\cmr Some find Computer Modern a bit `thin' and prefer a 
somewhat `heavier' face for text, such as Lucida Bright or Times-Roman.}

% \vskip .1in

% \ipar
% {\times Some find Times-Roman a bit too compact
% and prefer an face with a large x-height such as Lucida Bright.}

\vskip .1in

\noindent
The generation of Acrobat {\PDF} files is particularly simple when
using a {\TeX} system designed from the ground up for use with
scalable outline fonts.  Such a system will produce
Post\-Script files referring directly to the Type~1 fonts.  
This {\PS} file can then be passed through the Acrobat Distiller.
In this case, little preparation is needed, other than to first
install the fonts using Adobe Type Manager 
% ({\ATM}).
\httpx{({\ATM})}{http://www.Adobe.com/prodindex/atm/main.html},

Other {\TeX} systems require more work, including 
(i) multiple metric files, 
(ii) special configuration files that tell where the font files are,
(iii) information on how to reencode the fonts, and 
(iv) what the relationship is between the name used
for a font in {\TeX}, the Post\-Script Fontname, and the name of the
font file(s).  
But this is just what it takes for such systems to use fonts in 
Type~1 format in the first place case when making Post\-Script files.  
There isn't anything specific to preparing for Acrobat Distiller.

\vskip .10in

% \noindent {\bf Some Practical Details}
\bfsection{Some Practical Details}

\vskip .05in

\noindent 
There are other issues that need to be addressed when creating {\PDF} files:

\vskip .05in

\ftpar{(1)}
Distiller should be set up to {\it always\/} subset fonts (i.e. use
`partial font downloading').  This not only reduces the size of the 
{\PDF} file, but is {\it required\/} by some font licensing agreements.  
%
In Distiller~3.0 or later,
use `Distiller \menudown\ Job Options' and click on the `Font Embedding' tab.
Check both `Embed All Fonts' and `Subset fonts below.'
Change the percentage to 99\% (instead of the default 35\%).
% You need to add a file to Distiller's `{\lstnr startup}' 
% directory/folder containing the line:

\vskip .05in

\noindent
\ipar
In older versions of Distiller the same effect can be achieved
by adding a file to the `{\lstnr startup}' directory containing the line

{\lstnr {<}{<}/SubsetFonts true /MaxSubsetPct 99 {>}{>} setdistillerparams}

\vskip .05in

\ftpar{(2)}
Check that Distiller has embedded all fonts, {\it and\/} that all 
embedded fonts are subsetted (partial).  
% To do this, read the resulting {\PDF} file into Acrobat Reader.  
In Acrobat Reader~3.0, open the {\PDF} file, and page through 
the file in order to `touch' all fonts.
Then select `File\menudown Document Info\menudown Fonts.'
Click on `List All Fonts'  and look for `Embedded Subset' 
in the `Used Font' column.

\ipar
In Acrobat Reader~2.1 % open the {\PDF} file, 
% select `File\menudown Document Info\menudown Fonts.'
% From the `{\menu File}' menu select  `{\menu Document Info}.'
% Then select `{\menu Fonts}' and click on `{\menu List All Fonts}'. 
% Click on `List All Fonts.'
% All partial fonts have `Original Names' that have a six letter prefix 
the `Original Names' of all subsetted fonts have a six letter prefix 
followed by  `{\lstnr +}.'  
Fonts listed {\it without\/} such a prefix are (i) not embedded at all, 
(ii) simulated using generic Multiple Master fonts,
or (iii) are embedded as complete fonts.
Generally this is not what you want.

\vskip .05in

% \ipar
% (Do not be confused by a `Built In' listing in the `Encoding' column.
% This refers to the {\it encoding\/}, i.e. character layout, {\it not\/} 
% the font itself.)

% \vskip .05in

\ipar
% The `Used Font' column is important.  It should say `Embedded' if the
% font is embedded.  
By the way, if the font is installed on the system on which
you are reading the PDF file, then the `Used Font' column
may instead list the font name,
indicating that the reader is using the locally installed font
rather then the embedded one.
This can lead to problems if the local version can not be
reencoded, for example.
% By the way, this field will be blank if you have not yet looked 
By the way, the `Used Font' field will only be filled in 
if you have looked at a page in the document that uses this font.
% If it is blank, page through the file first.
% to `touch' all fonts. % referenced.

% \vskip .05in
% \ipar
% In Acrobat Reader 3.0, the 6 letter prefix is not shown,
% instead check the `Used Font' column for `Embedded Subset.'

\vskip .05in

\ipar
If you notice that {\it complete\/} fonts are embedded,
then do not have Distiller's job options set up correctly.
% the new file in
% the `{\lstnr startup}' directory set up properly.  
If fonts are not embedded at all, then the Acrobat Reader will try 
to simulate them using the generic Multiple Master fonts.  
In the case of output from {\TeX} this is almost {\it never\/} what you want. 
You have to go back to the Distiller and tell it to always embed 
fonts.  Then run the {\PS} file through Distiller again.

% \ipar
% Forcing the Distiller to always embed fonts may seem counter
% to the idea of preventing complete fonts from being included,
% but the idea is to include all fonts to ensure quality rendering,
% yet prevent inclusion of complete fonts.

\vskip .05in

\ftpar{(3)}
There may be difficulties when using fonts that use the `control
character' range (character code 0--31).  This can be avoided by either
using normal text fonts that do not use the control character range,
or --- when this is not possible (as with Computer Modern fonts) ---
use a printer driver that remaps such character codes to higher up
(the encoding vectors for the CM fonts in Type~1 format repeat the 
0--31 range to 161--170 and 172--195).
% The same applies to fonts that do not have a `{\lstnr space}' character in 
% character code position 32.

\vskip .05in

\ftpar{(4)}
To exploit fonts other than Computer Modern, use a {\TeX} system that
can {\it fully reencode\/} the fonts to provide access to all characters.
Otherwise you may have problems getting 
at some glyphs.
% at glyphs that are not available on some platforms.  

\vskip .05in

\ipar
For example, out of 228 `standard' characters found in
most text fonts, 21 are not normally accessible on the Macintosh.
In Windows, 15 are missing, including important glyphs such as 
% `{\lstnr dotaccent}' (\.{}), `{\lstnr dotlessi}' (\i), 
% `{\lstnr fi}' (fi), and `{\lstnr fl}' (fl).
`\i' ({\lstnr dotlessi}), `\.{}' ({\lstnr dotaccent}), 
and the ligatures `fi' ({\lstnr fi}) 
and `fl' ({\lstnr fl}).\footnote{${}^\dagger$}
{Ideally the system should also provide access to `ff' `ffi' `ffl' `\j.'}
% and so on.}


% and ff, ffi, ffl if the font has them

\vskip .05in

\ftpar{(5)}
Different printer drivers produce very different Post\-Script code.
Post\-Script code that works fine on a printer or image setter 
% may % typically
often is 
{\it not\/} set up to work optimally with Acrobat.
Check, for example, that equal thickness rules render equally thick
in the Acrobat Reader, independent of where on the page they lie.
This is not the case with some popular PS drivers.

\vskip .05in

\ftpar{(6)}
Since Acrobat {\PDF} format provides strong hypertext support,
you may want to use a {\TeX} system that itself has hypertext support,
and that {\it also\/} translates % its own 
hypertext links to appropriate
Acrobat `{\lstnr pdfmarks}' embedded in the Post\-Script file
(see \ref{Contact Information)}{Contact Information}).

\vskip .05in

\ftpar{(7)} Before posting PDF files 
with embedded fonts on the InterNet,
% on WWW or for access via anonymous FTP, 
check with the font vendor whether any special licensing is required.
% this is permitted by the end user license.

\vskip .10in

% \noindent {\bf Legacy Documents}
\bfsection{Legacy Documents}

\vskip .05in

\noindent 
The above assumes that the {\TeX} source files are available.
% (or at least the {\DVI} files produced by {\TeX}).
% In some cases 
Often all that is left are Post\-Script files made some years ago.
These typically will contain bitmapped fonts that lead to
exactly the rendering problems described above.
It is non-trivial to `reverse engineer' one of these files,
since there is no information in the file on what the fonts 
are called, and since the bitmaps for a given font are {\it not\/} 
fixed entities, 
depending on the printer resolution and the {\METAFONT} 
`mode\char95def' 
settings used when creating them.
% However, work on a program to automate replacement of bitmapped
% fonts in such `legacy documents' is under way.
Look for `Converting Legacy PostScript Files' and 
`{\lstnr FixFont}' on the 
\httpx{emerge}
{http://emerge.pdfzone.com/resources/texpdf.html} 
% \noindent 
\ web site 
for an attempt at a solution.

\vskip .10in

% \noindent {\bf Sample PDF Documents}
\bfsection{Sample PDF Documents}

\vskip .05in

\noindent 
Some samples typeset using {\TeX} and converted
to Acrobat {\PDF} format may be found at
% {\lstnr http://www.YandY.com}.
% {\lstnr \http{http://www.YandY.com/yyfonts.htm}}.
{\lstnr \http{http://www.YandY.com/products.htm\#fonts}}, % #fonts
% add specific page to this ?
There are {\PDF} files illustrating the 
first three font choices 
% (CM, LB+LNM, and \MathTime)
% (\httpx{CM}{http://www.YandY.com/chironcm.pdf},
% \httpx{LB+LNM}{http://www.YandY.com/chironlb.pdf}, and
% \httpx{\MathTime\/}{http://www.YandY.com/chironmt.pdf})
% add links here direct to PDF ?
discussed above:
% Look under % `What's New,' 
% `\httpx{What's New}{http://www.YandY.com/yynews.htm},' 
% add links here direct to page ?
% or go to the section discussing fonts.

\vskip .2in  % \vskip .25in 

\hskip .6in % \hskip .8in % \hskip 1in %\quad\quad % \noindent
%\special{button: 1191116 1318735 http://www.yandy.com/chironcm.pdf}%
\special{button: 1191116 1318735 http://www.yandy.com/download/chironcm.pdf}%
% \special{insertimage: pdficonl.tif 1191116}% 28 x 31
\special{insertimage: pdficonl.tif 1000000}% 28 x 31
\hskip .3in
\raise.05in
% \hbox{(CM),}
% \hbox{\httpx{CM}{http://www.YandY.com/chironcm.pdf},}
\hbox{\httpx{CM}{http://www.YandY.com/download/chironcm.pdf},}
\hskip .5in % \hskip .25in %\quad
% \special{button: 1191116 1318735 http://www.yandy.com/chironlb.pdf}%
\special{button: 1191116 1318735 http://www.yandy.com/download/chironlb.pdf}%
% \special{insertimage: pdficonl.tif 1191116}% 28 x 31
\special{insertimage: pdficonl.tif 1000000}% 28 x 31
\hskip .3in
\raise.05in
% \hbox{(LB+LNM),}
% \hbox{\httpx{LB+LNM}{http://www.YandY.com/chironlb.pdf},}
\hbox{\httpx{LB+LNM}{http://www.YandY.com/download/chironlb.pdf},}
\hskip .2in % \quad 
% \special{button: 1191116 1318735 http://www.yandy.com/chironmt.pdf}%
\special{button: 1191116 1318735 http://www.yandy.com/download/chironmt.pdf}%
% \special{insertimage: pdficonl.tif 1191116}% 28 x 31
\special{insertimage: pdficonl.tif 1000000}% 28 x 31
\hskip .3in
\raise.05in
% \hbox{(\MathTime\/).}
% \hbox{\httpx{\MathTime\/}{http://www.YandY.com/chironmt.pdf}.}
\hbox{\httpx{\MathTime\/}{http://www.YandY.com/download/chironmt.pdf}.}

% (\httpx{CM}{http://www.YandY.com/chironcm.pdf},
% \httpx{LB+LNM}{http://www.YandY.com/chironlb.pdf}, and
% \httpx{\MathTime\/}{http://www.YandY.com/chironmt.pdf})

\vskip .10in

% \noindent {\bf Additional Information}
\bfsection{Additional Information}

\vskip .05in

\noindent
For additional information on % setting 
Distiller's parameters, check the Distiller on-line help file; 
% as well as
and page~127 in ``{\it Acrobat~2.1: Your Personal Consultant}'' 
by Roy Christmann,
Ziff-Davis Press, Emeryville, California, 1995, ISBN 1-56276-336-9.
%
% You may also wish to consult
Also see page 15 in Adobe's Technical Note 5151
``{\it Acrobat Distiller Parameters}'' 
available from
% \httpx{{\lstnr ftp.adobe.com}}{ftp://ftp.adobe.com}
% \ in % directory
% {\lstnr /pub/adobe/Documents/PDFs}
% as file

% \httpx{\lstnr 5151.distparams.pdf}
% {ftp://ftp.adobe.com/pub/adobe/Documents/PDFs/5151/distparams.pdf}
{\lstnr
\http{http://www.adobe.com/supportservice/devrelations/devtechnotes.html}}.
% {http://www.adobe.com/supportservice/devrelations/devtechnotes.html}
% ``{\it Acrobat Distiller Parameters}.''
% Adobe's Technical Note 5151.

% \vskip .10in

\vfill\eject

\noindent
\xref{contacts}%
% \noindent {\bf Contact Information}
\bfsection{Contact Information}

\vskip .05in

\noindent 
{\CTAN} is the `Comprehensive {\TeX} Archive Network.'
The main servers are 
% ftp.shsu.edu
% {\lstnr \http{ftp://ftp.shsu.edu}} % currently outdated ???
% {\lstnr \http{ftp://tug2.cs.umb.edu}}
{\lstnr \http{ftp://ftp.ctan.org}} %% ??
(USA),
% {\lstnr ftp.dante.de} 
{\lstnr \http{ftp://ftp.dante.de}}
\ (Germany), % and
% {\lstnr ftp.tex.ac.uk} 
{\lstnr \http{ftp://ftp.tex.ac.uk}}
\ (United Kingdom).
Look in `{\lstnr tex-archive}.'

Blue Sky Research, makers of `Textures' --- 
an integrated {\TeX} system for the Macintosh --- may be found at 

\vskip .05in

% {\lstnr http://www.bluesky.com}.
{\lstnr \http{http://www.BlueSky.com}}.

\vskip .05in

\noindent
{\Y&Y}, Inc. makers of the {\Y&Y} {\TeX} system for Windows ---
% also specializes in {\ATM} compatible Type~1 fonts for use with {\TeX} ---
may be be found at 

\vskip .05in

% {\lstnr http://www.YandY.com}.
{\lstnr \http{http://www.YandY.com}}.

\vskip .05in
\noindent
{\Y&Y} also specializes in {\ATM} compatible Type~1 fonts for use with {\TeX}.

% \vskip .10in

% \noindent {\bf Try this without using {\TeX}!}

% \vskip .05in

% \vskip .25in
% \vskip .15in
% \vskip .1in
% \vskip .05in

\xref{sample}%

%%%%%%%%%%%%%%%%%%%%%%%%%%%%%%%%%%%% diagram

$$\def\normalbaselines{\baselineskip20pt
    \lineskip3pt \lineskiplimit3pt }
  \def\mapright#1{\smash{
      \mathop{\longrightarrow}\limits^{#1}}}
  \def\mapdown#1{\Big\downarrow
    \rlap{$\vcenter{\hbox{$\scriptstyle#1$}}$}}
  \matrix{&&&&&&0\cr
    &&&&&&\mapdown{}\cr
    0&\mapright{}&{\cal O}_C&\mapright\iota&
      \cal E&\mapright\rho&\cal L&\mapright{}&0\cr
    &&\Big\Vert&&\mapdown\phi&&\mapdown\psi\cr
    0&\mapright{}&{\cal O}_C&\mapright{}&
      \pi_*{\cal O}_D&\mapright\delta&
      R^1f_*{\cal O}_V(-D)&\mapright{}&0\cr
    &&&&&&\mapdown{\theta_i\otimes\gamma^{-1}}\cr
    &&&&&&\hidewidth R^1f_*\bigl({\cal O}
      _V(-iM)\bigr)\otimes\gamma^{-1}\hidewidth\cr
    &&&&&&\mapdown{}\cr
    &&&&&&0\cr}$$
% }

% \vskip .10in
% \vskip .05in

% \noindent {\bf Try that without using {\TeX}!}
% \bfsectionx{Try that without TeX}{Try that without {\TeX}!}
% \bfsectionx{Typeset the above without using TeX}%
% {Typeset the above without using {\TeX}!}
% \bfsectionx{Try typesetting the above without using TeX}%
% {Try typesetting the above without using {\TeX}!}
\bfsection{Try typesetting the above without using {\TeX}!}%

\vskip .10in

% \vfill\eject

\bfsection{Colophon}

\vskip .03in

% \noindent
\bpar
This document was typeset using the {\Y&Y} {\TeX} System 
release 2.0 --- % release 1.2 ---
translated to Post\-Script form by {\DVIPSONE} ---
then to {\PDF} form by % Acrobat Distiller~2.1 for Windows.
Acrobat Distiller~3.0.2 for Windows.

\vskip .03in

\bpar
DocInfo {\lstnr pdfmark} 
(check out `File\menudown Document Info\menudown General'
in Acrobat Reader)
produced automatically by {\DVIPSONE}.
Keywords (in DocInfo {\lstnr pdfmark}) 
supplied via {\lstnr \bs special\lb ...\rb}.

\vskip .03in

\bpar
CropBox {\lstnr pdfmark} (which defines the area shown in Acrobat Reader) 
supplied via {\lstnr \bs special\lb ...\rb}.

\vskip .03in

\bpar
Hypertext links to `named destinations' 
automatically translated by {\DVIPSONE} to appropriate {\lstnr pdfmarks}.
Similarly, {\DVIPSONE} translated hypertext links to {\URL}s 
to standard {\lstnr /URI /Action} annotations 
(these require the 
% Acrobat Web\-Link plug-in 
(\httpx{Acrobat Web\-Link plug-in}{http://www.Adobe.com/acrobat/plugins.html})
for proper operation).

\vskip .03in

\bpar
{\TIFF} image (the 
\ref{{\Y&Y} logo}{logo})
% {\Y&Y} logo) 
inserted via {\lstnr \bs special\lb insertimage ...\rb}.
Colored text produced via {\lstnr \bs special\lb color rgb ...\rb}.

\vskip .03in

\bpar
Fonts used include several from the Lucida Bright + Lucida New Math
super family, as well as a few {\cmr Computer Modern} fonts
and {\tii Times-Italic}
(check out `File\menudown Document Info\menudown Fonts'
in Acrobat Reader).

\vskip .10in

\bfsection{Trademark Information}

\vskip .03in

% \bpar
\noindent
{\TeX} is a registered trademark of % the American Mathematical Society
% ({\AMS}).
% \httpx{({\AMS\/})}{http://www.AMS.org}
the \httpx{{\AMS\/}}{http://www.AMS.org}\ 
(American Mathematical Society).
Lucida is a registered trademark of Bigelow \& Holmes, Inc.
\MathTime\ is a trademark of Publish or Perish, Inc.
{\Y&Y} and the % {\Y&Y} 
\ref{{\Y&Y} logo}{logo}
\ are registered trademarks of {\Y&Y}, Inc.
{\DVIPSONE} and {\DVIWindo} are trademarks of {\Y&Y}, Inc.
Post\-Script, Acrobat, and Adobe Type Manager are registered
trademarks of 
% Adobe Systems, Inc.
\httpx{Adobe Systems, Inc.}{http://www.Adobe.com},
Windows is a registered trademark of 
% Microsoft, Inc.
\httpx{Microsoft, Inc.}{http://www.Microsoft.com} % .

\vfill\eject

\bfsection{Index}
\footnote{${}^\dagger$}{Click on text in green to make hypertext jump.}

\vskip .15in

\bfindexx{From TeX to Acrobat}{From {\TeX} to Acrobat}

\vskip .05in

\bfindex{Bitmapped Font Woes}

\vskip .05in

\bfindex{Font Choices}

\vskip .05in

\bfindex{Some Practical Details}

\vskip .05in

\bfindex{Legacy Documents}

\vskip .05in

\bfindex{Sample PDF Documents}

\vskip .05in

\bfindex{Additional Information}

\vskip .05in

\bfindex{Contact Information}

% \vskip .05in

% \bfindex{Trademark Information}

\vskip .05in

\bfindex{Colophon}

\vskip .05in

\bfindex{Trademark Information}

\end

use TeX system set up from the ground up for scalable outline fonts.

partial fonts: need for, setting up, checking

control character range, what to do

character encoding

equal thickness rules

hyper-text

trademarks

PSNFSS

PREPDF

http://www.adobe.com/supportservice/custsupport/SOLUTIONS/2d7a.htm

add Y&Y logo

add complex math (commutative diagram) at end?

add reference to self, such as look in Doc Info

enable http://www... links

add link to http://www.adobe.com

add links internal (e.g. to math example)

Keywords/CropBox do internal without need to edit?

Font Licensing Issues

Colored fonts for headings

URL's implement pdfmark

`View' menu `Fit Page'

No commas in Keywords?

Split up that long section: Some Practical Details

About this document (colophon)
------------------------------

DocInfo provided automatically by DVIPSONE

Keywords supplied via \special

CropBox supplied vie \special

Color provided by \special

TIFF image provided by \special

Fonts used are Lucida Bright + Lucida New Math

Hypertext links for DVIWindo automaticaly translated by DVIPSONE

URL

launch application

add table of contents with hyper-text links

need to have Acrobat WebLink plug-in for URI link (or is it URL)

% trademarks

Kendall Whitehorse approach via DVIPS:

http://www.adobe.com/supportservice/custsupport/SOLUTIONS/2d7a.htm

http://www.adobe.com/supportservice/devrelations/devtechnotes.html

Hyptex hype:

http://xxx.lanl.gov/hypertex/

Explain about hyper-text jumps requiring (i) WebLink plugin (ii) on line 

ytex pdf-from.tex

Then run DVIPSONE with -k -*2 -*c on command line

dvipsone -v -k -*2 -*c pdf_from

Then edit PDF file for /Differences 

94/circumflex	=>	39/quoteright 94/circumflex 96/quoteleft

Then read into Acrobat Exchange 3.0 (Viewer32.exe)

and `Save As' to get the optimization
