% plain TeX Document dvi_help.tex
% Copyright 2007 TeX Users Group.
% You may freely use, modify and/or distribute this file.

% DVIWindo release 1.2

% This is the Times/Courier/Symbol version using Windows TrueType fonts

% Copyright 1994 Dr. Donald P. Story,
% Department of Mathematical Sciences,
% The University of Akron,
% Akron, OH  44325
% phone:  (216)972-7514
% e-mail: dpstory@uakron.edu

\magnification=1200

% following three needed for that stupid LaTeX logo...

\font\sevenrm=times at 7bp
\font\sevenbf=timesbd at 7bp
\font\sevensl=timesi at 7bp

% \input lcdplain
% \input mtplain
\input dvi_help.sty

% \def\LaTeX{La\TeX}			% original definition

% \def\LaTeX{%
%   \setbox0\hbox{T}%
%   \dimen0=\ht0
%   L\kern-.3em%
%   \setbox2\hbox{%
%      \expandafter$\expandafter\fam\the\fam\scriptstyle A$}%   
%   \advance\dimen0-\ht2
%   \raise\dimen0\box2
%   \kern-.15em%
%   \TeX
%   \null}

\def\LaTeX{%
  \setbox0\hbox{T}%
  \dimen0=\ht0
  L\kern-.3em%
  \setbox2\hbox{%
  \sevenrm A}%
  \advance\dimen0-\ht2
  \raise\dimen0\box2
  \kern-.15em%
  \TeX
  \null}

\def\LaTeXbf{%
  \setbox0\hbox{T}%
  \dimen0=\ht0
  L\kern-.25em%
  \setbox2\hbox{%
  \sevenbf A}%
  \advance\dimen0-\ht2
  \raise\dimen0\box2
  \kern-.13em%
  \TeX
  \null}

\def\LaTeXsl{
  {\sl
  \setbox0\hbox{T}%
  \dimen0=\ht0
  L\kern-.2em%
  \setbox2\hbox{%
  \sevensl A}%
  \advance\dimen0-\ht2
  \raise\dimen0\box2
  \kern-.1em%
  \TeX
  \null}}

\def\Y&Y{Y\kern-.2em{\sevenrm \&}\kern-.2emY}

% \input ansiacce % makes ^ and ~ also problematic...

% `quoteright' has moved from 39 to 146 make space for `quotesingle' in ANSI:

\catcode`\'=\active \chardef'=146

% `quoteleft' has moved from 96 to 145 make space for `grave' in ANSI:

\catcode`\`=\active \chardef`=145

\overfullrule=1pt	% limit ugliness somewhat...

% This has been tweaked to only use: 

% Times New Roman=times
% Times New Roman Italic=timesi
% Times New Roman Bold=timesbd
% Courier New=cour
% Symbol=symbol
% Arial=arial

\font\times=times at 10bp
\font\timesi=timesi at 10bp
\font\timesbd=timesbd at 10bp
\font\cour=cour at 10bp
\font\sy=symbol at 10bp

\def\rm{\times}
\def\it{\timesi}
\def\bf{\timesbd}
\def\tt{\cour}

\rm

\def\AmSTeX{{\it AMS}-\TeX} % no calligraphic letters available

\let\sl=\it	% avoid slanted Times

% \font\Chapterfont=cmr17
% \font\Chapterfont=lbr at 15pt
% \font\Chapterfont=tir at 17pt
% \font\Chapterfont=times at 17pt
\font\Chapterfont=arial at 17bp

% Follow appropriate for Computer Modern and maybe Lucida Bright:

% \def\asterisk{$\ast$}
% \def\plus{$+$}
% \def\minus{$-$}
% \def\bullit{$\bullet$}
% \def\ellipses{$\ldots$}
% \def\copyrght{$\copyright$}

% Follow appropriate for Times/Courier and maybe Lucida Bright:

\def\minus{{\sy\char45}}
% \def\asterisk{{\sy\char42}}
% \def\plus{{\sy\char43}}
% \def\bullit{{\sy\char183}}
% \def\ellipses{{\sy\char188}}
% \def\copyrght{{\sy\char211}}
\def\asterisk{\char42}
\def\plus{\char43}
\def\bullit{\char149}
\def\ellipses{\char133}
\def\copyrght{\char169} 

\hsize=5truein \vsize=8.8truein \hoffset=.75truein

\title{HomePage}DVIWindo Help File.

\hb{DVIWindo}, a product of \hb{{\Y&Y}, Inc.}, is a Windows-based
application for viewing \ref{DVI}{DVIfile}-files on IBM-PC compatible
computers.
   \bigskip

\List{ {The Menu System}{Menu} \\
       {Mouse Tools/Tricks}{MouseTools} \\
       {A Short Tour of DVIWindo}{Tour} \\
       {Getting Around}{Around} \\
       {Glossary of Terms}{Glossary} \\
       {Index of Basic Functions}{Index} \\}

\medskip

Words in {\it green\/} are hypertext buttons --- click on them with
the left mouse button to jump to an explanation.  Use the right mouse button
to return from a 
% hypertext 
jump  (Make sure you have turned off
\ref{Color Fonts}{ColorFonts}\ in the \hb{Fonts}\ menu).
	\medskip
You can also use
\ref{Search}{Search}\ and \ref {Search Again}{Sagain}\ 
in this file.
	\medskip
Should you get lost, press the `\hb{Home}' key to return to this menu.
	\medskip
To exit, press the `\hb{Esc}' key.  If you called up this help file
from DVIWindo's \TeX~Menu, you will then return to 
that copy of 
DVIWindo.
	\bigskip

\def\nl{\hfill\break}

Copyright \copyrght\ 1994 Dr.\ Donald P.\ Story,
    \medskip
Department of Mathematical Sciences \nl
The University of Akron \nl
Akron, OH  44325 \nl

% phone:  (216)972-7514
% e-mail: dpstory@uakron.edu

\newpage

\title{Menu}The Menu System.

\List{ {^File}{File}\\
       {P^references}{Preferences}\\
       {^Previous!}{Previous}\\
       {^Next!}{Next}\\
       {^UnMagnify!}{UnMagnify}\\
       {^Magnify!}{Magnify}\\
       {F^onts}{Fonts}\\
       {^TeX}{TeXMenu}\\}
       
\newpage

\title{File}The File Menu.

\List{ {^Open\ellipses}{Open}               &\key{Ins}\\
       {^Close}{Close}                     &\key{Del}\\
       {^Print}{Print}                     &\key{Ctrl-P}\\
       {Prin^t View}{PrintV}               &\key{Ctrl-T}\\
       {Pr^inter Setup\ellipses}{PrtSetup}  &\key{Ctrl-I}\\
       {Se^lect Page\ellipses}{SelPage}     &\key{Num \asterisk}\\
       {Sea^rch\ellipses}{Search}           &\key{Ctrl-R}\\
       {^Search Again}{SAgain}             &\key{Ctrl-S}\\
       {^Zoom in\ellipses}{Zoom}            &\key{Ctrl-Z}\\
       {Restore ^View}{RestoreV}           &\key{Ctrl-V}\\
%      {^Copy to Clipboard}{CopyClip}      &\key{Ctrl-C}\\
       {Copy to Clipboard}{CopyClip}      &\key{Ctrl-C}\\
       {Store ^New Page Scale}{StoreScale} &\key{Ctrl-Q}\\
       {^Default Scale}{Default}           &\key{Ctrl-D}\\
       {^Factory Defaults}{Factory}        &\\ % bkph
       {Pa^ge Top}{pTop}                   &\key{Ctrl-N}\\
       {Page ^Bottom}{pBottom}             &\key{Ctrl-B}\\
       {^About DVIWindo\copyrght\ellipses}{AboutDVI}&\\
       {E^xit}{Exit}                       &\key{Esc}\\}

\newpage

\title{Preferences}The Preferences Menu.

\List{ {P^age Size\ellipses}{PageSize}	      &\\
       {^Ruler Units\ellipses}{RuleUnits}     &\\
       {Ruler Dimensions ^True}{RuleDim}      &\\
       {Spr^ead}{Spread}		      &\key{Ctrl-E}\\
       {^Use count0}{count}		      &\key{Ctrl-U}\\
       {^Landscape}{Landscape}		      &\key{Ctrl-L}\\
       {Reset Scale Each ^File}{ResetFile}    &\\
       {Reset Scale Each ^Page}{ResetPage}    &\\
%       {Draw ^Page Border}{DrawBorder}	      &\\
       {Draw Page ^Border}{DrawBorder}	      &\\
       {Draw Text ^Outline}{TextOutline}      &\\
%       {Ignore ^All Specials}{IgnSpec}	      &\\
       {^Ignore All Specials}{IgnSpec}	      &\\
       {Show Pre^views}{ShowPrev}	      &\\
       {Pa^ss Through EPS}{PassEPS}	      &\\
       {Complain Missing Fonts}{CompFonts}    &\\
       {Complain Bad Encoding}{CompEncoding}  &\\ % new bkph
       {Complain Missing Files}{CompFiles}    &\\
       {Complain Bad Specials}{CompSpec}      &\\
%       {^Save Preferences}{SavePref}	      &\\
       {Save Settings on E^xit}{SavePref}     &\\
       {Save Settings No^w}{SaveNow}	      &\\} % new bkph
      
\newpage

\title{Fonts}The Fonts Menu.

\List{ {D^VI File Fonts}{DVIFonts}        &\key{Ctrl-K}\\
       {DVI File Inf^o}{DVIFileInfo}      &\key{Ctrl-O}\\
       {^Color Fonts}{ColorFonts}         &\key{Ctrl-X}\\
       {^Grey Text}{GreyText}             &\key{Ctrl-G}\\
       {Text \plus\ Gre^y}{TextGrey}         &\key{Ctrl-Y}\\
       {Show ^Buttons}{ShowButtons}       &\\
       {F^ill Rules}{FillRules} 	  &\key{Ctrl-J}\\
       {Favor ^Positions}{FavPos}         &\\
       {Show ^Font\ellipses}{ShowFonts}    &\key{Ctrl-F}\\
       {Show ^Widths\ellipses}{ShowWidths} &\\
       {Write ^AFM\ellipses}{WriteAFM}     &\\
       {Write ^TFM\ellipses}{WriteTFM}     &\key{Ctrl-M}\\ % new bkph
       {Write All TFMs\ellipses}{WriteAllTFMs}     &\\ % new bkph
       {View ^Hidden}{ViewHidden} 	  &\\       % new bkph
       {^System Info}{SystemInfo} 	  &\\}       % new bkph
\newpage

\title{TeXMenu}The \TeX~Menu.

The \hb{\TeX{}}\ menu is a user {\it customizable\/} pulldown menu. 
The only hard-wired entry is the first one, the \hb{Preview}.	If
no \ref{DVI}{DVIfile}-file is open for viewing, you can % quickly 
load the file {\it last viewed} by clicking on the \hb{Preview}\ item
under the \hb{\TeX{}}\ menu.  The default accelerator key for
\hb{Preview}\ is `\key{Ctrl-F1}', though this can be changed by editing
the \key{dviwindo.ini}\ file.  To change the accelerator key, open the
file \key{dviwindo.ini}\ using an editor such as `NotePad'.  Under the
section labeled `[Window]', add the line `PreviewHotKey=\key{F1}'.
This will change the preview hot key to \key{F1}, for example.
   \bigskip
When \hb{DVIWindo}\ is installed, several other items may  be added to
the \hb{\TeX{}}\ menu for `\TeX{}ing' documents: \ref{ini\TeX}{iniTeX},
\ref{\TeX}{TeX}, \ref{\LaTeX}{LaTeX}, and \ref{\AmSTeX}{AmSTeX}.  
Also included are some utilities: \ref{CleanUp}{CleanUp}\ and
\ref{SysSeg}{SysSeg}.
   \bigskip
In order 
to add or delete a menu item (or add/delete a separator rule),
open the \hb{\TeX{}}\ menu, hold down the `\key{Ctrl}' key and click on 
any 
menu item listed, except \hb{Preview}.  The `Edit Menu Item' dialog box
will appear; it will already be filled in by the menu item you clicked. 
To enter a new item, fill in `Menu Item Name (key):', and  `Application
Call Pattern (value):', then click either {\bf Add}, to add the item
({\it after\/} the chosen item), or {\bf Replace}, to replace the old item.
   \bigskip
There are also the buttons: {\bf Delete}, to delete the the current
item; {\bf Delete Next}, to delete the next menu item listed; {\bf
Separator}, to create a separator bar; and {\bf Cancel}, to exit with
making any changes.
   \bigskip
To see the format of the call patterns, look first at existing entries.

\newpage

\title{Pagination}The Previous!/Next!~Menus.

\altsec{Previous}The Previous!~Menu.

Click the \hb{Previous!}\ menu item with the left mouse button to move
to the next page.  The \hb{Previous!}\ can also be executed by pressing
the `\key{PgUp}' key, or the `\key{Backspace}' key.  For a more extensive
discussion on pagination, see 
\hb{Moving Between}\ \ref{Consecutive}{Paginate}\ \hb{Pages}.
   \bigskip

\altsec{Next}The Next!~Menu.
  
Click the \hb{Next!}\ menu item with the left-mouse to move to the
previous page.  The \hb{Next!}\ can also be executed by pressing
the `\key{PgDn}' key, or the `\key{Spacebar}'.  For a more extensive
discussion on pagination, 
see \hb{Moving Between}\ \ref{Consecutive}{Paginate}\ \hb{Pages}.
   \bigskip

\newpage   

\title{magnification}The UnMagnify!/Magnify! Menus.

\altsec{UnMagnify}The UnMagnify!~Menu.

Click the \hb{UnMagnify!}\ menu item with the left mouse button to
unmagnify the screen by a small amount.  This same function can be
accomplished by pressing the `\minus' key on the numeric key pad.  To decrease
magnification by greater increments, click \hb{UnMagnify!}\ using the
left mouse button while pressing, at the same time, the \key{Shift} key.
Do a \ref{Restore View}{RestoreV}\ to return to the original magnification.
   \medskip
See also \ref{Restore View}{RestoreV}.
   \bigskip

\altsec{Magnify}The Magnify!~Menu.

Click the \hb{Magnify!}\ menu item with the left mouse button to magnify
the screen by a small amount.  This same function can be accomplished by
pressing the `\plus' key on the numeric pad.  To increase magnification by
greater increments, click \hb{Magnify!}\ using the left mouse button
while pressing, at the same time, the \key{Shift} key. \ref{Restore
View}{RestoreV}\ may be used to return to the original magnification.
    \medskip
See also \ref{Zoom in}{Zoom}.

\newpage

\section{Open}{File}File/Open\ellipses. Key: Ins.

To view a DVI file, select \hb{Open\ellipses}\ from \hb{DVIWindo}'s 
\hb{File}\ menu. A file selection menu pops up listing files,
directories and drives.  To move downward in the directory structure,
click on one of the listed directories; move upwards by clicking on
`[..]'.  A file name can also be entered directly.  If the file name
contains `wildcard' characters (`\asterisk' or `?'), then a list of
files matching the wildcard specification is shown. Click {\bf OK} when
the desired file as been selected.
   \bigskip

\section{Close}{File}File/Close. Key: Del.

When you are finished viewing the current DVI file, choose \hb{Close}\ from
the \hb{File}\ menu.  The viewing screen will `blank out'.  At this point
you can either \ref{Open}{Open}\ another file, or \ref{Exit}{Exit}\
\hb{DVIWindo}.
   \bigskip

\newpage
   
\section{Print}{File}File/Print. Key: Ctrl-P.

The purpose of this menu item is to print all or a portion of the
current document.   The default printer will be used unless another
printer has been previously selected using
\ref{Printer Setup\ellipses}{PrtSetup}.
   \bigskip

The dialog box that appears when \hb{Print\ellipses}\ is selected makes it
possible to request \ref{DVIPSONE}{DVIPSONE}\ -- if installed -- be called
to do the printing.  Simply check the `{\bf Use DVIPSONE}' box.  Within this
dialog box, you can also choose a range of pages to be printed; there are
check boxes for printing in reverse order, printing odd pages only, or
printing even pages (reverse order).  The number of copies may also be
specified. 
   \bigskip

\section{PrintV}{File}File/Print View. Key: Ctrl-T.

To print the page currently being viewed, select \hb{Print View}\ from
the \hb{File}\ menu.  To set the printer see \ref{Printer
Setup\ellipses}{PrtSetup}.  Note that the page will be printed using the
current \hb{mapping mode}.  This is useful, for example, for seeing part
of a page magnified.  To have the full page properly positioned on the
paper instead, select \ref{Default Scale}{Default}\ from the 
\hb{File}\ menu before selecting \hb{Print View}.

\newpage
   
\section{PrtSetup}{File}File/Printer Setup. Key: Ctrl-I.

In order to choose another printer (and configure it as desired), select
\hb{PrtSetup\ellipses}\ from the \hb{File}\ menu.  If the correct printer
has already been selected, click {\bf Cancel}. The \hb{Printer Setup}\ 
menu is a standard Windows menu; for more information on setting up your
printer, see the help file in the 
\launch{Print Manager}{winhelp.exe printman.hlp}.
   \bigskip

\section{SelPage}{File}File/Select Page\ellipses. Key: `\asterisk' on NumPad.

To move to an arbitrary page, use \hb{Select Page\ellipses}\ from the
\hb{File}\ menu.  The \hb{Select Page\ellipses}\ may be selected two other
ways: (1) By pressing `\asterisk' on the number key pad, following by the
number of the desired page; or (2)  Entering a numerical value followed by
`\key{Enter}'; however, if the value is entered on the number key pad, the
number lock must be on.

\newpage

\section{Search}{File}File/Search. Key: Ctrl-R.

In order to make a text string search, use \hb{Search\ellipses}\ from the
\hb{File}\ menu.  Specify a search string in the popup dialog box.  A
check box allows one to select case-sensitive search. Another check box
allows selection of wrap-around at the end of the file back to the
beginning.  The page containing the first matching string found is shown
with the arrow cursor placed right after the string.  The search always
starts on the page currently being viewed. To repeat the the search,
select \ref{Search Again}{SAgain}\ (\key{Ctrl-S}).
   \medskip
Since DVI files do not contain information on where words start and
end, you may at times get a spurious match.  Note that text search
works even if a word contains an `fi' or `fl' ligatures.
   \bigskip

\section{SAgain}{File}File/Search Again. Key: Ctrl-S.

A search for the next instance of the text search string is obtained by
using \hb{Search Again}\ from the \hb{File}\ menu.  It is usually more
convenient though to use the accelerator key `\key{Ctrl-S}'.
A message box appears if the string is not found.
    \medskip
Note that using the Search facility puts up a `modeless' dialog box.
You can remove this dialog box by selecting `Close' from its `System Menu'.

\newpage

\section{Zoom}{File}File/Zoom In. Key: Ctrl-Z.

To zoom in on a part of an image, select \hb{Zoom In}\ from the
\hb{File}\ menu --- or hold down the `\key{Ctrl}' key while pressing
and holding down the left mouse button.  
The cursor will change into a magnifying glass.  
By pressing and holding down the left mouse button, you can define a
rectangular zoom region.  The region will be magnified upon release of
the left mouse button.  While the left mouse button is depressed,
should you want to move the rectangle to another location, simply press
the right mouse button -- the rectangle is now movable.  To undo the
magnification, you need to \ref{Restore View}{RestoreV}\ 
(`\key{Ctrl-V}').
   \medskip
Note that the rectangle defined this way will have the shape (aspect
ratio) of the current window.
   \medskip
For additional \hb{Mouse Tricks}\ for Zoomin in, see 
% \ref{The Magnifying Glass Icon}{MagGlass}
\ref{The Magnifying Glass Cursor}{MagGlass}
   \bigskip


\section{RestoreV}{File}File/Restore View. Key: Ctrl-V.

To restore the screen to the original magnification after viewing the
enlarged region, use the \hb{Restore View}\ in the \hb{File}\ menu.  The
same function can also be executed by either pressing the accelerator
key `\key{Ctrl-V}', or  by clicking the right mouse button
--- unless hypertext linkage is being used, in which case the right
mouse button is used to return from a hypertext jump.
    \medskip
Note that a stack of screen magnifications and positions is maintained
by \hb{DVIWindo}, so that it is possible to `zoom in' several times, yet
easily undo the changes in magnification.


\newpage

\section{CopyClip}{File}File/Copy to Clipboard. Key: Ctrl-C. % check

\hb{DVIWindo}\ can `copy' a specified region to the Clipboard.  This
region can then be `pasted' into another application.  The pasted region
acts as a graphical object that can be moved, scaled and cropped.  Typeset
math using \TeX, then import into your favorite WYSIWYG application.
   \bigskip
To copy a rectangular region of the screen to Window's clipboard, hold
down the `\key{Shift}' and `\key{Ctrl}' keys, press the left mouse
button at one corner of the rectangle -- the cursor changes to a
scissors -- drag to the opposite corner of the rectangle -- then release
the mouse button.  To move the rectangle while it is being defined, hold
down the right mouse button as well as the left.  To copy the whole page
to the clipboard, simply bring the cursor back to where you started --
\hb{DVIWindo}\ copies everything to the clipboard if the rectangle is very
small.
   \bigskip

\newpage   
   
\section{Default}{File}File/Default Scale. Key: Ctrl-D.

The \hb{Default Scale}\ controls the default `top of page' screen
mapping and the default magnification scale that \hb{DVIWindo}\ 
displays a DVI-file at.  This is a ``hard-wired'' value that cannot be
changed; however, see \ref{Store New Page Scale}{StoreScale}\ for a
workable alternative.
   \bigskip
When printing using \ref{Print View}{PrintV}, the page will print using
the magnification currently in effect.  To have the full page properly
positioned on the paper, select \hb{Default Scale}\ from the 
\hb{File}\ menu before selecting \ref{Print View}{PrintV}.
   \bigskip

\section{Factory}{File}File/Factory Defaults. Key: \none. % new bkph

\hb{Factory Default}\ resets most preferences to `out of the box' default.
This can be handy when preferences inadvertently are changed to values
that make the text appear off screen, or as grey blocks, for example.
   \bigskip
Note that 
% \ref{Save Preferences}{SavePref}\ 
\ref{Save Settings on Exit}{SavePref}\ 
is reset by this action so as to avoid making the factory defaults the
new permanent preferences. You can override this by explicitly checking
% \ref{Save Preferences}{SavePref}.
\ref{Save Settings on Exit}{SavePref}.
   \bigskip

\newpage

\section{StoreScale}{File}File/Store New Page Scale. Key: Ctrl-Q.

The \hb{Store New Page Scale}\ controls two aspects of how
\hb{DVIWindo}\ functions: (1) it defines the `top of page'; and (2) it
determines the `user-defined default screen magnification'.
   \bigskip
It is possible to set the `top of page' screen mapping to something
other than the \ref{default screen mapping}{Default}. First, position
the page on the screen at a point you want \hb{DVIWindo}\ to consider
the `top of page', then  select \hb{Set New Page Scale}\ from the
\hb{File}\ menu.  Now when you press `\key{Page Down}' or click on
\hb{Next!}\ (and if \ref{Reset Scale Each Page}{ResetPage}\ has been
checked), \hb{DVIWindo}\ will advance to the next page, and move to the
`top of page' as defined by \hb{Store New Page Scale}.
  \bigskip
Similarly, to set the `user-defined default screen magnification',
adjust the magnification to the desired level using the
\hb{UnMagnify!/Magnify!}\ menus, or the `\key{\plus/\minus}' keys, then
click on the \hb{Store New Page Scale}\ (be sure to position page to
the desired `top of page' as well).
   \bigskip
To get the `default screen mapping' simply click on \ref{Default
Scale}{Default}.
   \bigskip

\newpage

\section{pTop}{File}File/Page Top. Key: Ctrl-N.

This menu item moves the screen to the top of the page. What is
considered the appropriate position for the top of the page is
determined by the positioning in force when \ref{Set New Page
Scale}{StoreScale}\ was last invoked.
   \bigskip

\section{pBottom}{File}File/Page Bottom. Key: Ctrl-B.

This menu item moves the screen to the bottom of the page. The position
for the bottom of the page, on the other hand, is determined by the page
depth specified in the DVI file (which may not be accurate).
   \bigskip

\newpage

\section{AboutDVI}{File}File/About DVIWindo\copyrght. Key: \none.

Copyright, version number, and licensed user located here.
Please note that \hb{DVIWindo}\ is normally licensed to a single
end user.  Contact \hb{{\Y&Y}, Inc.}\ for generous site license terms.
   \bigskip

\section{Exit}{File}File/Exit. Key: Esc.

To exit \hb{DVIWindo}, select \hb{Exit}\ from the \hb{File}\ menu, or simply
press the \key{Esc} key.  It is not necessary to explicitly close the file
currently being viewed since \hb{DVIWindo}\ does not leave open files.
Preferences will be saved if the 
% \ref{Save Preferences}{SavePref}\ 
\ref{Save Settings on Exit}{SavePref}\ 
has been checked.
   \bigskip
It is possible to exit \hb{DVIWindo}\ without saving your preferences 
by holding down the `\key{Shift}' key while exiting, see
% \ref{Save Preferences}{SavePref}\ 
\ref{Save Settings on Exit}{SavePref}\ 
for more details.
   \bigskip
Note: if a dialog box is open when you press the \key{Esc} key, then that
dialog box is closed, rather than \hb{DVIWindo}\ itself.
Press the \key{Esc} key again to close \hb{DVIWindo}.

\newpage

\section{PageSize}{Preferences}Preferences/Page Size. Key: \none.

Different page sizes can be specified using the \hb{Page Size}\ in the
\hb{Preferences}\ menu. Page size choices affect the \ref{page border}
{DrawBorder}\ that is drawn on screen, and are also
passed along to the printer driver when \ref{Print View}{PrintV}\ or
\ref{Print\ellipses}{Print}\ are selected.
   \bigskip

\section{RuleUnits}{Preferences}Preferences/Ruler Units. Key: \none.

Choose the desired units from \TeX's standard list of two-letter
abbreviations in the \hb{Ruler Units}\ sub-menu of \hb{Preferences}.
These measurements will be the ones displayed when the user activates
the \ref{measurement tool}{meastool}.  Measurements can be given either
in terms of the original \TeX\ source document or in the final (`true')
version - as affected by \TeX's magnification.  Check the \ref{Ruler
Dimensions True}{RuleDim}\ in the \hb{Preferences}\ menu to get the
magnified measurements.
   \bigskip

\section{RuleDim}{Preferences}Preferences/Ruler Dimensions True. Key: \none.

When using the \ref{measurement tool}{meastool}, measurements can be given
either in terms of the original \TeX\ source document or in the final (`true')
version - as affected by \TeX's magnification.  Check the
\hb{Ruler Dimensions True}\ in the \hb{Preferences}\ menu to get
the magnified measurements.
   \bigskip

\newpage   
   
\section{Spread}{Preferences}Preferences/Spread. Key: Ctrl-E.

Check \hb{Spread}\ in the \hb{Preferences}\ menu to see two pages side by
side.  In this mode \hb{Next!}\ advances by two pages, and
\hb{Previous!}\ retreats by two.  Hold down `\key{Shift}' when clicking
\hb{Next!}\ or \hb{Previous!}\ to move in {\it single\/} page increments.
   \bigskip

\section{count}{Preferences}Preferences/Use count0. Key: Ctrl-U.

The page number shown in the \hb{DVIWindo} % window
title bar can be either: (a)
\key{dvipage} -- which is simply the sequential position of the current
page within the DVI file -- or (b) \key{count0} -- used for the actual page
number in plain \TeX\ and many other \TeX-based programs.
   \bigskip

% \newpage
   
\section{Landscape}{Preferences}Preferences/Landscape.  Key: Ctrl-L.

It is possible to check the \hb{Landscape}\ orientation in the
\hb{Preferences}\ menu.  This reorients the page outline on the screen
and is passed along to the printer driver when the user selects either
\ref{Print View}{PrintV}, or \ref{Print\ellipses}{Print}.
It does {\it not\/} rotate text on the screen. 
   \bigskip

\newpage

\section{ResetFile}{Preferences}Preferences/Reset Scale Each File.
   Key: \none.

To reset the image mapping whenever a new file is opened, check \hb{Reset
Scale Each File}\ from the \hb{Preferences}\ menu.  The image mapping is
reset to the `Page Scale', which can be set by clicking the
\ref{Set New Page Scale}{StoreScale}\ in the \hb{File}\ menu.
   \bigskip

% \newpage
   
\section{ResetPage}{Preferences}Preferences/Reset Scale Each Page.
   Key: \none.

To reset the image mapping when moving to a new page, check \hb{Reset Scale
Each Page}\ from the \hb{Preferences}\ menu.  The image mapping is reset to
the `Page Scale', which can be set by clicking the
\ref{Set New Page Scale}{StoreScale}\ in the \hb{File}\ menu.
   \bigskip
The default is to go to the top of the page when selecting \hb{Next!},
and to go to the bottom of the page when selecting \hb{Previous}.  This
behavior can be controlled using \hb{Reset Scale each Page}\ in the
\hb{Preferences}\ menu.  What is considered the appropriate position for
the top of the page now is determined by the positioning in force when
\hb{Set New Page Scale}\ was last invoked.  The position for the bottom of
the page, on the other hand, is determined by the page depth specified in
the DVI file (which may not be accurate).
   \bigskip

\newpage
   
\section{DrawBorder}{Preferences}Preferences/Draw Page Border. Key: \none.

\hb{DVIWindo}\ will draw the page outline if \hb{Draw Page Border}\ is
checked in the \hb{Preferences}\ menu.
The shape and size of the page outline is determined by the selection
of \ref{Page Size}{PageSize}.
   \bigskip

\section{TextOutline}{Preferences}Preferences/Draw Text Outline. Key: \none.

\hb{DVIWindo}\ will draw the rectangular area used for text, if
\hb{Draw Text Outline}\ is checked in the \hb{Preferences}\ menu.  This text
outline is not always meaningful since it is based on incomplete information
obtained from the DVI file.
   \bigskip

\newpage

\section{IgnSpec}{Preferences}Preferences/Ignore All Specials. Key: \none.

When checked, \hb{DVIWindo}\ will ignore all specials.  This, among
other things will turn off color fonts and the hypertext capabilities. 
   \bigskip

\section{PassEPS}{Preferences}Preferences/Pass Through EPS. Key: \none.

\hb{DVIWindo}\ can pass through EPS files when printing to a PS printer
(instead of printing the TIFF or EPSI preview in an EPSF file).
Use a check box \hb{Pass Through EPS}\ in the \hb{Preferences} menu to
control which is used.
   \bigskip

\section{ShowPrev}{Preferences}Preferences/Show Previews.  Key: \none.

\hb{DVIWindo}\ will show previews in inserted EPS graphics if the 
EPS files contains a preview in either TIFF or EPSI format. This
can be turned off by unchecking \hb{Show Previews}\ in the
\hb{Preferences}\ menu. 
   \bigskip

\newpage

\section{CompFonts}{Preferences}Preferences/Complain Missing Fonts. Key: \none.

Normally \hb{DVIWindo}\ complains when it cannot find a scalable font
installed in Windows (either Adobe Type 1 or TrueType) that matches
the TFM file name in the DVI file.
When checked, messages concerning missing fonts are suppressed.
   \bigskip

\section{CompEncoding}{Preferences}Preferences/Complain Bad Encoding. Key: \none.

\hb{DVIWindo}\ checks whether the TFM files used by \TeX\ when creating
the DVI file were set up using the encoding
used by \hb{DVIWindo}\ as controlled by the `ENCODING' environment
variable and/or the `TEXANSI' environment variable.  
Note that this applies only to {\it text\/} fonts, and only works for
TFM files made using AFMtoTFM.  You may want to turn this check off if
you are using TFM files made some other way. 
    \bigskip

\section{CompFiles}{Preferences}Preferences/Complain Missing Files. Key: \none.

\hb{DVIWindo}\ complains if it can't find an inserted EPS, TIFF, or WMF
files. This behavior can be disabled by unchecking \hb{Complain Missing
Files}\ in the \hb{Preferences}\ menu.
   \bigskip

\section{CompSpec}{Preferences}Preferences/Complain Bad Specials. Key: \none.

\hb{DVIWindo}\ complains if it does not understand a `special' that it thinks
it should be able to understand.  This behavior can be disabled by
unchecking \hb{Complain Bad Specials}\ in the \hb{Preferences}\ menu.
   \bigskip
   
\newpage   

% \section{SavePref}{Preferences}Preferences/Save Preferences. Key: \none. 
\section{SavePref}{Preferences}Preferences/Save Settings on Exit. Key: \none.

Preferences made by the user are saved at the end of the session in a private
profile file named `\key{dviwindo.ini}' in the Windows directory.
Preferences are saved automatically if the 
% \hb{Save Preferences}\ 
\hb{Save Settings on Exit}\ 
item is `checked' in the \hb{Preferences}\ menu.
   \bigskip
The saving of preferences may be {\it temporarily\/} disabled by holding
down the `\key{Shift}' key while selecting \ref{Exit}{Exit}\ from the
\hb{File}\ menu.  However, if the 
% \hb{Save Preferences}\ 
\hb{Save Settings on Exit}\ 
is {\it unchecked}, when holding down `\key{Shift}' while exiting will
{\it temporarily reenable\/} saving of preferences.

   \bigskip

\section{SaveNow}{Preferences}Preferences/Save Settings Now. Key: \none.

You can force \hb{DVIWindo} to write out the current settings
immediately by selecting \hb{Save Settings Now} from the
\hb{Preferences} menu.

\newpage

\section{DVIFonts}{Fonts}Fonts/DVI Fonts. Key: Ctrl-K.

To see what fonts are used in a particular DVI file, {\bf Check}
\hb{DVI Fonts}\ in the \hb{Fonts}\ menu to create a dialog box with a
list of all fonts and the sizes at which they are used.	 {\bf Uncheck}
\hb{DVI Fonts}\ to remove the dialog box, or simply select {\bf Close}
from the dialog's own {\bf System} menu.
   \bigskip
To see what part of the text is in which font, click on a font in the
dialog box.  All parts of the text using that font, in the indicated
size, will appear in `reverse video'.
   \bigskip
It is also possible to go the other way:  to see what font is used for a
particular part of the text, click on the text holding down the
`\key{Shift}' key.  All text with the same font and size with be
highlighted on the page, and the font name will be highlighted in the
font list.
   \bigskip
There is one additional feature in this regard.  Move the mouse cursor
to a particular character of interest; while holding down `\key{Shift}'
and `\key{Ctrl}', click on the character with the left mouse button.  
A dialog box will appear containing diverse information about the font
that character is displayed in.
   \bigskip
While utilizing any of the above mentioned features, try ckecking the
\ref{Color Fonts}{ColorFonts}\ under the \hb{Fonts}\ menu.

\newpage

\section{DVIFileInfo}{Fonts}Fonts/DVI File Info. Key: Ctrl-O.

When \hb{DVI File Info}\ is chosen form the \hb{Fonts}\ menu, a modeless
box pops up containing a variety of information contained in the DVI
file, including number of pages, number of fonts, number of bytes,
magnification factor, creation date, and much, much more.
   \bigskip
   
\section{ColorFonts}{Fonts}Fonts/Color Fonts. Key: Ctrl-X.

On a color monitor, it is possible to color code fonts and sizes.  Simply
click \hb{Color Fonts}\ in the \hb{Preferences}\ menu.  This is useful in
combination with using \ref{DVI Fonts}{DVIFonts}\ in the
\hb{Preferences}\ menu.
   \medskip
Note: Turn \hb{Color Fonts} off while printing.  On a black and white printer,
colors are simulated using grey levels, and grey-levels in turn
are presented using half-tone patterns, which give the letters a 
mottled or fuzzy look.
   \bigskip

\newpage

\section{GreyText}{Fonts}Fonts/Grey Text. Key: Ctrl-G.

When looking for `widows' and `orphans', or checking the layout of your
document, checking \hb{Grey Text}\ in the \hb{Preferences}\ menu speeds up
screen updating considerably.  Each block of contiguous characters in the
DVI file is represented by a grey rectangle.
(See also \ref{Text \plus\ Grey}{TextGrey}.)
   \bigskip

\section{TextGrey}{Fonts}Fonts/Text \plus\ Grey.   Key: Ctrl-Y.

Each block of contiguous characters in the DVI file is represented by a grey
rectangle, with the text superimposed.  This also works with
\ref{Color Fonts}{ColorFonts}.
   \bigskip

\section{FillRules}{Fonts}Fonts/Fill Rules. Key: Ctrl-J.

Showing outlined rules as opposed to filled rules can be helpful when
trying to accurately fit together rules to draw boxes and grids.
\ref{Zoom in}{Zoom}\ to check whether rules are drawn as outlines
or as filled rectangles (normal presentation).
    \medskip
If rules look outlined rather than filled (or if they look unusually dark),
check the \hb{Fill Rules}\ in the \hb{Fonts}\ menu.
   \bigskip

\newpage

\section{ShowFonts}{Fonts}Fonts/Show Font\ellipses.  Key: Ctrl-F.

% reference p. 29

A list of all scalable fonts known to Windows may be obtained by selecting
\hb{Show Font\ellipses}\ from the \hb{Fonts}\ menu in \hb{DVIWindo}.
Double-click on a font name to see all the characters in that font.  It is
possible to choose `regular,' `bold,' `italic' or `bold italic'
styles of that font (if available).
   \bigskip
When viewing a font character set, you can magnify using \ref{Zoom in}{Zoom}\
techniques, or you can \ref{Print}{Print}, or \ref{Print View}{PrintV}.
   \bigskip
If {\bf Show Bounding Boxes} in the `Font Selection' dialog box is checked,
then each character is shown inside a rectangular box.
    \bigskip

\section{ShowWidths}{Fonts}Fonts/Show Widths\ellipses. Key: \none.

Select \hb{Char Widths}\ in the \hb{Fonts}\ menu to see a list of character
widths instead of the character's shapes.  The widths are based on 1000
units per `em' (that is, standard Adobe Type 1 coordinate system).
   \bigskip

\newpage

% \section{WriteAFM}{Fonts}Fonts/Write AFM\ellipses. Key: Ctrl-M.
\section{WriteAFM}{Fonts}Fonts/Write AFM\ellipses. Key: \none.

The item \hb{Write AFM}\ under the \hb{Fonts}\ menu is used to write
out an Adobe Font Metric (AFM) file for any scalable font known to Windows.
The resulting AFM file appears in the AFM sub-directory of the
\hb{DVIWindo}\ directory.
   \bigskip

\section{WriteTFM}{Fonts}Fonts/Write TFM\ellipses. Key: Ctrl-M.

The item \hb{Write TFM}\ under the \hb{Fonts}\ menu is used to write
out a TFM file for any scalable font known to Windows.  
\hb{DVIWindo}\ first writes an AFM file and then invokes AFMtoTFM
with the current encoding as specified by the `ENCODING' environment
variable.
The resulting TFM file appears in the first directory in the
list of directories in the TEXFONTS environment variable.
   \bigskip

\section{WriteAllTFMs}{Fonts}Fonts/Write All TFMs\ellipses. Key: \none.

The item \hb{Write All TFMs}\ under the \hb{Fonts}\ menu is used to write
out TFM files for selected groups of fonts.
It is possible to select all fonts in Adobe Type 1 format,
or all fonts in TrueType form. 
It is possible to select all text fonts, or all symbol/math/decorative
fonts. 
Processing can also be confined to `new fonts' --- those fonts for which
no TFM files exists yet.
   \bigskip

\newpage

\section{FavPos}{Fonts}Fonts/Favor Positions. Key: \none.

When \hb{Favor Positions}\ is checked, each character is placed
individually on the page.  Positional offsets cannot accumulate.  {\it
Output is slowed}. Spacing between characters may be slightly irregular.
   \bigskip
When \hb{Favor Positions}\ is NOT checked, sequences of characters are
placed on the page as a group.  Positional offsets may accumulate.  Offsets
are absorbed between words (and where there is explicit kerning within
a word).
   \bigskip
Usually, \hb{Favor Positions}\ should be left {\it unchecked}.
   \bigskip

\section{ShowButtons}{Fonts}Fonts/Show Buttons. Key: \none.

Should you check \hb{Show Buttons}\ in the \hb{Fonts}\ menu,
\hb{DVIWindo}\ will display all hypertext buttons.
   \bigskip

\newpage

\section{ViewHidden}{Fonts}Fonts/View Hidden. Key: \none.

It is possible to use `specials' to create rules and text that are
only visible in preview and do not get printed.
\hb{Show Hidden}\ in the \hb{Fonts}\ menu controls whether such
text and rules are shown in \hb{DVIWindo}.
    \bigskip

\section{SystemInfo}{Fonts}Fonts/System Info. Key: \none.

\hb{System Info}\ in the \hb{Fonts}\ menu shows some system usage
info, such as how much memory Windows claims is available,
what percentage of the Windows GDI and USER segments (64k byte each)
are used up, and so on.  This information may at times be useful
in debugging Windows problems in systems with either limited resources
(8 meg RAM), or heavy resource usage (such as network TSRs).

\newpage

\title{Around}Getting Around.

Below is a complete % {\it complete\/} 
listing of % all 
keyboard and mouse functions with a brief description of each, along
with pointers to a more detailed explanation.
   \bigskip 

\BeginDblList{1.75truein}{10truept}
\\{\bf Key/Mouse Press}           &{\bf Description}.\cr   
\\\key{Alt}-Left Mouse              &Position page on screen by dragging page
				     using a hand cursor.\cr
\\\key{Ctrl}-Left Mouse             &\ref{Zoom}{Zoom}-rectangle.\cr
\\Right Mouse Click               &\ref{Undo}{RestoreV}\ previous zoom, 
				     or previous hyperlink.\cr
\\\key{Ctrl}-Left-Right Mouse       &\ref{Move the zoom-rectangle}{Zoom}.\cr
\\\key{Shift-Ctrl}-Left Mouse       &\ref{Copy to clipboard}{CopyClip}\ (If
				     rectangle too small, everything is
				     copied to clipboard).\cr
\\\key{Shift}-Left Mouse            &\ref{Measurement tool}{meastool}.\cr
\\\key{Shift}-Left Right Mouse      &\ref{Move}{meastool}\ the measurement
				     tool.\cr
\\Double-Click Mouse              &Refresh Screen.\cr
\\\key{Enter}                       &Refresh Screen.\cr
\\\key{F10}                         &This function key is equivalent to
				     pressing and releasing the Alt key.\cr
\\\key{Esc}                         &\ref{Exit}{Exit}\ \hb{DVIWindo}.\cr
\\\key{Ins}                         &\ref{Open}{Open}\ a file for viewing.\cr
\\\key{PgDn}                        &\ref{Next!}{Next}\ (Go to Next Page).\cr
\\\key{PgUp}                        &\ref{Previous!}{Previous}\
				     (Go to Previous Page).\cr
\\\key{Home}                        &Beginning of File (Go to First Page).\cr
\\\key{End}                         &End of File (Go to Last Page).\cr
\\\key{Ctrl-F1}                     &Preview last file.\cr
\\\key{Ctrl-B}                      &Page Bottom.\cr
\\\key{Ctrl-N}                      &Page Top.\cr
\\\key{Ctrl-R}                      &\ref{Search}{Search}.\cr
\\\key{Ctrl-S}                      &\ref{Search Again}{SAgain}.\cr
\\\key{Ctrl-P}                      &\ref{Print}{Print}.\cr
\\\key{Ctrl-T}                      &\ref{Print View}{PrintV}.\cr
\\\key{Ctrl-F}                      &\ref{Show Font}{ShowFonts}.\cr
\\\key{Ctrl-V}                      &\ref{Restore View}{RestoreV}\ --
				     following a Zoom In.\cr
\\\key{Ctrl-Z}                      &\ref{Zoom In}{Zoom}\ --
				     Magnifying glass appears,
				     depress left mouse to create bounding
				     box.\cr
\\\key{Ctrl}-Left/Right Arrow       &Center page horizontally on screen.\cr
\\\key{Ctrl}-Up/Down Arrow          &Center page vertically on screen.\cr
\\\key{Shift-Esc} 		    &Exits \hb{DVIWindo}, while toggling
                                     the setting of
%				     \ref{Save Preferences}{SavePref}.\cr
				     \ref{Save Settings on Exit}{SavePref}.\cr
\\\key{Ctrl-D}                      &Map screen using \ref{Default Scale}
				     {Default}.\cr
\\\key{SpaceBar}                    &Moves forward one page without
				     changing the window scale or the
				     offsets.\cr
\\\key{Backspace}                   &Moves backward one page without
				     changing the window scale or the
				     offsets.\cr
\\`\asterisk' \plus \key{num}            &Select Page {\tt num}.\cr
\\\key{num}\plus{\tt Enter}           &Select page {\tt num} -- If {\tt num}
				     is entered on the NumPad, the number
				     lock must be on.\cr
\\`\plus' on NumPad                 &Magnify! (Small increments).\cr
\\`\minus' on NumPad                &Unmagnify! (Small increments).\cr
\\\key{Shift}-`\plus' on NumPad     &Magnify! (Large increments).\cr
\\\key{Shift}-`\minus' on NumPad    &UnMagnify! (Large increments).\cr
\\\key{Shift}-Click Magnify!        &Magnify! (Large Increments).\cr
\\\key{Shift}-Click Unmagfify!      &Unmagnify! (Large Increments).\cr
\\\key{Ctrl}-Next!                  &Temporarily inverts the `Reset Scale
				     Each Page' in the Preferences menu.\cr
\\\key{Ctrl}-Previous!              &Temporarily inverts the `Reset Scale
				     Each Page' in the Preferences menu.\cr

\head \TeX\ Menu.

\\\key{Ctrl}                        &Choose a menu item in \TeX\ menu while
				     holding down Ctrl, causes \hb{DVIWindo}\
				     to call the `Edit Menu Item'.\cr
   \bigskip

\head Two-Page Spread Mode.

\\Next!                           &Advance by two pages.\cr
\\Previous!                       &Decrement by two pages.\cr
\\\key{Shift}-Next!               &Advance by a single page.\cr
\\\key{Shift}-Previous!           &Decrement by a single page.\cr

   \bigskip

\head Fonts/DVI Fonts.

\\\key{Shift-Ctrl} Click a char     &Character Information.\cr
\\\key{Shift} Click a char          &Reverse Video of all text using the same
				     font.\cr
\\Next! on last page              &Goes to bottom of page.\cr
\\Previous! on first page         &Goes to top of page.\cr

\head Miscellaneous.		% following added 1995/April/26

%% the above Getting Around section may need checking and expansion

\\\key{Ctrl}-I			&Printer Setup.\cr
\\\key{Ctrl}-C			&Copy to Clipboard.\cr
\\\key{Ctrl}-E			&Spread.\cr
\\\key{Ctrl}-U			&Use Count 0.\cr
\\\key{Ctrl}-L			&Landscape.\cr
\\\key{Ctrl}-K			&DVI File Fonts.\cr
\\\key{Ctrl}-O			&DVI File Info.\cr
\\\key{Ctrl}-X			&Color Fonts.\cr
\\\key{Ctrl}-G			&Grey Text.\cr
\\\key{Ctrl}-Y			&Text + Grey.\cr
\\\key{Ctrl}-J			&Fill Rules.\cr
\\\key{Ctrl}-M			&Write TFM...\cr

\EndDblList

\newpage

\title{MouseTools}Mouse Tools.

\altsec{meastool}The Measurement Tool.

\hb{DVIWindo}\ has a tool for measuring horizontal and vertical distances in
the document.  To activate it, hold down the `\key{Shift}' key and press
the left mouse button.  Dragging the mouse creates a rectangle, the
width and height of which are indicated in the units selected by the
user in the menu \ref{Ruler Units}{RuleUnits}\ of the 
\hb{Preferences}\ menu.
   \bigskip
The rectangle can be moved - instead of resized - by also holding down the
right mouse button - just as with the rectangle used to define a
\ref{zoom area}{Zoom}\ and the rectangle used to define an area to be
\ref{copied}{CopyClip}\ to the clipboard.
   \bigskip
   
% \altsec{Handy}The Gripping Hand Icon.
\altsec{Handy}The Gripping Hand Cursor.

There is a very simple method of positioning the page on the screen
using the mouse.  Press `\key{Alt}' and then depress the left mouse
button. A ``hand'' will appear.  This ``hand'' will grip the page,
and as you move the mouse, the electronic page will move in tandem with
the mouse movement.  If \hb{DVIWindo}\ does not refresh the screen properly,
be sure to double-click the left mouse in the \hb{DVIWindo}\ display screen,
this forces \hb{DVIWindo}\ to refresh the screen.
   
	\medskip

Continued on next page\ellipses.

\newpage

% \altsec{MagGlass}The Magnifying Glass Icon.
\altsec{MagGlass}The Magnifying Glass Cursor.

In order to magnify a rectangular portion of the screen using the mouse,
press `\key{Ctrl}' and then depress the left mouse button.  A ``magnifying
glass'' will appear.  With `\key{Ctrl}' and left mouse button still
pressed, as you move the mouse around, a rectangular region is created
on screen.  When you release the left mouse button, the region defined
will be magnified enough to fill the entire screen.
   \bigskip
   
% \altsec{Scissors}The Scissors Icon.
\altsec{Scissors}The Scissors Cursor.

To copy a rectangular region of the screen to Window's clipboard, hold
down the `\key{Shift}' and `\key{Ctrl}' keys, press the left mouse
button at one corner of the rectangle -- the cursor changes to a
scissors -- drag to the opposite corner of the rectangle -- then release
the mouse button.  To move the rectangle while it is being defined, hold
down the right mouse button as well as the left.  To copy the whole page
to the clipboard, simply bring the cursor back to where you started --
\hb{DVIWindo}\ copies everything to the clipboard if the rectangle is very
small.
   \medskip
Th region copied to the clipboard can be `pasted' into another
application.  The pasted region acts as a graphical object that can be
moved, scaled and cropped.  Typeset math using \TeX, then import into
your favorite WYSIWYG application. 
   \bigskip


\newpage

\title{Tour}A Short Tour of \hb{DVIWindo}.

\parskip\bigskipamount

The purpose of this tutorial is to give the first-time user a very quick
introduction to the most basic features of \hb{DVIWindo}\ for getting
around.  There is quite a lot of information contained in this help
file, but it is not complete; when desperate, read the manual.
   \bigskip

\altsec{MoveAround}Moving Around Within a Page.

On the keyboard, you can move around within a page by pressing the arrow
keys. The `\key{Arrow up}' moves the page up a little, `\key{Arrow Down}'
moves the page down a little (`\key{Arrow Left}' and `\key{Arrow Right}'
move the page left and right a little). If you depress the `\key{Shift}'
key while you press one of the arrow keys, then the amount of change
(up, down, left, or right) is increased. The mouse can be used to
perform the same functions.  By clicking the left mouse button on either
the vertical or horizontal slide bar, the page moves vertically or
horizontally; pressing `\key{Shift}' before clicking the left mouse
button increases the magnitude of the vertical or horizontal shift.

A handy method of positioning the page to the desired location is pressing
% `\key{Shift}-Left Mouse button.' % David Carlisle
`\key{Alt}-Left Mouse button.'
A hand will appear that will ``grip'' the page, and the page will move
in tandem with the mouse movement.

You can move to the top of the page by pressing `\key{Ctrl-N}' and to by
the bottom of the page by pressing `\key{Ctrl-B}'.

	\medskip

Continued on next page\ellipses

\newpage

\altsec{Paginate}Moving Between Consecutive Pages.

Needless to say, the keys `\key{Page Down}' and `\key{Page Up}', can be used
to move between consecutive pages; the pair `\key{Spacebar}' and
`\key{Backspace}' perform basically the same function -- but with an
important difference.  Using the mouse, simple click the menu item labeled
\hb{Previous!}\ or \hb{Next!}.  The behavior of the `\key{Page Down/Page Up}'
pair, and the `\hb{Previous!/Next!}'\ pair can be modified using the
\key{Ctrl}-key, as explained below.

The menu item \ref{Reset Scale Each Page}{ResetPage}\ under the
\hb{Preferences}\ menu controls the actions of \hb{DVIWindo}\ in two
ways concerning pagination:

\BeginBulletList
   \Item the position of the new page on the screen (i.e.\ the page offset);
   \Item the magnification of the new page.
\EndBulletList

When \ref{Reset Scale Each Page}{ResetPage}\ is {\bf Checked},
\hb{DVIWindo}\ is functioning in its `top/bottom of page' mode, that is,
when you page forward, \hb{DVIWindo}\ will move to the {\it top\/} of
the next page, and when you page backward, \hb{DVIWindo}\ will move to
the {\it bottom\/} of the previous page.  This mode is nice for reading
through a document.  If \ref{Reset Scale Each Page}{ResetPage}\ is {\bf
UnChecked}, the offset of the page remains unchanged as you go forward
or backward in the document.

% Continued on next page\ellipses
 
% \newpage

\hb{DVIWindo}\ has a `user-defined default magnification', which can be
set by the \ref{Store New Page Scale}{StoreScale}\ under the 
\hb{File}\ menu.   Now when \ref{Reset Scale Each Page}{ResetPage}\ is {\bf
Checked}, and you paginate, \hb{DVIWindo}\ will reset the magnification
scale to the current default (in addition to moving to the top or bottom
of the page); conversely, when \hb{Reset Scale Each Page}\ {\bf
UnChecked}, the magnification scale is left unchanged.

	\medskip

Continued on next page\ellipses

\newpage

The behavior described above can be modified on the keyboard as
described in the paragraphs that follow. 

Begin by using the `\key{Page Down/Page Up}' combination to paginate. 
First, open the \hb{Preferences}\ menu, and make sure \ref{Reset Scale
Each Page}{ResetPage}\ is {\bf Checked}.  Now, press `\key{Page Up}',
\hb{DVIWindo}\ will move to the {\it top of the next page}; pressing
`\key{Page Down}' will force DVIWindo to move to the {\it bottom of the
previous page}.  This `top/bottom of the page' behavior is modified by
the \key{Ctrl}-key.  While holding `\key{Ctrl}' down, now press 
`\key{Page Down}' or `\key{Page Up}'; 
now, \hb{DVIWindo}\ does not change the offsetof the page as you page.

Now open the \hb{Preferences}\ menu, and make sure \ref{Reset Scale Each
Page}{ResetPage}\ is {\bf UnChecked}.  Now, as you press `\key{Page Down}'
or `\key{Page Up}', DVIWindo will not reset the page offset; consequently,
offsets within the page will not change as you paginate.  Depressing and
holding down `\key{Ctrl}' will toggle the behavior of `\key{Page Down/Page
Up}' back to the `top/bottom of the page' behavior.

Using the mouse to click on \hb{Previous!}\ or \hb{Next!}\ is functionally
equivalent to the `\key{Page Down/Page Up}' key combination.  The `\key{Ctrl}'
key may be used to temporarily toggle the setting of \ref{Reset Scale Each
Page}{ResetPage}\ for that mouse click.

	\medskip

Continued on next page\ellipses

\newpage

Contrast the behavior of the `\key{Page Down/Page Up}' key combination
with that of the `\key{Spacebar/Backspace}' combination.  Use
`\key{Spacebar}' to advance the page, and `\key{Backspace}' to retreat a
page.  You'll notice that when you advanced to the next page using
`\key{Spacebar}', the positioning on the page did not change (e.g., if you
were half way down the page, then after the `\key{Spacebar}' you are now 
half way down the next page. `\key{Backspace}' works similarly -- the
offset on the page does not change as you paginate.  The
`\key{Spacebar/Backspace}' combination is uneffected by the setting of
\ref{Reset Scale Each Page}{ResetPage}

\newpage

\altsec{NonCont}Move between Noncontiguous Pages.

In order to move to a specific page, either choose \ref{Select
Page\ellipses}{SelPage}\ in the \hb{File}\ menu using the mouse, or use the
accelerator key `\asterisk', on the numeric key pad.  A small dialog
box will appear, enter the desired page number.  There is a third, even
faster method of moving to a specified page: Enter a number (with
NumLock {\bf ON} if entered on the key pad) followed by `\key{Enter}', no
dialog box will appear, \hb{DVIWindo}\ will move to the page entered.

Should you press `\key{Home}', \hb{DVIWindo}\ will change pages to the
{\it first page} of the document.  Similarly, pressing `\key{End}' causes
\hb{DVIWindo}\ to change to the {\it last page}.
   \bigskip

\altsec{CenterPg}Centering the Page on the Screen.

The page can be centered on the page horizontally by pressing `\key{Ctrl}'
followed by either the `\key{Left Arrow}' or `\key{Right Arrow}' key.  To
center the page vertically, press the `\key{Ctrl}' key followed by
either the `\key{Up Arrow}' or `\key{Down Arrow}' key.

\newpage

%% the glossary may want to be re-organized
%% alphabetical order?
%% ordered based on function: 
%% print drivers, compilers, macro packages, utilities?

\parskip0pt

\title{Glossary}Glossary of Terms.

\altsec{DVIfile}DVI-File.

A \hb{DVI}-file is the type of file that \hb{DVIWindo}\ reads and
displays on the screen for viewing.  A \hb{DVI}-file, which is an
acronym for {\it\hb{\underbar{d}}e\hb{\underbar{v}}ice
\hb{\underbar{i}}ndependent}, is the output file of the \ref{\TeX}{TeX}\
compiler.  A \hb{DVI}-file may be identified by its {\it extension\/} of
`\key{.dvi}'.  For example, the file name of this help file is
% `\key{dvi\_help.dvi}'. % \char95 
`\key{\jobname.dvi}'. % 95/April/19
   \bigskip

The \hb{DVI}-file may also be printed, using such programs as
\ref{DVIPSONE}{DVIPSONE}\ or \hb{DVIPS}, yielding a high quality 
typeset manuscript.
   \bigskip

\altsec{DVIPSONE}DVIPSONE.

The program \hb{DVIPSONE}, distributed by \hb{{\Y&Y}, Inc.}, produces
resolution-independent PostScript (PS) output from \ref{DVI}{DVI}\ input.
DVIPSONE can be called from the \ref{Print}{Print}\ 
dialog in \hb{DVIWindo} or used directly from the DOS command line.
   \bigskip

\altsec{TeX}The \TeX\ compiler.

The \hb{\TeX}\ compiler, written by Donald E. Knuth, is a computer program
for typesetting technical manuscripts.  The input of the \hb{\TeX}\
compiler is an ASCII-based manuscript, and the output of the \hb{\TeX}\
compiler is a \ref{DVI}{DVIfile}-file.  The ultimate reference on how to
write a manuscript acceptable to the \hb{\TeX}\ compiler is {\sl The
\TeX{}book} by Donald E. Knuth.
   \bigskip
   
\altsec{iniTeX}Ini\TeX.

\hb{ini\TeX}\ is a `raw' % `primitive' 
version of \ref{\TeX}{TeX}.  Its purpose
is to create format files.  Format files are binary versions of a
collection of macros, such as plain \TeX, \LaTeX, or AmS\TeX, that the
\hb{\TeX}\ compiler is able to load and read at high speed.

	\medskip

Continued on next page\ellipses

\newpage

% \altsec{LaTeX}\LaTeX.	% need bold sc A

\altsec{LaTeX}\LaTeXbf.	% need bold sc A

\hb{\LaTeX}\ is a popular macro package containing many features to aid
the writing of books, technical documents, and manuals.  The
original reference manual for \LaTeX\ is % {\sl \LaTeX} % need slant sc A 
\LaTeXsl\ by Leslie Lamport.
   \bigskip

\altsec{AmSTeX}\AmSTeX.

The macro package \hb{\AmSTeX{}}\ was written by Michael Spivak under the
sponsorship of the American Mathematical Society.  The package includes
additional macros for building complex mathematical structures,
additional fonts, and symbols.  A complete reference for \AmSTeX\ is
{\sl The Joy of \TeX}, written by Michael Spivak.
   \bigskip

\altsec{CleanUp}The CleanUp Utility.

The \hb{CleanUp}\ utility, supplied by \hb{{\Y&Y}, Inc.}, closes any
inactive DOS boxes.
   \bigskip
   
\altsec{SysSeg}The SysSeg Utility.

The \hb{SysSeg}\ utility, supplied by \hb{{\Y&Y}, Inc.}, gives useage
readings on the `GDI' and `USER' data segments
(In Windows 95 use \hb{System Info} from the \hb{Help} menu instead).
   \bigskip

\newpage

\parskip\medskipamount

\title {Index}Index of Basic Functions.

Moving Around \ref{Within a Page}{MoveAround}.

Moving Between \ref{Consecutive Pages}{Paginate}.

Moving Between \ref{Noncontiguous Pages}{NonCont}.

\ref{Centering}{CenterPg}\ the Page on the Screen.

\ref{Zooming in}{MagGlass}.

\ref{Measuring}{meastool}\ distances.


\end
