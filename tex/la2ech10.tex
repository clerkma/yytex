% Copyright 2007 TeX Users Group.
% You may freely use, modify and/or distribute this file.
\documentclass{article}

%%%%%% start section 10.3.1

\usepackage{epic}

\usepackage{eepic}

\begin{document}

\setlength{\unitlength}{1mm}
\begin{picture}(175,20)
\put(0,0){\line(4,1){80}}
\put(40,10){\circle*{5}}
\put(40,10){\circle{10}}
\end{picture}

\setlength{\unitlength}{1mm}
\begin{picture}(150,15)(0,0)
\thicklines
\dottedline{2}(0,10)(70,10)
\dottedline[$\bullet$]{3}(0,5)(70,5)
\dottedline[$\diamond$]{4}(0,0)(70,0)
\end{picture}

\vskip .2in

\begin{picture}(70,22)(0,-2)
\dashline{3}[0.7](0,18)(63,18)
\thicklines
\dashline{3}(0,13)(63,13)
\dashline[-30]{3}(0,8)(63,8)
\dashline[+15]{3}(0,4)(63,4)
\dashline[+30]{3}(0,0)(63,0)
\end{picture}

% \quad\multiputlist(0,0)(50,0){1.00,1.25,1.50,1.75,2.00}
\quad\multiputlist(0,0)(20,0){1.00,1.25,1.50,1.75,2.00}

\vskip .2in

\setlength{\unitlength}{1mm}
\begin{picture}(62,32)(-1,-1)
\thicklines
\matrixput(0,0)(10,0){7}(0,10){4}{\circle{2}}
\matrixput(10,0)(20,0){3}(0,20){2}{\circle*{2}}
\matrixput(0,10)(20,0){4}(0,20){2}{\circle*{2}}
\matrixput(1,0)(10,0){6}(0,10){4}{\line(1,0){8}}
\matrixput(0,1)(10,0){7}(0,10){3}{\line(0,1){8}}
\end{picture}

\vskip .2in

\quad
\put(0,0){\grid(50,100)(5,10)}
\put(0,0){\tiny\grid(100,100)(5,5)[-50,0]} % make numbers \tiny

\vskip .2in

\end{document}

%%%%%% start section 10.2.6

% \usepackage{curves}

\usepackage{curvesls}

\begin{document}

\thinlines
\setlength{\unitlength}{1mm}
\begin{picture}(80,46)(0,0)
% 1 mm dashes interspersed with 2 mm gaps
  \curvedashes[1mm]{0,1,2}
  \put(10,0){\curve(0,0, 19.6,39.2, 39.2, 0)}
  \curvedashes{} % reset dashes
  \newcounter{time}
  \curvesymbol{\textsf{\thetime\,s}
	\addtocounter{time}{1}}
  \put(10,4){\curve[-2](0,0, 19.6,39.2, 39.2,0)}
  \curvesymbol{\phantom{\circle*{1}}\circle*{1}}
  \put(10,0){\curve[-2](0,0, 19.6,39.2, 39.2,0)}
\end{picture}

\vskip .2in

\setlength{\unitlength}{1.8mm}
\begin{picture}(40,30)
   \thicklines
   \multiput(20,5)(20,12){2}{%
   \line(0,-1){2}\line(-5,3){20}}
   \multiput(20,5)(-20,12){2}{\line(5,3){20}}
   \put(20,3){\line(5,3){20}}
   \put(20,3){\line(-5,3){20}}
   \put(0,15){\line(0,1){2}}
   \linethickness{1pt}
   \put(20,5){%
      \renewcommand{\xscale}{1}
      \renewcommand{\xscaley}{-1}
      \renewcommand{\yscale}{0.6}
      \renewcommand{\yscalex}{0.6}
      \scaleput(10,10){\bigcircle{10}}
      \put(0,-2){%
	\scaleput(10,10){\arc(5,0){121}}
	\scaleput(10,10){\arc(5,0){-31}}} 
    }
\end{picture}

\newcommand{\RAFsixE}{%
 \scaleput(1.25,1.25){\arc(0,-1.25){-135}}
 \scaleput(0,0){\curve(0.36,2.13,
  1.25,3.19, 2.5,4.42, 5.0,6.10, 7.5,7.24,
  10,8.09, 15,9.28, 20,9.90, 30,10.3,
  40,10.22, 50,9.80, 60,8.98, 70,7.70,
  80,5.91, 90,3.79, 95,2.58, 99.24,1.52)}
 \scaleput(99.24,0.76){\arc(0,-0.76){180}}
 \scaleput(0,0){\curve(1.25,0, 99.24,0)}
}

% \begin{center}
 \begin{picture}(100,20)
  \RAFsixE
 \end{picture}
% \end{center}

\vskip .2in

\begin{picture}(120,50)(0,0)
  \renewcommand{\xscale}{0.9848}
  \renewcommand{\xscaley}{0.1736}
  \renewcommand{\yscale}{0.9848}
  \renewcommand{\yscalex}{-0.1736}
  \put(20,30){\RAFsixE}
  \thicklines
  \put(10,15){\vector(1,0){20}}
\end{picture}

\vskip .2in

\setlength{\unitlength}{0.4pt}
\linethickness{0.2mm}
\quad\arc(100, 0){270}
\bigcircle{100}

\vskip .2in

\setlength{\unitlength}{0.4pt}
\linethickness{0.2mm}


\setlength{\unitlength}{0.4pt}
\linethickness{0.7mm}
\begin{picture}(400,110)(-10,0)
	\tagcurve(80,0, 0,0, 40,100, 80,0, 0,0)
	\closecurve(150,0, 190,100, 230,0)
	\curve(300,0, 340,100, 380,0)
\end{picture}

\vskip .2in

\end{document}

%%%%%% start section 10.2.1

\begin{document}
\setlength{\unitlength}{1mm}
\begin{picture}(50,30)
\linethickness{1pt}
\bezier{20}(0,0)(10,30)(50,30)
\bezier{200}(0,0)(40,0)(50,30)
\end{picture}

\end{document}

%%%%%% start section 10.1.3

\usepackage{fancybox}

\begin{document}

\fbox{This is an fbox}

\shadowbox{This is a shadowbox}

\doublebox{This is a doublebox}

\ovalbox{This is an ovalbox}

\Ovalbox{ This is an Ovalbox}

\newenvironment{Boxedminipage}%
{\begin{Sbox}\begin{minipage}}%
{\end{minipage}\end{Sbox}\fbox{\TheSbox}}
\begin{Boxedminipage}{5cm}
An example of a boxed minipage
defined using the Sbox command.
\end{Boxedminipage}

\end{document}

%%%%%% start section 10.1.2

\usepackage{shadow}
% \usepackage{fancybox} % Note: \shadowbox is in fancybox

\begin{document}

\hsize=3in

% \sboxrule \sboxsep \sdim

\noindent
A standard framed box \fbox{framed text},
then a shadow box
% \shadowbox{framed text with shadow}.
\shabox{framed text with shadow}.
\par\bigskip
% \renewcommand{\shadowdim}{1.5\fboxsep}
 \renewcommand{\sdim}{1.5\sboxsep}
% \shadowbox{\parbox{6cm}{%
\shabox{\parbox{6cm}{%
A complete paragraph can be highlighted
by putting it in a parbox, nested
% inside a \texttt{framebox}.}}
inside a \texttt{shabox}.}}

\end{document}

%%%%%% start section 10.1.1

\usepackage{boxedminipage}  % Note: on IBM PC: \usepackage{boxedmin}

\begin{document}

\begin{boxedminipage}[t]{5cm}
This is an example of a small 
boxed minipage\footnote{Very simple
example} which moreover has a footnote.
\end{boxedminipage}

\end{document}

