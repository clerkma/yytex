% *** *** *** *** *** *** *** *** *** *** *** *** *** *** *** *** *** ***
% Copyright (C) 1996, 1997, 1998 Y&Y, Inc.
% Copyright (C) 2007 TeX Users Group.
% You may freely use, modify and/or distribute this file.
%       
% *** *** *** *** *** *** *** *** *** *** *** *** *** *** *** *** *** ***

% *** *** *** *** *** *** *** *** *** *** *** *** *** *** *** *** *** ***
%
%       emplain.tex             Version 1.2             1998 June 5
%
%       This TeX macro header file is used for replacing CM fonts with the
%       EM fonts from Y&Y, Inc. in Adobe Type 1 format.
%
%       \input emplain.tex  at the top of your TeX source file.
%       This countermands declarations in plain.tex which set up the CM fonts.
%
%       This is for use with plain TeX --- for LaTeX 2.09 use `emlatex.tex'
%       instead.  For LaTeX 2e, instead run `em.ins' on `em.dtx' from
%       (read em.txt for details) and then \usepackage{em}.
%
% NOTE: Loading many new fonts on top of a predefined format may cause
%       some implementations of TeX to run out of space for fonts.
%       You may wish to create a new `TeX format' in that case using 
%       `ini TeX' (or use a `big' TeX, or better yet a `dynamic' TeX).
%
%		updated 1998/06/03 by bnb for use with tugboat.sty
%
% *** *** *** *** *** *** *** *** *** *** *** *** *** *** *** *** *** *** %

% We assume this is loaded on top of an existing plain TeX format
% We try and avoid loading too many unnecessary fonts

\font\tenrm=emr10 % roman text
% \font\preloaded=emr9
% \font\preloaded=emr8
\font\sevenrm=emr7
% \font\preloaded=emr6
\font\fiverm=emr5

\font\teni=emmi10 % math italic
% \font\preloaded=emmi9
% \font\preloaded=emmi8
\font\seveni=emmi7
% \font\preloaded=emmi6
\font\fivei=emmi5

\font\tensy=cmsy10 % math symbols
% \font\preloaded=cmsy9
% \font\preloaded=cmsy8
\font\sevensy=cmsy7
% \font\preloaded=cmsy6
\font\fivesy=cmsy5

\font\tenex=cmex10 % math extension
\font\sevenex=cmex7
% \font\fiveex=cmex7 at 5pt

% \font\preloaded=emss10 % sans serif
% \font\preloaded=emssq8

% \font\preloaded=emssi10 % sans serif italic
% \font\preloaded=emssqi8

\font\tenbf=embx10 % boldface extended
% \font\preloaded=embx9
% \font\preloaded=embx8
\font\sevenbf=embx7
% \font\preloaded=embx6
\font\fivebf=embx5

\font\tentt=emtt10 % typewriter
% \font\preloaded=emtt9
% \font\preloaded=emtt8

% \font\preloaded=emsltt10 % slanted typewriter

\font\tensl=emsl10 % slanted roman
% \font\preloaded=emsl9
% \font\preloaded=emsl8

\font\tenit=emti10 % text italic
% \font\preloaded=emti9
% \font\preloaded=emti8
% \font\preloaded=emti7

% More random fonts

% \font\preloaded=emu10 % unslanted text italic
\font\tenuit=emu10

% \font\preloaded=emmib10 % bold math italic
% \font\preloaded=cmbsy10 % bold math symbols

% \font\preloaded=emcsc10 % caps and small caps
\font\tensmc=emcsc10

% \font\preloaded=emssbx10 % sans serif bold extended

\chardef\atcode=\catcode`\@	% save catcode of atsign
\catcode`\@=11			% make atsign `letter'
\ifx\tugstyloaded@\relax
% \newfam\sectitlefam                  % family used in tugboat.sty
\font\seventeenssb=emssbx10 scaled \magstep3
\font\twelvessb=emssbx10 scaled \magstep1
\textfont\sectitlefam=\seventeenssb 
\scriptfont\sectitlefam=\twelvessb
\fi
\catcode`\@=\atcode		% restore original catcode of atsign

% \font\preloaded=emdunh10 % Dunhill style

% Additional \preloaded fonts can be specified here.
% (And those that were \preloaded above can be eliminated.)

\let\preloaded=\undefined % preloaded fonts must be declared anew later.

\skewchar\teni=127 \skewchar\seveni=127 \skewchar\fivei=127
\skewchar\tensy=45 \skewchar\sevensy=45 \skewchar\fivesy=45

% *** *** *** *** *** *** *** *** *** *** *** *** *** *** *** *** *** *** %

% Pick up the math font family numbers defined by plain TeX already

% For TUB, comment out the following...

\textfont0=\tenrm \scriptfont0=\sevenrm \scriptscriptfont0=\fiverm
% \def\rm{\fam\z@\tenrm}
\textfont1=\teni \scriptfont1=\seveni \scriptscriptfont1=\fivei
% \def\mit{\fam\@ne} \def\oldstyle{\fam\@ne\teni}
\textfont2=\tensy \scriptfont2=\sevensy \scriptscriptfont2=\fivesy
% \def\cal{\fam\tw@}
% \textfont3=\tenex \scriptfont3=\sevenex \scriptscriptfont3=\fiveex
\textfont3=\tenex \scriptfont3=\tenex \scriptscriptfont3=\tenex
% \newfam\itfam \def\it{\fam\itfam\tenit} % \it is family 4
\textfont\itfam=\tenit
% \newfam\slfam \def\sl{\fam\slfam\tensl} % \sl is family 5
\textfont\slfam=\tensl
% \newfam\bffam \def\bf{\fam\bffam\tenbf} % \bf is family 6
\textfont\bffam=\tenbf 
        \scriptfont\bffam=\sevenbf \scriptscriptfont\bffam=\fivebf
% \newfam\ttfam \def\tt{\fam\ttfam\tentt} % \tt is family 7
\textfont\ttfam=\tentt

% For use in TUB, load them through call to size collection instead:
% \tenpoint
% Which can be done below, after remainder of fonts defined.

% *** *** *** *** *** *** *** *** *** *** *** *** *** *** *** *** *** *** %

% Set up some additional sizes.  If you need to conserve TeX memory 
% (or gain some speed), then comment out the rarely used sizes marked with %*

% math italic
\font\eighteeni=emmi12 at 18pt  %*
\font\fourteeni=emmi12 at 14pt  %*
\font\twelvei=emmi12            %*
\font\eleveni=emmi12 at 10.95pt %*
%\font\teni=emmi10
\font\ninei=emmi9               %*
\font\eighti=emmi8              %*
%\font\seveni=emmi7
\font\sixi=emmi6                %*
%\font\fivei=emmi5

% math symbols
\font\eighteensy=cmsy10 at 18pt %*
\font\fourteensy=cmsy10 at 14pt %*
\font\twelvesy=cmsy10 at 12pt   %*
\font\elevensy=cmsy10 at 10.95pt        %*
%\font\tensy=cmsy10
\font\ninesy=cmsy9              %*
\font\eightsy=cmsy8             %*
%\font\sevensy=cmsy7
\font\sixsy=cmsy6               %*
%\font\fivesy=cmsy5

% math extension
\font\eighteenex=cmex10 at 18pt %*
\font\fourteenex=cmex10 at 14pt %*
\font\twelveex=cmex10 at 12pt           %*
\font\elevenex=cmex10 at 10.95pt        %*
%\font\tenex=cmex10

% roman text
\font\eighteenrm=emr17 at 18pt  %*
\font\fourteenrm=emr12 at 14pt  %*
\font\twelverm=emr12
\font\elevenrm=emr10 at 10.95pt %*
%\font\tenrm=emr10
\font\ninerm=emr9
\font\eightrm=emr8
%\font\sevenrm=emr7
\font\sixrm=emr6                %*
%\font\fiverm=emr5

% text italic
\font\eighteenit=emti12 at 18pt %*
\font\fourteenit=emti12 at 14pt %*
\font\twelveit=emti12 at 12pt   %*
\font\elevenit=emti10 at 10.95pt                %*
%\font\tenit=emti10
\font\nineit=emti9
\font\eightit=emti8             %*

% boldface extended
\font\eighteenbf=embx12 at 18pt %*
\font\fourteenbf=embx12 at 14pt %*
\font\twelvebf=embx12
\font\elevenbf=embx10 at 10.95pt                %*
%\font\tenbf=embx10
\font\ninebf=embx9
\font\eightbf=embx8             %*
%\font\sevenbf=embx7
\font\sixbf=embx6               %*
%\font\fivebf=embx5

% typewriter
\font\eighteentt=emtt12 at 18pt %*
\font\fourteentt=emtt12 at 14pt %*
\font\twelvett=emtt12           %*
\font\eleventt=emtt10 at 10.95pt        %*
%\font\tentt=emtt10
\font\ninett=emtt10 at 9pt      %*
\font\eighttt=emtt8 at 8pt      %*

% slanted roman
\font\eighteensl=emsl12 at 18pt %*
\font\fourteensl=emsl12 at 14pt %*
\font\twelvesl=emsl12
\font\elevensl=emsl10 at 10.95pt        %*
%\font\tensl=emsl10
\font\ninesl=emsl9              %*
\font\eightsl=emsl8             %*

% If we do define math fonts here we would need to also set \skewchar:

\skewchar\eighteeni=127 \skewchar\fourteeni=127 \skewchar\twelvei=127
\skewchar\eleveni=127 \skewchar\teni=127 \skewchar\ninei=127
\skewchar\eighti=127 \skewchar\seveni=127 \skewchar\sixi=127
\skewchar\fivei=127
\skewchar\eighteensy=48 \skewchar\fourteensy=48 \skewchar\twelvesy=48
\skewchar\elevensy=48 \skewchar\tensy=48 \skewchar\ninesy=48
\skewchar\eightsy=48 \skewchar\sevensy=48 \skewchar\sixsy=48
\skewchar\fivesy=48

% Can now load TUB fonts here:
% \tenpoint

% *** *** *** *** *** *** *** *** *** *** *** *** *** *** *** *** *** *** %

% The following are the standard plain TeX defaults for CM

% \delimiterfactor=901
% \delimitershortfall=5pt
% \nulldelimiterspace=1.2pt
% \scriptspace=0.5pt
% \thinmuskip=3mu
% \medmuskip=4mu plus 2mu minus 4mu
% \thickmuskip=5mu plus 5mu

% *** *** *** *** *** *** *** *** *** *** *** *** *** *** *** *** *** *** % 

% Draw small radical from CMEX (do this ONLY if CMEX exist in three sizes)

% \def\sqrt{\radical"39F370 }   % \def\sqrt{\radical"270370 }

% *** *** *** *** *** *** *** *** *** *** *** *** *** *** *** *** *** *** %

% The following depends on text font encoding so needs to be taken
% care of separately by something like \input texnansi.tex

% \chardef\%=`\%
% \chardef\&=`\&
% \chardef\#=`\#
% \chardef\$=`\$
% \chardef\ss=25
% \chardef\ae=26
% \chardef\oe=27
% \chardef\o=28
% \chardef\AE=29
% \chardef\OE=30
% \chardef\O=31
% \chardef\i=16 \chardef\j=17 % dotless letters
% \def\aa{\accent23a}
% \def\l{\char32l}
% \def\L{\leavevmode\setbox0\hbox{L}\hbox to\wd0{\hss\char32L}}

% \def\aa{^^e5} \def\AA{^^c5}           % aring (229), Aring (197)
% \def\l{^^90} \def\L{^^80}             % lslash (144), Lslash (128)

% The EM fonts have a real underline

% \def\_{\leavevmode \kern.06em \vbox{\hrule width.3em}}
% \def\_{^^5F}                          % underline (95)

% \def\AA{\leavevmode\setbox0\hbox{h}\dimen@\ht0\advance\dimen@-1ex%
%   \rlap{\raise.67\dimen@\hbox{\char'27}}A}

% We could pull section, paragraph, dagger, daggerdbl from text font

% \def\mathhexbox#1#2#3{\leavevmode
%   \hbox{$\m@th \mathchar"#1#2#3$}}
% \def\dag{\mathhexbox279}
% \def\ddag{\mathhexbox27A}
% \def\S{\mathhexbox278}
% \def\P{\mathhexbox27B}

% \def\dag{^^86} \def\ddag{^^87}        % dagger (134), daggerdbl (135)
% \def\S{^^a7}  \def\P{^^b6}            % section (167), paragraph (182)

% Uncomment the following standard plain TeX definitions for TUB

% \def\`#1{{\accent18 #1}}
% \def\'#1{{\accent19 #1}}
% \def\^#1{{\accent94 #1}} \let\^^D=\^
% \def\~#1{{\accent126 #1}}
% \def\=#1{{\accent22 #1}}
% \def\v#1{{\accent20 #1}} \let\^^_=\v
% \def\u#1{{\accent21 #1}} \let\^^S=\u
% \def\"#1{{\accent127 #1}}
% \def\.#1{{\accent95 #1}}
% \def\H#1{{\accent125 #1}}
% \def\t#1{{\edef\next{\the\font}\the\textfont1\accent127\next#1}}

% following four lines are for TeX n ANSI encoding only
% \def\.##1{{\accent5 ##1}}%     dotaccent      % not in 95
% \def\H##1{{\accent6 ##1}}%     hungarumlaut   % not in 125
% \def\k##1{{\accent7 ##1}}%     ogonek         % not in CM fonts
% \def\"##1{{\accent168 ##1}}%   dieresis       % alternate to 127

% Lower case Greek letters in smae place as in CM

% \mathchardef\Gamma="7000
% \mathchardef\Delta="7001
% \mathchardef\Theta="7002
% \mathchardef\Lambda="7003
% \mathchardef\Xi="7004
% \mathchardef\Pi="7005
% \mathchardef\Sigma="7006
% \mathchardef\Upsilon="7007
% \mathchardef\Phi="7008
% \mathchardef\Psi="7009
% \mathchardef\Omega="700A

% The upright upper case Greek letters are in the math italic font

\mathchardef\Gamma="01D0
\mathchardef\Delta="01D1
\mathchardef\Theta="01D2
\mathchardef\Lambda="01D3
\mathchardef\Xi="01D4
\mathchardef\Pi="01D5
\mathchardef\Sigma="01D6
\mathchardef\Upsilon="01D7
\mathchardef\Phi="01D8
\mathchardef\Psi="01D9
\mathchardef\Omega="01DA

% \def\acute{\mathaccent"7013 }
% \def\grave{\mathaccent"7012 }
% \def\ddot{\mathaccent"707F }
% \def\tilde{\mathaccent"707E }
% \def\bar{\mathaccent"7016 }
% \def\breve{\mathaccent"7015 }
% \def\check{\mathaccent"7014 }
% \def\hat{\mathaccent"705E }
% \def\vec{\mathaccent"017E }
% \def\dot{\mathaccent"705F }

% \def\ddot{\mathaccent"70A8 }  % 168 dieresis

% \normalbaselines\rm % select roman font
% \nonfrenchspacing % punctuation affects the spacing

% The following not needed for TUB, taken care of in \tenpoint above.
\rm

% *** *** *** *** *** *** *** *** *** *** *** *** *** *** *** *** *** *** %

% AMS TeX with amsppt.sty:

% For use with amsppt.sty, need the following fonts (make sure uncommented):

% \eightbf, \eightit, \eightsl, \eighttt, \sevenit, \sixrm, \sixbf, 
% \eighti, \sixi, \eightsy, \sixsy

% *** *** *** *** *** *** *** *** *** *** *** *** *** *** *** *** *** *** %

% Note that plain TeX has the accent character positions hardwired to:

% 16 for `dotlessi', 17 for `dotlessj',
% 18 for `grave', 19 for `acute', 20 for `caron',
% 21 for `breve', 22 for `macron',
% 23 for `ring', 24 for `cedilla',
% 25 for `germandbls', 26 for `ae', 27 for `oe',
% 28 for `oslash', 29 for `AE', 30 for 'OE', 31 for `Oslash',
% 94 for `circumflex', 95 for `dotaccent', 125 for `hungarumlaut',
% 126 for `tilde', 127 for `dieresis',
% (see page 356 of the TeX book, and plain.tex for additional information)

% These should be adjusted - if these characters are to be used -
% AND if the text fonts are encoded to something other than TeX text

% \input texnansi.tex if you are using `TeX n ANSI' encoding
% \input ansiacce.tex if you are using Windows ANSI encoding
% \input dcaccent.tex if you are using Cork (T1) encoding
